% Page setup
\documentclass[a5paper,10pt]{article}
\pagenumbering{gobble}

% Margin setup
\usepackage{geometry}
\geometry{left=1cm}
\geometry{right=1cm}
\geometry{top=1cm}
\geometry{bottom=1.5cm}

% %For inserting tables
% \usepackage{array}

% Formula aligning
\usepackage{amsmath}

% For formula diacritics
%\usepackage{amsfonts}

% For additional symbols
\usepackage{amssymb}

% For bigger integral signs
\usepackage{bigints}

% For appropriate vector signings
%\usepackage{esvect}

% For semantic advancements
%\usepackage{enotez}

% For degree sign
% \usepackage{gensymb}

%For pictures
\usepackage{graphicx}

%For framing
\usepackage{framed}

% Automatic paragraph indentations
\usepackage{indentfirst}
\setlength{\parindent}{1em}

% Formula spacing configuration
\delimitershortfall-1sp
\usepackage{mleftright}
\mleftright

% For resizing
\usepackage{relsize}

%For svg support
%\usepackage[inkscapelatex=false]{svg}

%For underlining
\usepackage{ulem}

% For floaty text
%\usepackage{wrapfig}

% Extra math fonts
\usepackage{yhmath}

% For cyrillic characters support
\usepackage[english, russian]{babel}

% For proper font
\usepackage{fontspec}

% TNR within formulas
\usepackage{mathptmx}

% TNR within text
\setmainfont{Times New Roman}

% Automatic brackets placement
\newcommand{\br}[1]{\left(#1\right)}
\newcommand{\vbr}[1]{\left|#1\right|}
\newcommand{\cbr}[1]{\left\{#1\right\}}
\newcommand{\rbr}[1]{\left[#1\right]}
\renewcommand{\le}{\leqslant}
\renewcommand{\ge}{\geqslant}

% Redefinition of some operators (to the appropriation of Russian notation)
\renewcommand{\Re}{\operatorname{Re}}
\renewcommand{\Im}{\operatorname{Im}}

% Enfore fractures to be full-sized everywhere
% \renewcommand{\frac}{\dfrac}

% Special operator definitions
% \DeclareMathOperator{\dis}{\rho}
% \DeclareMathOperator{\dif}{\d}

% Theorem definitions
\usepackage{amsthm}
\theoremstyle{definition}
% \newtheorem*{theorem}{Теорема}
% \newtheorem*{corollary}{Следствие}
% \newtheorem*{criteria}{Критерий}
% \newtheorem*{algorithm}{Алгоритм}
\newtheorem*{formula}{Формула}
% \newtheorem*{remark}{Замечание}
% \newtheorem*{proposition}{Предложение}

% Restrict inlne formula breaking
\binoppenalty=10000 
\relpenalty=10000 

\begin{document}

\begin{framed}
\begin{formula}[вспомогательного угла]
$$A \sin{x} \pm B \cos{x} = \sqrt{A^2 + B^2} \sin\br{x \pm \phi},$$

где $\phi$ называется вспомогательным углом таким, что

$$
\sin{\phi} = \frac {B}{\sqrt{A^2 + B^2}},\ 
\cos{\phi} = \frac {A}{\sqrt{A^2 + B^2}},\
\tg{\phi} = \frac {B}{A}.
$$
В частности, можно взять $\phi = \arctg\br{\frac {B}{A}}$.
\end{formula}
\end{framed}


Уравнение силы тока $I = I\br{t}$ будет иметь вид

$$R I\br{t} + L I'\br{t} + \frac{1}{C} q\br{t} = V \sin\br{\omega t},$$

где $q\br{t}$ -- заряд конденсатора.

Взяв производную, получим

$$LC I''\br{t} + RC I\br{t} + I\br{t} = VC\omega \cos\br{\omega t}.$$

Характеристическое уравнение имеет вид $LC \lambda^{2} + RC \lambda + \lambda = 0$. Вне зависимости от знака дискриминанта по теореме Виета $\lambda_{1} + \lambda_{2} = 1 > 0$, но $\lambda_{1} + \lambda_{2} = -RC < 0$. Следовательно, корни этого уравнения таковы, что $\Re{\lambda_{1}} < 0$ и $\Re{\lambda_{2}} < 0$. Последнее означает, что оба элемента фундаметнальной системы решений будут стремиться к нулю со скоростью, сопоставимой с $e^{\mathlarger{-\frac {R}{2L}} \displaystyle{t}}$ (в случае кратных корней $t e^{\mathlarger{-\frac {R}{2L}} \displaystyle{t}}$).

Итак, установившийся режим не вытекает из решения однородного уравнения, значит он полностью совпадает с частным решением неоднородного, которое в соответствии с видом правой части имеет вид

$$I_{1}\br{t} = a \sin\br{\omega t} + b \cos\br{\omega t},$$

где $a$, $b$ -- искомые коэффициенты.
\begin{align*}
    I'_{1}\br{t} = a \omega \cos\br{\omega t} - b \omega \sin\br{\omega t} && I''_{1}\br{t} = -a \omega^2 \sin\br{\omega t} - b \omega^2 \cos\br{\omega t}.
\end{align*}
\begin{equation*}
    \begin{cases}
        \br{-LC\omega^2 + 1} a - RC \omega b = 0 \\
        RC \omega a + \br{-LC\omega^2 + 1} b = VC\omega 
    \end{cases}
\end{equation*}
\begin{align*}
    \Delta_{0} = \br{-LC\omega^2 + 1}^2 + \br{RC\omega}^2 && a = \frac {VRC^2\omega^2} {\Delta_{0}} && b = \frac {\br{1 - LC\omega^2}VC\omega} {\Delta_{0}}
\end{align*}

Подставив, получаем
$$I\br{t} = \frac{1}{\Delta_{0}} VC\omega \br{CR\omega \sin\br{\omega t} + \br{1 - LC\omega^2} \cos\br{\omega t}}.$$

Со вспомогательным углом $\phi = \arctan\br{\dfrac{LC\omega^2 - 1}{CR\omega}}$ имеем установившийся режим $I_{1}\br{t}$ в следующей форме:
$$I_{1} = \frac {VC\omega} {\sqrt{\Delta_{0}}} \sin\br{\omega t - \phi} = \frac {V} {\sqrt{R^2 + \br{L\omega - \dfrac {1}{C\omega}}^2}} \sin\br{\omega t - \phi}.$$

Амплитуда $A = A\br{\omega} = \displaystyle{\frac {V} {\sqrt{R^2 + \br{L\omega - \dfrac {1}{C\omega}}^2}} }$ силы тока тем больше, чем меньше $\br{L\omega - \dfrac {1}{C\omega}}^2$. Наибольшая амплитуда достигается при $\omega = \pm \dfrac{1}{\sqrt{LC}}$ и равна $\dfrac {V} {R}$.

\end{document}