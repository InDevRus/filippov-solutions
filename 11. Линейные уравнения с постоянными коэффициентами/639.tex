% Page setup
\documentclass[a5paper,10pt]{article}
\pagenumbering{gobble}

% Margin setup
\usepackage{geometry}
\geometry{left=1cm}
\geometry{right=1cm}
\geometry{top=1cm}
\geometry{bottom=1.5cm}

% %For inserting tables
% \usepackage{array}

% Formula aligning
\usepackage{amsmath}

% For formula diacritics
%\usepackage{amsfonts}

% For additional symbols
\usepackage{amssymb}

% For bigger integral signs
\usepackage{bigints}

% For appropriate vector signings
%\usepackage{esvect}

% For semantic advancements
%\usepackage{enotez}

% For degree sign
% \usepackage{gensymb}

%For pictures
\usepackage{graphicx}

%For framing
\usepackage{framed}

% Automatic paragraph indentations
\usepackage{indentfirst}
\setlength{\parindent}{1em}

% Formula spacing configuration
\delimitershortfall-1sp
\usepackage{mleftright}
\mleftright

% For resizing
\usepackage{relsize}

%For svg support
%\usepackage[inkscapelatex=false]{svg}

%For underlining
\usepackage{ulem}

% For floaty text
%\usepackage{wrapfig}

% Extra math fonts
\usepackage{yhmath}

% For cyrillic characters support
\usepackage[english, russian]{babel}

% For proper font
\usepackage{fontspec}

% TNR within formulas
\usepackage{mathptmx}

% TNR within text
\setmainfont{Times New Roman}

% Automatic brackets placement
\newcommand{\br}[1]{\left(#1\right)}
\newcommand{\vbr}[1]{\left|#1\right|}
\newcommand{\cbr}[1]{\left\{#1\right\}}
\newcommand{\rbr}[1]{\left[#1\right]}
\renewcommand{\le}{\leqslant}
\renewcommand{\ge}{\geqslant}

% Enfore fractures to be full-sized everywhere
% \renewcommand{\frac}{\dfrac}

% Special operator definitions
% \DeclareMathOperator{\dis}{\rho}
% \DeclareMathOperator{\dif}{\d}

% Theorem definitions
\usepackage{amsthm}
\theoremstyle{definition}
% \newtheorem*{theorem}{Теорема}
% \newtheorem*{corollary}{Следствие}
% \newtheorem*{criteria}{Критерий}
% \newtheorem*{algorithm}{Алгоритм}
\newtheorem*{formula}{Формула}
% \newtheorem*{remark}{Замечание}
% \newtheorem*{proposition}{Предложение}

% Restrict inlne formula breaking
\binoppenalty=10000 
\relpenalty=10000 

\begin{document}

\begin{framed}
\begin{formula}[вспомогательного угла]
$$A \sin{x} \pm B \cos{x} = \sqrt{A^2 + B^2} \sin\br{x \pm \phi},$$

где $\phi$ называется вспомогательным углом таким, что

$$
\sin{\phi} = \frac {B}{\sqrt{A^2 + B^2}},\ 
\cos{\phi} = \frac {A}{\sqrt{A^2 + B^2}},\
\tg{\phi} = \frac {B}{A}.
$$
В частности, можно взять $\phi = \arctg\br{\frac {B}{A}}$.
\end{formula}
\end{framed}

\begin{framed}
$$\int \sin\br{at} e^{bt} dt = \frac {e^{bt}}{a^2 + b^2} \br{b \sin\br{ax} - a \cos\br{ax}} + C$$
\end{framed}

Уравнение силы тока $I\br{t}$ имеет вид $LI'\br{t} + RI\br{t} = V \sin\br{\omega t}$. 
$I'\br{t} + \frac {R} {L} I\br{t} = \linebreak = \frac {V}{L} \sin\br{\omega t}$.
$I'\br{t} e^{\mathlarger{\frac{R}{L} t}} + \frac {R} {L} I\br{t} e^{\mathlarger{\frac{R}{L} t}} = \frac {V}{L} e^{\mathlarger{\frac{R}{L} t}} \sin\br{\omega t}$. 
$\br{I\br{t} e^{\mathlarger{\frac{R}{L} t}}}' = \frac {V}{L}  \sin\br{\omega t} e^{\mathlarger{\frac{R}{L} t}}$.

$I\br{t} e^{\mathlarger{\frac{R}{L} t}} = \bigintsss{\frac {V}{L} \sin\br{\omega t} e^{\mathlarger{\frac{R}{L} t}} dt} = \dfrac {V}{L^2 \omega^2 + R^2} e^{\mathlarger{\frac {R}{L}t}} \br{R \sin\br{\omega t} - L \omega \cos\br{\omega t}} + C$.

$$I\br{t} = I_{0}\br{t} + I_{1}\br{t} = C e^{-\mathlarger{\frac {R}{L}t}} + \dfrac {V}{L^2 \omega^2 + R^2} \br{R \sin\br{\omega t} - L \omega \cos\br{\omega t}}.$$

Из вида общего решения видно, что оно при $t \to +\infty$ стремится к установившемуся режиму 
$$ I_{1}\br{t} = \dfrac {V}{L^2 \omega^2 + R^2} \br{R \sin\br{\omega t} - L \omega \cos\br{\omega t}}. $$

Приведя к виду со вспомогательным углом $\phi = \arctan\br{\dfrac{L\omega}{R}}$, получим

$$ I_{1}\br{t} = \frac {V}{\sqrt{L^2 \omega^2 + R^2}} \sin\br{\omega t - \phi}. $$


\end{document}