\documentclass[a5paper,10pt]{article}

% https://github.com/InDevRus/filippov-solutions
% Page setup
\pagenumbering{gobble}

% Margin setup
\usepackage{geometry}
\geometry{left=1cm}
\geometry{right=1cm}
\geometry{top=1cm}
\geometry{bottom=1.5cm}

% For inserting tables
\usepackage{array}

% Formula aligning
\usepackage{amsmath}

% For formula diacritics
\usepackage{amsfonts}

% Theorem definitions
\usepackage{amsthm}
\theoremstyle{remark}
\newtheorem*{necessity}{Необходимость}
\newtheorem*{sufficiency}{Достаточность}
\theoremstyle{definition}
\newtheorem*{theorem}{Теорема}
\newtheorem*{lemma}{Лемма}
\newtheorem*{corollary}{Следствие}
\newtheorem*{criteria}{Критерий}
\newtheorem*{algorithm}{Алгоритм}
\newtheorem*{formula}{Формула}
\newtheorem*{remark}{Замечание}
\newtheorem*{proposition}{Предложение}

% For bigger integral signs
\usepackage{bigints}

% For semantic advancements
\usepackage{enotez}

% For pictures
\usepackage{graphicx}

%For framing
\usepackage{framed}

% Automatic paragraph indentations
\usepackage{indentfirst}
\setlength{\parindent}{1em}

% Formula spacing configuration
\delimitershortfall-1sp
\usepackage{mleftright}
\mleftright

% For resizing
\usepackage{relsize}

% More convenient text-style common notations
\DeclareMathOperator*\lowlim{\underline{\lim}}
\DeclareMathOperator*\uplim{\overline{\lim}}

\newcommand{\tlowlim}[1]{\lowlim\limits_{#1}}
\newcommand{\tuplim}[1]{\uplim\limits_{#1}}

\newcommand{\tpow}[2]{{#1}^{\mathlarger{#2}}}

\newcommand{\tint}{\displaystyle{\int}}
\newcommand{\tintlim}[2]{\displaystyle{\int\limits_{#1}^{#2}}}
\newcommand{\tbigint}{\displaystyle{\mathlarger{\int}}}
\newcommand{\tbigintlim}[2]{\displaystyle{\mathlarger{\int}\limits_{#1}^{#2}}}

\newcommand{\tsumlim}[2]{\displaystyle{\mathlarger{\sum}\displaylimits_{#1}^{#2}}}
\newcommand{\tprodlim}[2]{\displaystyle{\mathlarger{\prod}\displaylimits_{#1}^{#2}}}

\newcommand{\tmin}[1]{\min\limits_{#1}}
\newcommand{\tmax}[1]{\max\limits_{#1}}
\newcommand{\tlim}[1]{\lim\limits_{#1}}

\newcommand{\norm}[1]{\left\lVert#1\right\rVert}

% For floaty text
\usepackage{wrapfig}
\usepackage{floatflt}

% For cyrillic characters support
\usepackage[english, russian]{babel}

% For proper font
\usepackage[no-math]{fontspec}

% TNR within text
\setmainfont{Times New Roman}

% TNR within formulas
\usepackage{newtxmath}
\DeclareSymbolFont{operators}{OT1}{ntxtlf}{m}{n}
\SetSymbolFont{operators}{bold}{OT1}{ntxtlf}{b}{n}

% For degree sign
\usepackage{siunitx}

% Automatic brackets placement
\newcommand{\br}[1]{\left(#1\right)}
\newcommand{\vbr}[1]{\left|#1\right|}
\newcommand{\cbr}[1]{\left\{#1\right\}}
\newcommand{\rbr}[1]{\left[#1\right]}
\renewcommand{\le}{\leqslant}
\renewcommand{\ge}{\geqslant} 

% Automatic replacement for two greek letters
\renewcommand{\epsilon}{\varepsilon}
\renewcommand{\phi}{\varphi}

% Redefinition of some operators (to the appropriation of Russian notation)
\DeclareMathOperator{\arcsh}{arcsh}
\DeclareMathOperator{\arcch}{arcch}
\DeclareMathOperator{\arcth}{arcth}
\DeclareMathOperator{\arccth}{arccth}
\DeclareMathOperator{\rank}{rank}
\DeclareMathOperator{\inv}{inv}
\DeclareMathOperator{\sgn}{sgn}
\renewcommand{\Re}{\operatorname{Re}}
\renewcommand{\Im}{\operatorname{Im}}

\renewcommand{\gcd}{\text{НОД}}
\newcommand{\lcm}{\text{НОК}}

% Restrict inlne formula breaking
\binoppenalty=10000 
\relpenalty=10000

% Greek letters setup
% Old greek letters setup
\DeclareSymbolFont{old_letters}{OML}{ztmcm}{m}{it}
\SetSymbolFont{old_letters}{bold}{OML}{ztmcm}{b}{it}

\newcommand{\Alpha}{\text{A}}
\newcommand{\Beta}{\text{B}}
\newcommand{\Epsilon}{\text{E}}
\newcommand{\Zeta}{\text{Z}}
\newcommand{\Eta}{\text{H}}
\newcommand{\Iota}{\text{I}}
\newcommand{\Kappa}{\text{K}}
\newcommand{\Mu}{\text{M}}
\newcommand{\Nu}{\text{N}}
\newcommand{\Omicron}{\text{O}}
\newcommand{\Rho}{\text{P}}
\newcommand{\Tau}{\text{T}}
\newcommand{\Chi}{\text{X}}

\DeclareMathSymbol{\alpha}{\mathord}{old_letters}{11}
\DeclareMathSymbol{\beta}{\mathord}{old_letters}{12}
\DeclareMathSymbol{\gamma}{\mathord}{old_letters}{13}
\DeclareMathSymbol{\delta}{\mathord}{old_letters}{14}
\DeclareMathSymbol{\varepsilon}{\mathord}{old_letters}{15}
\DeclareMathSymbol{\zeta}{\mathord}{old_letters}{16}
\DeclareMathSymbol{\eta}{\mathord}{old_letters}{17}
\DeclareMathSymbol{\theta}{\mathord}{old_letters}{18}
\DeclareMathSymbol{\iota}{\mathord}{old_letters}{19}
\DeclareMathSymbol{\kappa}{\mathord}{old_letters}{20}
\DeclareMathSymbol{\lambda}{\mathord}{old_letters}{21}
\DeclareMathSymbol{\mu}{\mathord}{old_letters}{22}
\DeclareMathSymbol{\nu}{\mathord}{old_letters}{23}
\DeclareMathSymbol{\xi}{\mathord}{old_letters}{24}
\newcommand{\omicron}{\text{\textit{\larger[1]{o}}}}
\DeclareMathSymbol{\pi}{\mathord}{old_letters}{25}
\DeclareMathSymbol{\rho}{\mathord}{old_letters}{26}
\DeclareMathSymbol{\sigma}{\mathord}{old_letters}{27}
\DeclareMathSymbol{\tau}{\mathord}{old_letters}{28}
\DeclareMathSymbol{\upsilon}{\mathord}{old_letters}{29}
\DeclareMathSymbol{\varphi}{\mathord}{old_letters}{39}
\DeclareMathSymbol{\chi}{\mathord}{old_letters}{31}
\DeclareMathSymbol{\psi}{\mathord}{old_letters}{32}
\DeclareMathSymbol{\omega}{\mathord}{old_letters}{33}






\usepackage{pgfplots}
\pgfplotsset{compat=newest}
\pgfplotsset{
    every axis/.append style = {
        xlabel={$x$},
        ylabel={$y$}, 
        axis lines = middle,
        unit vector ratio = 1 1,
        font=\small
    },
    every axis plot/.append style = {smooth, samples = 100}
}

\begin{document}

Найдём траектории этой системы.
$\dfrac {dy} {dx} = \dfrac {\dot{y}} {\dot{x}} = \dfrac {x^3 \br{1 + y^2}} {y}$.
$\dfrac {2y y'_{x}\br{x}} {1 + y^2} = 2x^3$.
$\ln\br{1 + y^2} \linebreak = \dfrac {1} {2} x^4 + C$.
$y^2 = C \tpow{e}{\frac {1} {2} x^4} - 1$. Эти траектории определены только при $C > 0$.

\begin{floatingfigure}{0.4\textwidth}
\begin{tikzpicture}
        \pgfplotsset{
        VectorGraph/.style = {
            -stealth,
            quiver = {
                u = y / (x^2 + y^2) ^ 0.5, 
                v = x^3 * (1 + y^2) / (x^2 + y^2) ^ 0.5, 
                scale arrows = 0.1}
            }
        }
        \begin{axis}[
            width = \textwidth,
            height = 9.5cm,
            ymin = -3.4,
            ymax = 3.4,
            xmin = -1.9,
            xmax = 1.8, 
            xtick = {-3, ..., 3},
            ytick = {-5, ..., 5},
            minor tick num = 3,
            declare function = {
                f(\C, \x) = sqrt(\C * exp(0.5 * x^4) - 1);
            }]
            
            \addplot[domain = -2:2, thick, gray] {f(1, x)};
            \addplot[domain = -2:2] {f(1.5, x)};
            \addplot[domain = -2:2] {f(2, x)};
            \addplot[domain = -2:2] {-f(1, x)};
            \addplot[domain = -2:2] {-f(1.5, x)};
            \addplot[domain = -2:2] {-f(2, x)};
            
            \addplot[domain = 1.75:ln(4)^(1/4), samples = 2000]
            {f(0.5, x)};
            \addplot[domain = 1.75:ln(4)^(1/4), samples = 2000]
            {-f(0.5, x)};
            
            \addplot[domain = -1.75:-ln(4)^(1/4), samples = 2000]
            {f(0.5, x)};
            \addplot[domain = -1.75:-ln(4)^(1/4), samples = 2000]
            {-f(0.5, x)};
            
            \addplot[VectorGraph, samples = 6, domain = -1.2 : 1.2, gray]
            {f(1, x)};
            \addplot[VectorGraph, samples = 6, domain = -1.2 : 1.2]
            {f(1.5, x)};
            \addplot[VectorGraph, samples = 6, domain = -1.2 : 1.2]
            {f(2, x)};
            \addplot[VectorGraph, samples = 6, domain = -1.2 : 1.2]
            {-f(1, x)};
            \addplot[VectorGraph, samples = 6, domain = -1.2 : 1.2]
            {-f(1.5, x)};
            \addplot[VectorGraph, samples = 6, domain = -1.2 : 1.2]
            {-f(2, x)};
            
            \addplot[VectorGraph, samples = 6, domain = ln(4)^(1/4) : 1.5]
            {f(0.5, x)};
            \addplot[VectorGraph, samples = 6, domain = ln(4)^(1/4) : 1.5]
            {-f(0.5, x)};
            
            \addplot[VectorGraph, samples = 6, domain = -1.6 : -ln(4)^(1/4)]
            {f(0.5, x)};
            \addplot[VectorGraph, samples = 6, domain = -1.5 : -ln(4)^(1/4)]
            {-f(0.5, x)};
        \end{axis}
    \end{tikzpicture}
\end{floatingfigure}

Направления траекторий изображены на графике. Характер траекторий показывает, что с ростом $t$ точка
$\left\{ \begin{matrix} x = x\br{t} \\ y = y\br{t}\end{matrix} \right.$
удаляется от начала координат, поэтому решение неустойчиво.

Для доказательство неустойчивости будем использовать только траекторию $y = \sqrt{\tpow{e}{\frac {1}{2} x^4} - 1}$. \linebreak Она представляет особый интерес, так как проходит через начало координат, то есть через нулевое решение
$\Phi\br{t} = \begin{pmatrix}0 \\ 0 \end{pmatrix}$. Это значит, что для всякого $\delta > 0$ можно подобрать точку I-ой четверти на этой траектории, лежащую в окрестности $O\br{\br{0; 0}; \delta}$ начала координат радиуса $\delta$. Эта траектория изображена на фазовой плоскости серым цветом.

Для всякого решения $X\br{t} = \begin{pmatrix} x\br{t} \\ y\br{t} \end{pmatrix}$, составляющего такую траекторию, имеем при $t \ge t_{0} = 0$
\begin{gather*}
\vbr{X\br{t} - \Phi\br{t}} 
= \sqrt{x^2\br{t} + y^2\br{t}} 
= \sqrt{x^2\br{t} + \tpow{e}{\frac {1}{2} x^4\br{t}} - 1} \\
\vbr{X\br{t_{0}} - \Phi\br{t_{0}}} = \sqrt{x^2\br{0} + \tpow{e}{\frac {1}{2} x^4\br{0}} - 1}
\end{gather*}

Пусть $\epsilon = \sqrt{\dfrac {3} {2}}$ и $\delta > 0$. Приведём для них пример такого решения, для которого $\sqrt{x^2\br{0} + \tpow{e}{\frac {1}{2} x^4\br{0}} - 1} < \delta$, но при некоторых $t$ будет $\sqrt{x^2\br{t} + \tpow{e}{\frac {1}{2} x^4\br{t}} - 1} \ge \epsilon$. 

Рассмотрим следующую задачу Коши:
$$\begin{cases}
\dfrac {d} {dt} \begin{pmatrix}x\br{t} \\ y\br{t}\end{pmatrix} = \begin{pmatrix} y \\  x^3 \br{1 + y^2} \end{pmatrix}\\
x\br{0} = \min\br{1; \gamma} \\
y\br{0} = \sqrt{ \tpow{e}{\frac {1}{2} x^{4}\br{0}} - 1}
\end{cases}$$
где $\gamma = \sqrt{\ln\br{\dfrac {\delta^2} {2} + 1}}$. Она удовлетворяет условию единственности, причём поскольку начальная точка фазовой плоскости $A_{0} = \br{x\br{0}; y\br{0}}$ лежит в I-ой четверти, $\frac {dx} {dt}\br{0} > 0$ и $\frac {dy} {dt}\br{0} > 0$. Это означает, что в окрестности точки $A_{0}$ обе функции $x\br{t}$ и $y\br{t}$ возрастают. С другой стороны, при $t \ge t_{0}$ точка остаётся на траектории, а значит и в I-ой четверти, а значит обе производные остаются положительными, а функции неограниченно возрастают.

$0 < x\br{0} \le 1$, значит $x^2\br{0} \ge x^{4}\br{0}$. Для всех положительных $a$ выполняется неравенство $a < e^{a} - 1$, а значит
\begin{multline*}
\sqrt{x^2\br{0} + \tpow{e}{\frac {1}{2} x^4\br{0}} - 1} 
< \sqrt{\tpow{e}{x^2\br{0}} - 1 + \tpow{e}{\frac {1}{2} x^2\br{0}} - 1} 
< \sqrt{2\br{\tpow{e}{x^2\br{0}} - 1}} \le \\
\le \sqrt{2\br{\tpow{e}{\gamma^2} - 1}} = \delta 
\end{multline*}

Найдём такое $t_{1} > t_{0}$, что для $t \ge t_{1}$ будет выполняться $x\br{t} \ge 1$. Тогда для таких $t$ верно, что $x^4\br{t} \ge x^2\br{t}$ и
$$
\sqrt{x^2\br{t} + \tpow{e}{\frac {1}{2} x^4\br{t}} - 1} 
\ge \sqrt{x^2\br{t} + \tpow{e}{\frac {1}{2} x^2\br{t}} - 1} 
> \sqrt{x^2\br{t} + \dfrac {1} {2} x^2\br{t}} 
= \sqrt{\dfrac {3} {2} x^2\br{t}} 
\ge \sqrt{ \dfrac {3} {2} } = \epsilon,
$$
в чём и требовалось убедиться. 
\end{document}
