\documentclass[a5paper,10pt]{article}

% https://github.com/InDevRus/filippov-solutions
% Page setup
\pagenumbering{gobble}

% Margin setup
\usepackage{geometry}
\geometry{left=1cm}
\geometry{right=1cm}
\geometry{top=1cm}
\geometry{bottom=1.5cm}

% For inserting tables
\usepackage{array}

% Formula aligning
\usepackage{amsmath}

% For formula diacritics
\usepackage{amsfonts}

% Theorem definitions
\usepackage{amsthm}
\theoremstyle{definition}
\newtheorem*{theorem}{Теорема}
\newtheorem*{corollary}{Следствие}
\newtheorem*{criteria}{Критерий}
\newtheorem*{algorithm}{Алгоритм}
\newtheorem*{formula}{Формула}
\newtheorem*{remark}{Замечание}
\newtheorem*{proposition}{Предложение}

% For bigger integral signs
\usepackage{bigints}

% For semantic advancements
\usepackage{enotez}

% For pictures
\usepackage{graphicx}

%For framing
\usepackage{framed}

% Automatic paragraph indentations
\usepackage{indentfirst}
\setlength{\parindent}{1em}

% Formula spacing configuration
\delimitershortfall-1sp
\usepackage{mleftright}
\mleftright

% For resizing
\usepackage{relsize}
% More convenient text-style common notations
\newcommand{\tpow}[2]{{#1}^{\mathlarger{#2}}}
\newcommand{\tint}{\displaystyle{\int}}
\newcommand{\tintlim}[2]{\displaystyle{\int\limits_{#1}^{#2}}}
\newcommand{\tbigint}{\displaystyle{\mathlarger{\int}}}
\newcommand{\tbigintlim}[2]{\displaystyle{\mathlarger{\int}\limits_{#1}^{#2}}}
\newcommand{\tsumlim}[2]{\displaystyle{\mathlarger{\sum}\displaylimits_{#1}^{#2}}}
\newcommand{\tprodlim}[2]{\displaystyle{\mathlarger{\prod}\displaylimits_{#1}^{#2}}}
\newcommand{\tmin}[1]{\min\limits_{#1}}
\newcommand{\tmax}[1]{\max\limits_{#1}}
\newcommand{\tlim}[1]{\lim\limits_{#1}}

% For floaty text
\usepackage{wrapfig}
\usepackage{floatflt}

% For cyrillic characters support
\usepackage[english, russian]{babel}

% For proper font
\usepackage[no-math]{fontspec}

% TNR within text
\setmainfont{Times New Roman}

% TNR within formulas
\usepackage{newtxmath}
\DeclareSymbolFont{operators}{OT1}{ntxtlf}{m}{n}
\SetSymbolFont{operators}{bold}{OT1}{ntxtlf}{b}{n}

% For degree sign
\usepackage{siunitx}

% Automatic brackets placement
\newcommand{\br}[1]{\left(#1\right)}
\newcommand{\vbr}[1]{\left|#1\right|}
\newcommand{\cbr}[1]{\left\{#1\right\}}
\newcommand{\rbr}[1]{\left[#1\right]}
\renewcommand{\le}{\leqslant}
\renewcommand{\ge}{\geqslant} 

% Automatic replacement for two greek letters
\renewcommand{\epsilon}{\varepsilon}
\renewcommand{\phi}{\varphi}

% Redefinition of some operators (to the appropriation of Russian notation)
\DeclareMathOperator{\arcsh}{arcsh}
\DeclareMathOperator{\arcch}{arcch}
\DeclareMathOperator{\arcth}{arcth}
\DeclareMathOperator{\arccth}{arccth}
\DeclareMathOperator{\rank}{rank}
\DeclareMathOperator{\inv}{inv}
\renewcommand{\Re}{\operatorname{Re}}
\renewcommand{\Im}{\operatorname{Im}}

\renewcommand{\gcd}{\text{НОД}}
\newcommand{\lcm}{\text{НОК}}

% Restrict inlne formula breaking
\binoppenalty=10000 
\relpenalty=10000

% Greek letters setup
\usepackage{textalpha}

\newcommand{\zed}{\textit{z}}

\newcommand{\Alpha}{\text{A}}
\newcommand{\Beta}{\text{B}}
\newcommand{\Epsilon}{\text{E}}
\newcommand{\Zeta}{\text{Z}}
\newcommand{\Eta}{\text{H}}
\newcommand{\Iota}{\text{I}}
\newcommand{\Kappa}{\text{K}}
\newcommand{\Mu}{\text{M}}
\newcommand{\Nu}{\text{N}}
\newcommand{\Omicron}{\text{O}}
\newcommand{\Rho}{\text{P}}
\newcommand{\Tau}{\text{T}}
\newcommand{\Chi}{\text{X}}

\renewcommand{\alpha}{\text{\textalpha}}
\renewcommand{\beta}{\text{\textbeta}}
\renewcommand{\gamma}{\text{\textgamma}}
\renewcommand{\delta}{\text{\textdelta}}
\renewcommand{\epsilon}{\text{\textepsilon}}
\renewcommand{\zeta}{\text{\textzeta}}
\renewcommand{\eta}{\text{\texteta}}
\renewcommand{\theta}{\text{\texttheta}}
\renewcommand{\iota}{\text{\textiota}}
\renewcommand{\kappa}{\text{\textkappa}}
\renewcommand{\lambda}{\text{\textlambda}}
\renewcommand{\mu}{\text{\textmu}}
\renewcommand{\nu}{\text{\textnu}}
\renewcommand{\xi}{\text{\textxi}}
\newcommand{\omicron}{\text{\textomicron}}
\renewcommand{\pi}{\text{\textpi}}
\renewcommand{\rho}{\text{\textrho}}
\renewcommand{\sigma}{\text{\textsigma}}
\renewcommand{\tau}{\text{\texttau}}
\renewcommand{\upsilon}{\text{\textupsilon}}
\renewcommand{\phi}{\text{\textphi}}
\renewcommand{\chi}{\text{\textchi}}
\renewcommand{\psi}{\text{\textpsi}}
\renewcommand{\omega}{\text{\textomega}}






\begin{document}

Любая задача Коши вида 
$\tsys{& \ddot {x}\br{t} + p\br{t} x\br{t} = 0
\\& x\br{t_{0}} = \alpha
\\& \dot{x}\br{t_{0}} = \beta}$
имеет единственное продолжаемое на всю числовую прямую решение даже в точках $t_{0} = 2\pi n$, $n \in \mathbb{Z}$ разрыва функции $p\br{t}$. Действительно, и в левой, и в правой полуокрестностях всякой такой точки $t_{0}$ задача Коши будет удовлетворять условиям теоремы Пикара и иметь там единственное решение. Если соединить полученные, вообще говоря, разные функции в левой и правой полуокрестностях, то получится итоговое непрерывно дифференцируемое решение задачи Коши, вторая производная которого может быть разрывной там, где разрывна $p\br{t}$. Процесс также можно продолжить и посчитать значения $x\br{t}$ и $\dot{x}\br{t}$ в ближайшей точке разрыва $p\br{t}$ и составить из неё новую задачу Коши, решение которой также будет единственным и непрерывно дифференцируемым.

Перейдём к равносильной системе $\tsys{& \dot{x}\br{t} = u\br{t} \\& \dot{u}\br{t} = -p\br{t} x\br{t}}$, или в матричной форме 
$\dfrac {d} {dt}
\begin{pmatrix} x \\ u \end{pmatrix}
\linebreak = A\br{t}
\begin{pmatrix} x \\ u \end{pmatrix}$, где 
$A\br{t} = \begin{pmatrix}0 & 1 \\ -p\br{t} & 0 \end{pmatrix}$. Искать фундаментальную матрицу решений будем из $X\br{\Omicron_{2}} = E_{2}$. Искомая матрица монодромии $C = C_{2 \times 2}$ такова, что 
$X\br{t + 2\pi} = X\br{t} C$ для всех $t$. В частности, $X\br{\pi} = X\br{-\pi} C$ и $C = \br{X\br{-\pi}}^{-1} X\br{\pi}$.

$X\br{t} = \begin{pmatrix}x_{1}\br{t} & x_{2}\br{t} \\ u_{1}\br{t} & u_{2}\br{t}\end{pmatrix}$,
$X\br{\Omicron_{2}} = \begin{pmatrix}x_{1}\br{0} & x_{2}\br{0} \\ u_{1}\br{0} & u_{2}\br{0}\end{pmatrix} = \begin{pmatrix}1 & 0 \\ 0 & 1 \end{pmatrix}$. Условимся без ограничения общности, что $a \ge 0$ и $b \ge 0$.
Решив эти две задачи Коши на промежутке $t \in \rbr{-\pi; \pi}$, получаем
\begin{align*}
    &
    x_{1}\br{t} = \tsys{
    & \cos\br{bt}\text{, если } t \in \rbr{-\pi; 0} 
    \\& \cos\br{at}\text{, если } t \in \left(0; \pi\right] } 
    && 
    x_{2}\br{t} = \tsys{
    & g_{b}\br{t} \text{, если } t \in \rbr{-\pi; 0} 
    \\& g_{a}\br{t} \text{, если } t \in \left(0; \pi\right] } 
    \\
    &
    u_{1}\br{t} = \tsys{
    & -b\sin\br{bt}\text{, если } t \in \rbr{-\pi; 0} 
    \\& -a\sin\br{at}\text{, если } t \in \left(0; \pi\right] }
    &&
    u_{2}\br{t} = \tsys{
    & \cos\br{bt}\text{, если } t \in \rbr{-\pi; 0} 
    \\& \cos\br{at}\text{, если } t \in \left(0; \pi\right] },
\end{align*}
где $g_{s}\br{t} = \tsys{&\dfrac {1} {s} \sin\br{st} \text{, если } s > 0 \\& t \text{, если } s = 0}$ -- нечётная функция, необходимая для покрытия \linebreak случаев $a = 0$ и $b = 0$.

Пользуясь непрерывностью решений, будем находить значения этих четырёх функций в $\pi$ и $-\pi$ как соответствующие пределы при $x \to -\pi + 0$ и $x \to \pi - 0$ соответственно.
\begin{align*}
    X\br{-\pi} = 
    \begin{pmatrix}
        \cos\br{b\pi} & - g_{b}\br{\pi} 
        \\ b \sin\br{b\pi} & \cos\br{b\pi} 
    \end{pmatrix}
    &&
    X\br{\pi} =
    \begin{pmatrix}
        \cos\br{a\pi} & g_{a}\br{\pi}
        \\ -a \sin\br{a\pi} & \cos\br{a\pi}
    \end{pmatrix}
\end{align*}
\begin{align*}
    \det\br{X\br{-\pi}} = 1
    &&
    \br{X\br{-\pi}}^{-1} 
    = \adj\br{X\br{-\pi}} = \begin{pmatrix}
        \cos\br{b\pi} & g_{b}\br{\pi} 
        \\ -b \sin\br{b\pi} & \cos\br{b\pi} 
    \end{pmatrix}    
\end{align*}

Итак, матрица монодромии равняется
\begin{align*}
    C = \begin{pmatrix} 
        \cos\br{a\pi} \cos\br{b\pi} - a \sin\br{a\pi} g_{b}\br{\pi}
        &  g_{a}\br{\pi} \cos\br{b\pi} + \cos\br{a\pi} \cos\br{b\pi}
        \\
        -b \cos\br{a\pi} \sin\br{b\pi} - a \sin\br{a\pi} \cos\br{b\pi}
        & -b g_{a}\br{\pi} \sin\br{b\pi} + \cos\br{a\pi} \cos\br{b\pi}
    \end{pmatrix}.
\end{align*}

Для каждого из перечисленных далее случаев мультипликаторы $\rho_{1}$ и $\rho_{2}$ целесообразнее находить, если матрица монодромии предварительно упрощена. Факт устойчивости устанавливается по следующему критерию, формулировка которого взята из книги Ламберто Чезари <<Асимптотическое поведение и устойчивость решений обыкновенных дифференциальных уравнений>>, глава II, \S\ 4, предложение 4.1.2.

\begin{framed}
    \begin{criteria}[устойчивости по мультипликаторам] \end{criteria}
    Пусть дана линейная однородная система с периодическими коэффицентами.
    \begin{itemize}
        \item Для асимптотической устойчивости нулевого решения необходимо и достаточно, чтобы все мультипликаторы лежали внутри единичного круга $\vbr{\rho} < 1$.
        \item Для устойчивости необходимо и достаточно, чтобы
        \begin{enumerate}
            \item все мультипликаторы лежали в замкнутом единичном круге $\vbr{\rho} \le 1$;
            \item каждый мультипликатор $\rho_{k}$, лежащий на единичной окружности (т.е. $\vbr{\rho_{k}} = 1$), имел кратность как корень характеристического (векового) уравнения $\det\br{C - \rho E} = 0$, равную дефекту $\nullity\br{C - \rho_{k} E}$.
        \end{enumerate}
    \end{itemize}
\end{framed}

а) $a = \dfrac {1} {2}$, $b = 0$.
$C = \begin{pmatrix} -\frac {\pi} {2} \vspace{1mm} & 2 \\ -\frac {1} {2} & 0 \end{pmatrix}$.
$\det\br{C - \rho E} = \rho^2 + \dfrac {\pi} {2} \rho + 1$.
$2\rho^2 + \pi \rho + 2 = 0$.

$\rho_{1} = -\dfrac {\pi} {4} - \sqrt{1 - \dfrac {\pi^2} {16}}\ i$.
$\rho_{2} = -\dfrac {\pi} {4} + \sqrt{1 - \dfrac {\pi^2} {16}}\ i$.
$\vbr{\rho_{1}} = \vbr{\rho_{2}} = 1$ \vspace{2mm}, и нулевое решение устойчиво, но не асимптотически.

б) $a = \dfrac {1} {2}$, $b = 1$.
$C = \begin{pmatrix} 0 & -2 \\ \frac {1} {2} & 0 \end{pmatrix}$.
$\det\br{C - \rho E} = \rho^2 + 1$.
$\rho_{1} = -i$,
$\rho_{2} = i$.
$\vbr{\rho_{1}} = \vbr{\rho_{2}} = 1$ \vspace{2mm}, и нулевое решение устойчиво, но не асимптотически.

в) $a = \dfrac {1} {2}$, $b = \dfrac {3} {2}$.
$C = \begin{pmatrix} \frac {1} {3} & 0 \\ 0 & 3 \end{pmatrix}$.
$\det\br{C - \rho E} = \br{\rho - \dfrac {1} {3}} \br{\rho - 3}$.
$\rho_{1} = \dfrac {1} {3}$, $\rho_{2} = 3$, и нулевое решение неустойчиво.

г) $a = \dfrac {3} {4}$, $b = 0$.
$C = \begin{pmatrix} \vspace{1mm} \dfrac {-4 - 3\pi} {4\sqrt{2}} & \dfrac {1} {3\sqrt{2}} \\ -\dfrac {3} {4\sqrt{2}} & -\dfrac {1} {\sqrt{2}} \end{pmatrix}$.
$\det\br{C - \rho E} = \rho^2 + \br{\dfrac {3\pi + 8} {4\sqrt{2}}}\rho + \dfrac {3\pi + 5} {8}$. 
$\rho_{1} = \dfrac {-8 - 3\pi - \sqrt{9\pi^2 - 16}} {8\sqrt{2}}$,
$\rho_{2} = \dfrac {-8 - 3\pi + \sqrt{9\pi^2 - 16}} {8\sqrt{2}}$.
Но $\rho_{1} < \dfrac {-8 - 3 \cdot 3 - \sqrt{9 \cdot 3^2 - 16}} {8 \cdot 2} \linebreak 
= \dfrac {-17 - \sqrt{65}} {16} < \dfrac {-17 - 8} {16} = -\dfrac {25} {16} < -1$, а нулевое решение неустойчиво.

д) $a = 1$, $b = 0$. 
$C = \begin{pmatrix} -1 & -1 \\ 0 & -1 \end{pmatrix}$.
$\det\br{C - \rho E} = \br{\rho + 1}^2$.
$\rho_{1} = \rho_{2} = -1$.
$\vbr{\rho_{1}} = \vbr{\rho_{2}} = 1$.
$\nullity\br{C - \rho_{1} E} = \nullity \begin{pmatrix}0 & -1 \\ 0 & 0 \end{pmatrix} = 2 - 1 = 1$, а кратность корня равна $2$. Из-за этого нулевое решение неустойчиво.

е) $a = 1$, $b = \dfrac {3} {2}$.
$C = \begin{pmatrix} 0 & 0 \\ -\frac {3} {2} & 0 \end{pmatrix}$.
$\det\br{C - \rho E} = \rho^2$. $\rho_{1} = \rho_{2} = 0$, и решение асимптотически устойчиво.

\end{document}
