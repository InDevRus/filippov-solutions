\documentclass[a5paper,10pt]{article}

% https://github.com/InDevRus/filippov-solutions
% Page setup
\pagenumbering{gobble}

% Margin setup
\usepackage{geometry}
\geometry{left=1cm}
\geometry{right=1cm}
\geometry{top=1cm}
\geometry{bottom=1.5cm}

% For inserting tables
\usepackage{array}

% Formula aligning
\usepackage{amsmath}

% For formula diacritics
\usepackage{amsfonts}

% Theorem definitions
\usepackage{amsthm}
\theoremstyle{remark}
\newtheorem*{necessity}{Необходимость}
\newtheorem*{sufficiency}{Достаточность}
\theoremstyle{definition}
\newtheorem*{theorem}{Теорема}
\newtheorem*{lemma}{Лемма}
\newtheorem*{corollary}{Следствие}
\newtheorem*{criteria}{Критерий}
\newtheorem*{algorithm}{Алгоритм}
\newtheorem*{formula}{Формула}
\newtheorem*{remark}{Замечание}
\newtheorem*{proposition}{Предложение}

% For bigger integral signs
\usepackage{bigints}

% For semantic advancements
\usepackage{enotez}

% For pictures
\usepackage{graphicx}

%For framing
\usepackage{framed}

% Automatic paragraph indentations
\usepackage{indentfirst}
\setlength{\parindent}{1em}

% Formula spacing configuration
\delimitershortfall-1sp
\usepackage{mleftright}
\mleftright

% For resizing
\usepackage{relsize}

% More convenient text-style common notations
\DeclareMathOperator*\lowlim{\underline{\lim}}
\DeclareMathOperator*\uplim{\overline{\lim}}

\newcommand{\tlowlim}[1]{\lowlim\limits_{#1}}
\newcommand{\tuplim}[1]{\uplim\limits_{#1}}

\newcommand{\tpow}[2]{{#1}^{\mathlarger{#2}}}

\newcommand{\tint}{\displaystyle{\int}}
\newcommand{\tintlim}[2]{\displaystyle{\int\limits_{#1}^{#2}}}
\newcommand{\tbigint}{\displaystyle{\mathlarger{\int}}}
\newcommand{\tbigintlim}[2]{\displaystyle{\mathlarger{\int}\limits_{#1}^{#2}}}

\newcommand{\tsumlim}[2]{\displaystyle{\mathlarger{\sum}\displaylimits_{#1}^{#2}}}
\newcommand{\tprodlim}[2]{\displaystyle{\mathlarger{\prod}\displaylimits_{#1}^{#2}}}

\newcommand{\tmin}[1]{\min\limits_{#1}}
\newcommand{\tmax}[1]{\max\limits_{#1}}
\newcommand{\tlim}[1]{\lim\limits_{#1}}

\newcommand{\norm}[1]{\left\lVert#1\right\rVert}

% For floaty text
\usepackage{wrapfig}
\usepackage{floatflt}

% For cyrillic characters support
\usepackage[english, russian]{babel}

% For proper font
\usepackage[no-math]{fontspec}

% TNR within text
\setmainfont{Times New Roman}

% TNR within formulas
\usepackage{newtxmath}
\DeclareSymbolFont{operators}{OT1}{ntxtlf}{m}{n}
\SetSymbolFont{operators}{bold}{OT1}{ntxtlf}{b}{n}

% For degree sign
\usepackage{siunitx}

% Automatic brackets placement
\newcommand{\br}[1]{\left(#1\right)}
\newcommand{\vbr}[1]{\left|#1\right|}
\newcommand{\cbr}[1]{\left\{#1\right\}}
\newcommand{\rbr}[1]{\left[#1\right]}
\renewcommand{\le}{\leqslant}
\renewcommand{\ge}{\geqslant} 

% Automatic replacement for two greek letters
\renewcommand{\epsilon}{\varepsilon}
\renewcommand{\phi}{\varphi}

% Redefinition of some operators (to the appropriation of Russian notation)
\DeclareMathOperator{\arcsh}{arcsh}
\DeclareMathOperator{\arcch}{arcch}
\DeclareMathOperator{\arcth}{arcth}
\DeclareMathOperator{\arccth}{arccth}
\DeclareMathOperator{\rank}{rank}
\DeclareMathOperator{\inv}{inv}
\DeclareMathOperator{\sgn}{sgn}
\renewcommand{\Re}{\operatorname{Re}}
\renewcommand{\Im}{\operatorname{Im}}

\renewcommand{\gcd}{\text{НОД}}
\newcommand{\lcm}{\text{НОК}}

% Restrict inlne formula breaking
\binoppenalty=10000 
\relpenalty=10000

% Greek letters setup
% Old greek letters setup
\DeclareSymbolFont{old_letters}{OML}{ztmcm}{m}{it}
\SetSymbolFont{old_letters}{bold}{OML}{ztmcm}{b}{it}

\newcommand{\Alpha}{\text{A}}
\newcommand{\Beta}{\text{B}}
\newcommand{\Epsilon}{\text{E}}
\newcommand{\Zeta}{\text{Z}}
\newcommand{\Eta}{\text{H}}
\newcommand{\Iota}{\text{I}}
\newcommand{\Kappa}{\text{K}}
\newcommand{\Mu}{\text{M}}
\newcommand{\Nu}{\text{N}}
\newcommand{\Omicron}{\text{O}}
\newcommand{\Rho}{\text{P}}
\newcommand{\Tau}{\text{T}}
\newcommand{\Chi}{\text{X}}

\DeclareMathSymbol{\alpha}{\mathord}{old_letters}{11}
\DeclareMathSymbol{\beta}{\mathord}{old_letters}{12}
\DeclareMathSymbol{\gamma}{\mathord}{old_letters}{13}
\DeclareMathSymbol{\delta}{\mathord}{old_letters}{14}
\DeclareMathSymbol{\varepsilon}{\mathord}{old_letters}{15}
\DeclareMathSymbol{\zeta}{\mathord}{old_letters}{16}
\DeclareMathSymbol{\eta}{\mathord}{old_letters}{17}
\DeclareMathSymbol{\theta}{\mathord}{old_letters}{18}
\DeclareMathSymbol{\iota}{\mathord}{old_letters}{19}
\DeclareMathSymbol{\kappa}{\mathord}{old_letters}{20}
\DeclareMathSymbol{\lambda}{\mathord}{old_letters}{21}
\DeclareMathSymbol{\mu}{\mathord}{old_letters}{22}
\DeclareMathSymbol{\nu}{\mathord}{old_letters}{23}
\DeclareMathSymbol{\xi}{\mathord}{old_letters}{24}
\newcommand{\omicron}{\text{\textit{\larger[1]{o}}}}
\DeclareMathSymbol{\pi}{\mathord}{old_letters}{25}
\DeclareMathSymbol{\rho}{\mathord}{old_letters}{26}
\DeclareMathSymbol{\sigma}{\mathord}{old_letters}{27}
\DeclareMathSymbol{\tau}{\mathord}{old_letters}{28}
\DeclareMathSymbol{\upsilon}{\mathord}{old_letters}{29}
\DeclareMathSymbol{\varphi}{\mathord}{old_letters}{39}
\DeclareMathSymbol{\chi}{\mathord}{old_letters}{31}
\DeclareMathSymbol{\psi}{\mathord}{old_letters}{32}
\DeclareMathSymbol{\omega}{\mathord}{old_letters}{33}






\begin{document}

Тот факт, что
$x\br{t} y\br{t} = O\br{x^2\br{t} + y^2\br{t}} = O\br{o\br{\sqrt{x^2\br{t} + y^2\br{t}}}} = o\br{\vbr{X}}$ при $\vbr{X} \to 0$
был показан в решении задачи \textbf{899}. Таким образом,
\begin{align*} 
    \dfrac {d} {dt} \begin{pmatrix} x\br{t} \\ y\br{t} \end{pmatrix} = A \begin{pmatrix} x\br{t} \\ y\br{t} \end{pmatrix} + \begin{pmatrix} o\br{\vbr{X}} \\ o\br{\vbr{X}} \end{pmatrix} && A = \begin{pmatrix} a & -2 \\ 1 & 1 \end{pmatrix}
\end{align*}

$\det\br{A - \lambda E} = \br{\lambda - a}\br{\lambda - 1} + 2 = \lambda^2 - \br{a + 1} \lambda + a + 2$. Харакетристическое уравнение имеет вид:
$\lambda^2 - \br{a + 1} \lambda + a + 2 = 0$. $D = \br{a + 1}^2 - 4\br{a + 2} = a^2 - 2a - 7$.

$D \ge 0$ при $a \in \left(-\infty; 1 - \sqrt{8} \right] \cup \left[1 + \sqrt{8}; +\infty \right)$. При таких $a$ по теореме Виета $\lambda_{1} \cdot \lambda_{2} = \linebreak = a + 2$. Таким образом, если $a < 2 < 1 - \sqrt{8}$, то собственные значения разных знаков и нулевое решение неустойчиво. Если $a = -2$, то собственные значения равны $0$ и $-1$ и теорема Ляпунова не даёт ответ. Если $-2 < a \le 1 - \sqrt{8}$, то $\lambda_{1} + \lambda_{2} = a + 1 < 0$, оба корня отрицательны и нулевое решение устойчиво асимптотически. Наконец, в случае $a \ge 1 + \sqrt{8}$ будет $\lambda_{1} + \lambda_{2} = a + 1 > 0$ и оба корня положительны, то есть нулевое решение неустойчиво.

$D < 0$ при $a \in \br{1 - \sqrt{8}; 1 + \sqrt{8}}$. $\Re\br{\lambda} = \dfrac {1} {2} \br{a + 1}$, так что при $a \in \br{1 - \sqrt{8}; -1}$ решение асимптотически устойчиво, при $a \in \br{-1; 1 + \sqrt{8}}$ неустойчиво, а при $a = -1$ вещественные части обоих собственных значений равны $0$ и ответ об устойчивости получить не удастся.

Случаи $a = -2$ и $a = -1$ необходимо рассмотреть отдельно. В обоих случаях будем находить траектории.

\newtheorem{case*}{Случай}
\begin{case*} $a = -1$.

$\dfrac {dy} {dx} = \dfrac {xy + x + y} {x^2 - x - 2y}$. Это уравнение Дарбу и замена $y = x \cdot t\br{x}$ сведёт его к уравнению Бернулли.
$t + x \dfrac {dt} {dx} = \dfrac {x^2t + x + xt} {x^2 - x - 2xt} = \dfrac {xt + 1 + t} {x - 1 - 2t}$.
$x \dfrac {dt} {dx} = \dfrac {x} {x'\br{t}} = \dfrac {xt + 1 + t} {x - 1 - 2t} - t = \linebreak = \dfrac {1 + 2t + 2t^2} {x - 1 - 2t}$.
$x'\br{t} + \dfrac {1 + 2t} {2t^2 + 2t + 1} x\br{t} = \dfrac {1} {2t^2 + 2t + 1} x^2\br{t}$. $x\br{t} = \dfrac {1} {C \sqrt{2t^2 + 2t + 1} - 2t - 1}$ и общее решение (уравнение траекторий) можно записать по крайней мере в параметрической форме:
\begin{align*}
x\br{t} = \dfrac {1} {C \sqrt{2t^2 + 2t + 1} - 2t - 1} && y\br{t} = \dfrac {t} {C \sqrt{2t^2 + 2t + 1} - 2t - 1}
\end{align*}

$\tlim{\mathsmaller{x \to +\infty}} y\br{t} = \dfrac {1} {-2 + \sqrt{2} C}$ и $\tlim{\mathsmaller{x \to -\infty}} y\br{t} = \dfrac {1} {-2 - \sqrt{2} C}$, и ни один из этих пределов не равен нулю ни при каких $C$. Значит, нулевое решение этой системы, если даже и устойчиво, то устойчиво точно не асимптотически.

\end{case*}

\begin{case*}$a = -2$.

Поступим аналогично. Уравнение траекторий $\dfrac {dy} {dx} = \dfrac {xy + x + y} {x^2 - 2x - 2y}$ снова представляет собой уравнение Дарбу. $y\br{x} = x \cdot t\br{x}$.

$$
\dfrac {x} {x'\br{t}} 
= \dfrac {x^2 t + x + xt} {x^2 - 2x - 2xt} - t 
= \dfrac {xt + 1 + t} {x - 2 - 2t} - t
= \dfrac {1 + 3t + 2t^2} {x - 2 - 2t}
$$

$x'\br{t} \br{1 + 3t + 2t^2} = x \br{x - 2 - 2t}$.
$x'\br{t} + \dfrac {2} {2t + 1} x\br{t} = \dfrac {1} {\br{2t + 1}\br{t + 1}} x^2$.
Итак, уравнение траекторий имеет параметрический вид:
\begin{align*}
x\br{t} = \frac {1} {\br{2t + 1} \br{C + \ln\vbr{\dfrac {2t + 1} {2t + 2}}} + 1}
&& y\br{t} = \frac {t} {\br{2t + 1} \br{C + \ln\vbr{\dfrac {2t + 1} {2t + 2}}} + 1}
\end{align*}

Однако, $\tlim{t \to \infty} = \dfrac {1} {2C} \ne 0$, и асимптотическая устойчивость нулевого невозможна.
\end{case*}

Итак, нулевое решение решение асимптотически устойчиво только при $a \in \br{-2; -1}$.

\end{document}
