\documentclass[a5paper,10pt]{article}

% https://github.com/InDevRus/filippov-solutions
% Page setup
\pagenumbering{gobble}

% Margin setup
\usepackage{geometry}
\geometry{left=1cm}
\geometry{right=1cm}
\geometry{top=1cm}
\geometry{bottom=1.5cm}

% For inserting tables
\usepackage{array}

% Formula aligning
\usepackage{amsmath}

% For formula diacritics
\usepackage{amsfonts}

% Theorem definitions
\usepackage{amsthm}
\theoremstyle{remark}
\newtheorem*{necessity}{Необходимость}
\newtheorem*{sufficiency}{Достаточность}
\theoremstyle{definition}
\newtheorem*{theorem}{Теорема}
\newtheorem*{lemma}{Лемма}
\newtheorem*{corollary}{Следствие}
\newtheorem*{criteria}{Критерий}
\newtheorem*{algorithm}{Алгоритм}
\newtheorem*{formula}{Формула}
\newtheorem*{remark}{Замечание}
\newtheorem*{proposition}{Предложение}

% For bigger integral signs
\usepackage{bigints}

% For semantic advancements
\usepackage{enotez}

% For pictures
\usepackage{graphicx}

%For framing
\usepackage{framed}

% Automatic paragraph indentations
\usepackage{indentfirst}
\setlength{\parindent}{1em}

% Formula spacing configuration
\delimitershortfall-1sp
\usepackage{mleftright}
\mleftright

% For resizing
\usepackage{relsize}

% More convenient text-style common notations
\DeclareMathOperator*\lowlim{\underline{\lim}}
\DeclareMathOperator*\uplim{\overline{\lim}}

\newcommand{\tlowlim}[1]{\lowlim\limits_{#1}}
\newcommand{\tuplim}[1]{\uplim\limits_{#1}}

\newcommand{\tpow}[2]{{#1}^{\mathlarger{#2}}}

\newcommand{\tint}{\displaystyle{\int}}
\newcommand{\tintlim}[2]{\displaystyle{\int\limits_{#1}^{#2}}}
\newcommand{\tbigint}{\displaystyle{\mathlarger{\int}}}
\newcommand{\tbigintlim}[2]{\displaystyle{\mathlarger{\int}\limits_{#1}^{#2}}}

\newcommand{\tsumlim}[2]{\displaystyle{\mathlarger{\sum}\displaylimits_{#1}^{#2}}}
\newcommand{\tprodlim}[2]{\displaystyle{\mathlarger{\prod}\displaylimits_{#1}^{#2}}}

\newcommand{\tmin}[1]{\min\limits_{#1}}
\newcommand{\tmax}[1]{\max\limits_{#1}}
\newcommand{\tlim}[1]{\lim\limits_{#1}}

\newcommand{\norm}[1]{\left\lVert#1\right\rVert}

% For floaty text
\usepackage{wrapfig}
\usepackage{floatflt}

% For cyrillic characters support
\usepackage[english, russian]{babel}

% For proper font
\usepackage[no-math]{fontspec}

% TNR within text
\setmainfont{Times New Roman}

% TNR within formulas
\usepackage{newtxmath}
\DeclareSymbolFont{operators}{OT1}{ntxtlf}{m}{n}
\SetSymbolFont{operators}{bold}{OT1}{ntxtlf}{b}{n}

% For degree sign
\usepackage{siunitx}

% Automatic brackets placement
\newcommand{\br}[1]{\left(#1\right)}
\newcommand{\vbr}[1]{\left|#1\right|}
\newcommand{\cbr}[1]{\left\{#1\right\}}
\newcommand{\rbr}[1]{\left[#1\right]}
\renewcommand{\le}{\leqslant}
\renewcommand{\ge}{\geqslant} 

% Automatic replacement for two greek letters
\renewcommand{\epsilon}{\varepsilon}
\renewcommand{\phi}{\varphi}

% Redefinition of some operators (to the appropriation of Russian notation)
\DeclareMathOperator{\arcsh}{arcsh}
\DeclareMathOperator{\arcch}{arcch}
\DeclareMathOperator{\arcth}{arcth}
\DeclareMathOperator{\arccth}{arccth}
\DeclareMathOperator{\rank}{rank}
\DeclareMathOperator{\inv}{inv}
\DeclareMathOperator{\sgn}{sgn}
\renewcommand{\Re}{\operatorname{Re}}
\renewcommand{\Im}{\operatorname{Im}}

\renewcommand{\gcd}{\text{НОД}}
\newcommand{\lcm}{\text{НОК}}

% Restrict inlne formula breaking
\binoppenalty=10000 
\relpenalty=10000

% Greek letters setup
% Old greek letters setup
\DeclareSymbolFont{old_letters}{OML}{ztmcm}{m}{it}
\SetSymbolFont{old_letters}{bold}{OML}{ztmcm}{b}{it}

\newcommand{\Alpha}{\text{A}}
\newcommand{\Beta}{\text{B}}
\newcommand{\Epsilon}{\text{E}}
\newcommand{\Zeta}{\text{Z}}
\newcommand{\Eta}{\text{H}}
\newcommand{\Iota}{\text{I}}
\newcommand{\Kappa}{\text{K}}
\newcommand{\Mu}{\text{M}}
\newcommand{\Nu}{\text{N}}
\newcommand{\Omicron}{\text{O}}
\newcommand{\Rho}{\text{P}}
\newcommand{\Tau}{\text{T}}
\newcommand{\Chi}{\text{X}}

\DeclareMathSymbol{\alpha}{\mathord}{old_letters}{11}
\DeclareMathSymbol{\beta}{\mathord}{old_letters}{12}
\DeclareMathSymbol{\gamma}{\mathord}{old_letters}{13}
\DeclareMathSymbol{\delta}{\mathord}{old_letters}{14}
\DeclareMathSymbol{\varepsilon}{\mathord}{old_letters}{15}
\DeclareMathSymbol{\zeta}{\mathord}{old_letters}{16}
\DeclareMathSymbol{\eta}{\mathord}{old_letters}{17}
\DeclareMathSymbol{\theta}{\mathord}{old_letters}{18}
\DeclareMathSymbol{\iota}{\mathord}{old_letters}{19}
\DeclareMathSymbol{\kappa}{\mathord}{old_letters}{20}
\DeclareMathSymbol{\lambda}{\mathord}{old_letters}{21}
\DeclareMathSymbol{\mu}{\mathord}{old_letters}{22}
\DeclareMathSymbol{\nu}{\mathord}{old_letters}{23}
\DeclareMathSymbol{\xi}{\mathord}{old_letters}{24}
\newcommand{\omicron}{\text{\textit{\larger[1]{o}}}}
\DeclareMathSymbol{\pi}{\mathord}{old_letters}{25}
\DeclareMathSymbol{\rho}{\mathord}{old_letters}{26}
\DeclareMathSymbol{\sigma}{\mathord}{old_letters}{27}
\DeclareMathSymbol{\tau}{\mathord}{old_letters}{28}
\DeclareMathSymbol{\upsilon}{\mathord}{old_letters}{29}
\DeclareMathSymbol{\varphi}{\mathord}{old_letters}{39}
\DeclareMathSymbol{\chi}{\mathord}{old_letters}{31}
\DeclareMathSymbol{\psi}{\mathord}{old_letters}{32}
\DeclareMathSymbol{\omega}{\mathord}{old_letters}{33}






\begin{document}

В окрестности точки $t = 0$ найдём фундаментальную матрицу решений $X\br{t}$ такую, что $X\br{\Omicron_{2}} = E_{2}$.
\begin{align*}
    & X\br{t} = \begin{pmatrix} x_{1}\br{t} & x_{2}\br{t} \\ y_{1}\br{t} & y_{2}\br{t} \end{pmatrix}
    &&
    \begin{pmatrix} x_{1}\br{t} & x_{2}\br{t} \\ y_{1}\br{t} & y_{2}\br{t} \end{pmatrix} = \begin{pmatrix} 1 & 0 \\ 0 & 1 \end{pmatrix}
    \\
    & x_{1}\br{t} = 1
    &&
    x_{2}\br{t} = \tsys{
        & 0 \text{, если } t \in \rbr{-1; 0}
        \\& 
        at \text{, если } t \in \left(0; 1\right] }
    \\
    & y_{1}\br{t} = \tsys{
        & bt \text{, если } t \in \rbr{-1; 0}
        \\& 
        0 \text{, если } t \in \left(0; 1\right] }
    &&
    y_{2}\br{t} = 1
\end{align*}

Матрица монодромии $C$ такова, что для всех $X\br{t + 2} = X\br{t} C$. В частности, $X\br{1} = \linebreak = X\br{-1} C$ и $C = \br{X\br{-1}}^{-1} X\br{1}$.
\begin{align*}
    X\br{-1} = \begin{pmatrix} 1 & 0 \\ -b & 1 \end{pmatrix};
    &&
    \br{X\br{-1}}^{-1} = \begin{pmatrix} 1 & 0 \\ b & 1 \end{pmatrix};
    &&
    X\br{1} = \begin{pmatrix} 1 & a \\ 0 & 1 \end{pmatrix};
    &&
    C = \begin{pmatrix} 1 & a \\ b & ab + 1 \end{pmatrix}.
\end{align*}

Вековое уравнение имеет вид 
$$\det\br{C - \rho E} = \br{\rho - 1}\br{\rho - ab - 1} - ab = \rho^2 - \br{ab + 2}\rho + 1 = 0.$$

\begin{framed}
    \begin{criteria}[устойчивости по мультипликаторам] \end{criteria}
    Пусть дана линейная однородная система с периодическими коэффициентами
    \begin{itemize}
        \item Для асимптотической устойчивости нулевого решения необходимо и достаточно, чтобы все мультипликаторы лежали внутри единичного круга $\vbr{\rho} < 1$.
        \item Для устойчивости необходимо и достаточно, чтобы
        \begin{enumerate}
            \item все мультипликаторы лежали в замкнутом единичном круге $\vbr{\rho} \le 1$;
            \item каждый мультипликатор $\rho_{k}$, лежащий на единичной окружности (т.е. $\vbr{\rho_{k}} = 1$), имел кратность как корень характеристического (векового) уравнения $\det\br{C - \rho E} = 0$, равную дефекту $\nullity\br{C - \rho_{k} E}$.
        \end{enumerate}
    \end{itemize}
\end{framed}

Уравнение и факт устойчивости эффективно зависит только от $ab$, так что для удобства переобозначим $ab = \gamma$. Исследовать будем уравнение $f\br{\rho} = \rho^2 - \br{\gamma + 2}\rho + 1 = 0$.

\begin{enumerate}
    \item Случай $D = \gamma \br{\gamma + 4} > 0$.
    $\gamma \in \br{-\infty; -4} \cup \br{0; +\infty}$. Оба корня вещественны. \linebreak Условия $\rho_{1} \in \rbr{-1; 1}$ и $\rho_{2} \in \rbr{-1; 1}$ равносильны выполнению: $f\br{-1} \ge 0$, $f\br{1} \ge 0$ и $\gamma_{0} = \dfrac {\gamma + 2} {2} \in \rbr{-1; 1}$. При этом одновременное выполнение $f\br{1} = -\gamma > 0$ и $f\br{-1} = \linebreak = 4 + \gamma > 0$ означает, что $\gamma \in \rbr{-4; 0}$. Последнее несовместимо с предыдущим, так что при всех $\gamma \in \br{-\infty; -4} \cup \br{0; +\infty}$ нулевое решение неустойчиво.
    
    \item Случай $D = \gamma \br{\gamma + 4} < 0$. $\gamma \in \br{-4; 0}$. \\
    $\rho_{1} = \dfrac {\gamma + 2 + \sqrt{-\gamma \br{\gamma + 4}} i} {2}$.
    $\rho_{2} = \dfrac {\gamma + 2 - \sqrt{-\gamma \br{\gamma + 4}} i} {2}$.
    $\vbr{\rho_{1}} = \vbr{\rho_{2}} = \dfrac {\br{\gamma + 2}^2 - \gamma \br{\gamma + 4}} {4} = 1$. Поскольку в этом случае $\gamma = ab \ne 0$, обе матрицы $C - \rho_{1} E$ и $C - \rho_{2} E$ ненулевые, а потому их дефекты равняются $1$. Таким образом, в этом случае нулевое решение устойчиво, но не асимптотически.
    
    \item Случай $\gamma = ab = -4$. $b = -\dfrac {4} {a}$. $\rho_{1} = \rho_{2} = -1$. 
    $C - \rho_{1} E 
    = \begin{pmatrix} 2 & a \\ b & ab + 2 \end{pmatrix} 
    = \begin{pmatrix} 2 & a \\ -\dfrac {4} {a} & -2 \end{pmatrix} $. 
    $\rank\br{C - \rho_{1} E} = 1$, $\nullity\br{C - \rho_{1} E} = 2 - 1 = 1$. Итак, нулевое решение неустойчиво.
    
    \item Случай $\gamma = ab = 0$. $\rho_{1} = \rho_{2} = 1$. $C -\rho_{1} E = \begin{pmatrix} 0 & a \\ b & ab \end{pmatrix} = \begin{pmatrix} 0 & a \\ b & 0 \end{pmatrix}$.
    \begin{itemize}
        \item Если $a^2 + b^2 > 0$, то $\rank\br{C -\rho_{1} E} = 1$, $\nullity\br{C -\rho_{1} E} = 2 - 1 = 1$, и нулевое решение неустойчиво.
        \item Если $a = 0$ и $b = 0$, то $\rank\br{C -\rho_{1} E} = \rank \Omicron_{2} = 0$, $\nullity\br{C -\rho_{1} E} = 2$, и нулевое решение устойчиво, но не асимптотически.
    \end{itemize}
\end{enumerate}

Окончательно получаем, что при $ab \in \br{-4; 0}$, а также, если $a = 0$, $b = 0$, нулевое решение устойчиво, но не асимптотически; во всех остальных случаях нулевое решение неустойчиво.

\end{document}
