\documentclass[a5paper,10pt]{article}

% https://github.com/InDevRus/filippov-solutions
% Page setup
\pagenumbering{gobble}

% Margin setup
\usepackage{geometry}
\geometry{left=1cm}
\geometry{right=1cm}
\geometry{top=1cm}
\geometry{bottom=1.5cm}

% For inserting tables
\usepackage{array}

% Formula aligning
\usepackage{amsmath}

% For formula diacritics
\usepackage{amsfonts}

% Theorem definitions
\usepackage{amsthm}
\theoremstyle{definition}
\newtheorem*{theorem}{Теорема}
\newtheorem*{corollary}{Следствие}
\newtheorem*{criteria}{Критерий}
\newtheorem*{algorithm}{Алгоритм}
\newtheorem*{formula}{Формула}
\newtheorem*{remark}{Замечание}
\newtheorem*{proposition}{Предложение}

% For bigger integral signs
\usepackage{bigints}

% For semantic advancements
\usepackage{enotez}

% For pictures
\usepackage{graphicx}

%For framing
\usepackage{framed}

% Automatic paragraph indentations
\usepackage{indentfirst}
\setlength{\parindent}{1em}

% Formula spacing configuration
\delimitershortfall-1sp
\usepackage{mleftright}
\mleftright

% For resizing
\usepackage{relsize}
% More convenient text-style common notations
\newcommand{\tpow}[2]{{#1}^{\mathlarger{#2}}}
\newcommand{\tint}{\displaystyle{\int}}
\newcommand{\tintlim}[2]{\displaystyle{\int\limits_{#1}^{#2}}}
\newcommand{\tbigint}{\displaystyle{\mathlarger{\int}}}
\newcommand{\tbigintlim}[2]{\displaystyle{\mathlarger{\int}\limits_{#1}^{#2}}}
\newcommand{\tsumlim}[2]{\displaystyle{\mathlarger{\sum}\displaylimits_{#1}^{#2}}}
\newcommand{\tprodlim}[2]{\displaystyle{\mathlarger{\prod}\displaylimits_{#1}^{#2}}}
\newcommand{\tmin}[1]{\min\limits_{#1}}
\newcommand{\tmax}[1]{\max\limits_{#1}}
\newcommand{\tlim}[1]{\lim\limits_{#1}}

% For floaty text
\usepackage{wrapfig}
\usepackage{floatflt}

% For cyrillic characters support
\usepackage[english, russian]{babel}

% For proper font
\usepackage[no-math]{fontspec}

% TNR within text
\setmainfont{Times New Roman}

% TNR within formulas
\usepackage{newtxmath}
\DeclareSymbolFont{operators}{OT1}{ntxtlf}{m}{n}
\SetSymbolFont{operators}{bold}{OT1}{ntxtlf}{b}{n}

% For degree sign
\usepackage{siunitx}

% Automatic brackets placement
\newcommand{\br}[1]{\left(#1\right)}
\newcommand{\vbr}[1]{\left|#1\right|}
\newcommand{\cbr}[1]{\left\{#1\right\}}
\newcommand{\rbr}[1]{\left[#1\right]}
\renewcommand{\le}{\leqslant}
\renewcommand{\ge}{\geqslant} 

% Automatic replacement for two greek letters
\renewcommand{\epsilon}{\varepsilon}
\renewcommand{\phi}{\varphi}

% Redefinition of some operators (to the appropriation of Russian notation)
\DeclareMathOperator{\arcsh}{arcsh}
\DeclareMathOperator{\arcch}{arcch}
\DeclareMathOperator{\arcth}{arcth}
\DeclareMathOperator{\arccth}{arccth}
\DeclareMathOperator{\rank}{rank}
\DeclareMathOperator{\inv}{inv}
\renewcommand{\Re}{\operatorname{Re}}
\renewcommand{\Im}{\operatorname{Im}}

\renewcommand{\gcd}{\text{НОД}}
\newcommand{\lcm}{\text{НОК}}

% Restrict inlne formula breaking
\binoppenalty=10000 
\relpenalty=10000

% Greek letters setup
\usepackage{textalpha}

\newcommand{\zed}{\textit{z}}

\newcommand{\Alpha}{\text{A}}
\newcommand{\Beta}{\text{B}}
\newcommand{\Epsilon}{\text{E}}
\newcommand{\Zeta}{\text{Z}}
\newcommand{\Eta}{\text{H}}
\newcommand{\Iota}{\text{I}}
\newcommand{\Kappa}{\text{K}}
\newcommand{\Mu}{\text{M}}
\newcommand{\Nu}{\text{N}}
\newcommand{\Omicron}{\text{O}}
\newcommand{\Rho}{\text{P}}
\newcommand{\Tau}{\text{T}}
\newcommand{\Chi}{\text{X}}

\renewcommand{\alpha}{\text{\textalpha}}
\renewcommand{\beta}{\text{\textbeta}}
\renewcommand{\gamma}{\text{\textgamma}}
\renewcommand{\delta}{\text{\textdelta}}
\renewcommand{\epsilon}{\text{\textepsilon}}
\renewcommand{\zeta}{\text{\textzeta}}
\renewcommand{\eta}{\text{\texteta}}
\renewcommand{\theta}{\text{\texttheta}}
\renewcommand{\iota}{\text{\textiota}}
\renewcommand{\kappa}{\text{\textkappa}}
\renewcommand{\lambda}{\text{\textlambda}}
\renewcommand{\mu}{\text{\textmu}}
\renewcommand{\nu}{\text{\textnu}}
\renewcommand{\xi}{\text{\textxi}}
\newcommand{\omicron}{\text{\textomicron}}
\renewcommand{\pi}{\text{\textpi}}
\renewcommand{\rho}{\text{\textrho}}
\renewcommand{\sigma}{\text{\textsigma}}
\renewcommand{\tau}{\text{\texttau}}
\renewcommand{\upsilon}{\text{\textupsilon}}
\renewcommand{\phi}{\text{\textphi}}
\renewcommand{\chi}{\text{\textchi}}
\renewcommand{\psi}{\text{\textpsi}}
\renewcommand{\omega}{\text{\textomega}}






\usepackage{pgfplots}
\pgfplotsset{compat=newest}
\pgfplotsset{
    every axis/.append style = {
        xlabel={$x$},
        ylabel={$y$}, 
        axis lines = middle,
        unit vector ratio = 1 1,
        font=\small
    },
    every axis plot/.append style = {smooth, samples = 100}
}

\begin{document}

Траектории этой системы удовлетворяют уравнению $\dfrac {dy} {dx} = \dfrac {\dot{y}} {\dot{x}} = - \dfrac {\sin{x}} {y \cos{x}}$. $2yy'_{x}\br{x} = \linebreak = -2 \dfrac {\sin{x}} {\cos{x}}$. $y^2\br{x} = 2 \ln \vbr{\cos x} + C$. Итак, траектории задаются $y\br{x} = \pm \sqrt{C + \ln \br{\cos^2 x}}$.

\begin{floatingfigure}{0.35\textwidth}
    \begin{tikzpicture}
        \pgfplotsset{
        VectorGraph/.style = {
            -stealth,
            quiver = {
                u = -y * cos(deg(x)) / (x^2 + y^2) ^ 0.5, 
                v = sin(deg(x)) / (x^2 + y^2) ^ 0.5, 
                scale arrows = 0.1}
            }
        }
        \begin{axis}[
            width = \textwidth,
            height = 7cm,
            ymin = -2.2,
            ymax = 2.4,
            xmin = -1.6,
            xmax = 1.7, 
            xtick = {-2, -1, ..., 2},
            ytick = {-2, -1, ..., 2},
            minor tick num = 3,
            declare function = {
                f(\C, \x) = sqrt(\C + 2 * ln(abs(cos(deg(x)))));
            }],
            
            \addplot[domain = -2 : 2, samples = 1000]
            (rad(acos(exp(0.5 * (x^2 - 4)))), x);
            \addplot[domain = -2 : 2, samples = 1000]
            (-rad(acos(exp(0.5 * (x^2 - 4)))), x);
            
            \addplot[domain = -sqrt(2) : sqrt(2), samples = 1000]
            (rad(acos(exp(0.5 * (x^2 - 2)))), x);
            \addplot[domain = -sqrt(2) : sqrt(2), samples = 1000] 
            (-rad(acos(exp(0.5 * (x^2 - 2)))), x);
            
            \addplot[domain = -1 : 1, samples = 1000]
            (rad(acos(exp(0.5 * (x^2 - 1)))), x);
            \addplot[domain = -1 : 1, samples = 1000]
            (-rad(acos(exp(0.5 * (x^2 - 1)))), x);
            
            \addplot[domain = -sqrt(0.5) : sqrt(0.5), samples = 1000]
            (rad(acos(exp(0.5 * (x^2 - 0.5)))), x);
            \addplot[domain = -sqrt(0.5) : sqrt(0.5), samples = 1000]
            (-rad(acos(exp(0.5 * (x^2 - 0.5)))), x);
            
            \addplot[domain = -0.5 : 0.5, samples = 1000, dashed]
            (rad(acos(exp(0.5 * (x^2 - 0.25)))), x);
            \addplot[domain = -0.5 : 0.5, samples = 1000, dashed] 
            (-rad(acos(exp(0.5 * (x^2 - 0.25)))), x);
            
            \addplot[domain = -1/3 : 1/3, samples = 1000, dashed]
            (rad(acos(exp(0.5 * (x^2 - 1/9)))), x);
            \addplot[domain = -0.5 : 0.5, samples = 1000, dashed] 
            (-rad(acos(exp(0.5 * (x^2 - 1/9)))), x);
            
            \addplot[domain = -0.25 : 0.25, samples = 1000, dashed]
            (rad(acos(exp(0.5 * (x^2 - 0.0625)))), x);
            \addplot[domain = -0.5 : 0.5, samples = 1000, dashed] 
            (-rad(acos(exp(0.5 * (x^2 - 0.0625)))), x);

            \addplot[VectorGraph, samples = 7, domain = -1.43 : 1.435]
            {f(4, x)};
            \addplot[VectorGraph, samples = 5, domain = -1.4 : 1.4]
            {-f(4, x)};
            
            \addplot[VectorGraph, samples = 6, domain = -1.19 : 1.19]
            {f(2, x)};
            \addplot[VectorGraph, samples = 4, domain = -1.18 : 1.18]
            {-f(2, x)};
            
            \addplot[VectorGraph, samples = 5, domain = -0.91 : 0.91]
            {f(1, x)};
            \addplot[VectorGraph, samples = 5, domain = -0.8 : 0.8]
            {-f(1, x)};
            
            \addplot[VectorGraph, samples = 5, domain = -0.672 : 0.677]
            {f(0.5, x)};
            \addplot[VectorGraph, samples = 5, domain = -0.7 : 0.7]
            {-f(0.5, x)};
        \end{axis}
    \end{tikzpicture}
\end{floatingfigure}


Они определены только при $C > 0$ и представляют собой замкнутые кривые при всех таких $C$. Эти траектории вместе с направляющими векторами показаны на графике. Оценив рисунок, можно утверждать, что замкнутую траекторию можно подобрать настолько, насколько необходимо, близкой к началу координат фазовой плоскости, то есть к нулевому решению $\Phi\br{t} = \begin{pmatrix}0 \\ 0\end{pmatrix}$.

Вывод: решение устойчиво, хотя и не асимптотически.

\vspace{1.25 cm}

Докажем это по определению. Для всякой начальной точки $A_{0} = \br{x\br{t_{0}}; y\br{t_{0}}}$ при условии $x\br{t_{0}} \in \br{-\dfrac {\pi} {2}; -\dfrac {\pi} {2}}$ можно однозначно отыскать $C =  y^2\br{t_{0}} - \ln\br{\cos^2 x\br{t_{0}}}$, и это означает, что через $A_{0}$ проходит единственная траектория. При $t \ge t_{0}$ точка будет оставаться на этой же траектории с неизменным $C$. Для удобства положим $t_{0} = 0$.

Траектория $y^2\br{x} = 2 \ln \vbr{\cos x} + C$ определена для $\vbr{x} \le \arccos\br{\tpow{e} {-\frac {C} {2}}} < \dfrac {\pi} {2}$. Обозначим за $\rho = \rho\br{t}$ расстояние от точки на траектории $X\br{t} = \begin{pmatrix} x\br{t} \\ y\br{t} \end{pmatrix}$ до начала координат.
$$\rho\br{t} = \rho\br{x\br{t}} = \sqrt{x^2\br{t} + y^2\br{t}} = \sqrt{x^2\br{t} + \ln\br{\cos^2 x\br{t}} + C} = \sqrt{f\br{x\br{t}} + C},$$
где за $f\br{x}$ обозначено выражение $f\br{x} = x^2 + \ln\br{\cos^2 x}$. Эта функция чётная и убы-вает для $0 \le x < \dfrac {\pi} {2}$, потому что для таких $x$ имеем $f'\br{x} = 2\br{x - \tg{x}} < 0$. Это значит, что $x^2 + \ln\br{\cos^2 x} = f\br{x} \le f\br{0} = 0$.

Следовательно, в $x = 0$ расстояние наибольшее и равно $\rho_{\max}\br{C} = \sqrt{C}$; наименьшее достигается в концах отрезка $x = \pm \arccos\br{\tpow {e} {-\frac {C} {2}}}$ и равно $\rho_{\min}\br{C} = \arccos\br{\tpow {e} {-\frac {C} {2}}}$.
\begin{align*}
\vbr{X\br{t} - \Phi\br{t}} = \sqrt{x^2\br{t} + y^2\br{t}} &&
\vbr{X\br{t_{0}} - \Phi\br{t_{0}}} = \sqrt{x^2\br{0} + y^2\br{0}} &&
\end{align*}

Пусть теперь $\epsilon > 0$ -- произвольное и $\delta\br{\epsilon} = \arctg\br{\epsilon}$. Потребуем также, чтобы для начальной точки $A_{0}$ выполнялось $\sqrt{x^2\br{0} + y^2\br{0}} < \delta$. Из этого следует 
$$\delta > \sqrt{x^2\br{0} + y^2\br{0}} \ge \rho_{\min} = \arccos\br{\tpow {e} {-\frac {C} {2}}}.$$

Функция $g\br{s} = \arccos\br{\tpow {e} {-\frac {s} {2}}}$ возрастает, что позволяет решить относительно $C$ неравенство $\arccos\br{\tpow {e} {-\frac {C} {2}}} < \delta$ и получить $C < \ln\br{\dfrac {1}{\cos^{2} \delta}}$.

Наконец, учитывая неравенство $\ln a \le a - 1$ для всех $a$, основное тригонометрическое тождество и решение неравенства сверху, получаем при $t \ge t_{0} = 0$
$$\sqrt{x^2\br{t} + y^2\br{t}}
\le \rho_{\max}\br{C}
= \sqrt{C}
< \sqrt{\ln\br{\dfrac {1} {\cos^{2} \delta}}} 
\le \sqrt{ \dfrac {1}{\cos^{2} \delta} - 1 }
= \sqrt{\tg^2 \delta}
= \epsilon.$$

Устойчивость доказана.

\end{document}
