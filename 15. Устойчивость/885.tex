% Page setup
\documentclass[a5paper,10pt]{article}
\pagenumbering{gobble}

% Margin setup
\usepackage{geometry}
\geometry{left=1cm}
\geometry{right=1cm}
\geometry{top=1cm}
\geometry{bottom=1.5cm}

% %For inserting tables
% \usepackage{array}

% Formula aligning
\usepackage{amsmath}

% For formula diacritics
\usepackage{amsfonts}

% For additional symbols
\usepackage{amssymb}

% For bigger integral signs
\usepackage{bigints}

% For appropriate vector signings
%\usepackage{esvect}

% For semantic advancements
%\usepackage{enotez}

% For degree sign
% \usepackage{gensymb}

%For pictures
% \usepackage{graphicx}

%For framing
% \usepackage{framed}

% Automatic paragraph indentations
\usepackage{indentfirst}
\setlength{\parindent}{1em}

% Formula spacing configuration
\delimitershortfall-1sp
\usepackage{mleftright}
\mleftright

% For resizing
\usepackage{relsize}
% More convenient text-style power
\newcommand{\tpow}[2]{{#1}^{\mathlarger{#2}}}
\newcommand{\tint}{\displaystyle{\mathlarger{\int}}}
\newcommand{\tintlim}[2]{\displaystyle{\mathlarger{\int}\displaylimits^{#1}_{#2}}}

%For svg support
%\usepackage[inkscapelatex=false]{svg}

%For underlining
%\usepackage{ulem}

% For floaty text
% \usepackage{wrapfig}
\usepackage{floatflt}
\usepackage{placeins}

% Extra math fonts
\usepackage{yhmath}

% For cyrillic characters support
\usepackage[english, russian]{babel}

% For proper font
\usepackage{fontspec}

% TNR within formulas
\usepackage{mathptmx}

% TNR within text
\setmainfont{Times New Roman}

% Automatic brackets placement
\newcommand{\br}[1]{\left(#1\right)}
\newcommand{\vbr}[1]{\left|#1\right|}
\newcommand{\cbr}[1]{\left\{#1\right\}}
\newcommand{\rbr}[1]{\left[#1\right]}
\renewcommand{\le}{\leqslant}
\renewcommand{\ge}{\geqslant}

% Redefinition of some operators (to the appropriation of Russian notation)
\renewcommand{\Re}{\operatorname{Re}}
\renewcommand{\Im}{\operatorname{Im}}

% Enfore fractures to be full-sized everywhere
% \renewcommand{\frac}{\dfrac}

% Special operator definitions
% \DeclareMathOperator{\dis}{\rho}
% \DeclareMathOperator{\dif}{\d}

% Theorem definitions
% \usepackage{amsthm}
% \theoremstyle{definition}
% \newtheorem*{theorem}{Теорема}
% \newtheorem*{corollary}{Следствие}
% \newtheorem*{criteria}{Критерий}
% \newtheorem*{algorithm}{Алгоритм}
% \newtheorem*{formula}{Формула}
% \newtheorem*{remark}{Замечание}
% \newtheorem*{proposition}{Предложение}

% Restrict inlne formula breaking
\binoppenalty=10000 
\relpenalty=10000 

%For plotting
\usepackage{pgfplots}
\pgfplotsset{compat=newest}
\pgfplotsset{
    every axis/.append style = {
        xlabel={$x$},
        ylabel={$y$}, 
        axis lines = middle,
        unit vector ratio = 1 1,
        font=\small
    },
    every axis plot/.append style = {smooth, samples = 100}
}

\begin{document}

Исследовать решения этой системы на устойчивость путём нахождения общего решения будет затруднительно, так как общее решение, хотя и может быть выражено в квадратурах, будет включать в себя эллиптический интеграл
$$\bigintssss{\frac {dt} {\sqrt{C^4 - t^4}}}.$$

Сначала найдём траектории, задаваемые решением этой системы. $\dfrac {dy} {dx} = \dfrac {\dot{x}} {\dot{y}} = \dfrac {-2x^3} {y}$. $2yy'_{x} + 4x^3 = 0$. $y^2 + x^4 = C^4$ (в правой части взят $C^4$, потому что левая часть не даёт отрицательных значений). Для удобства также будем считать $C \ge 0$. Функция \linebreak $y\br{x} = \sqrt{C^4 - x^4}$ задаёт выпуклый вверх график, а в точках $x = \pm C$ значение равно нулю, так что уравнение $y^2 + x^4 = C^4$ задаёт замкнутую выпуклую кривую.

Направляющие векторы, посчитанные для некоторых точек, указаны на рисунке. Из рисунка видно, что нулевое решение $\Phi_{0}\br{t} = \begin{pmatrix} 0 \\ 0 \end{pmatrix}$ соответствует особой точке (началу координат) и эта точка относится к типу "центр". Решения, по всей видимости, являются периодическими функциями.

\begin{floatingfigure}{0.375\textwidth}
    \begin{tikzpicture}
        \pgfplotsset{
        VectorGraph/.style = {
                -stealth,
                quiver = {
                    u = -y / (x^2 + y^2) ^ 0.5, 
                    v = 2 * x^3 / (x^2 + y^2) ^ 0.5, 
                    scale arrows = 0.1}}
        }
        \begin{axis}[
            width = \textwidth,
            height = 10cm,
            ymin = -4.2,
            ymax = 4.7,
            xmin = -2.2,
            xmax = 2.45, 
            xtick = {-3, ..., 3},
            ytick = {-5, ..., 5},
            minor tick num = 3,
            declare function = {
                f(\C, \x) = sqrt(\C^4 - x^4);
            }]
            
            \addplot[domain=-0.5:0.5] {f(0.5, x)};
            \addplot[domain=-0.5:0.5] {-f(0.5, x)};
        
            \addplot[domain=-sqrt(1/2):sqrt(1/2)] {f(sqrt(1/2), x)};
            \addplot[domain=-sqrt(1/2):sqrt(1/2)] {-f(sqrt(1/2), x)};
            
            \addplot[domain=-1:1] {f(1, x)};
            \addplot[domain=-1:1] {-f(1, x)};
            \addplot[domain=-1:1] {f(1, x)};
            \addplot[domain=-1:1] {-f(1, x)};
            \addplot[domain=1:-1] {f(1, x)};
            \addplot[domain=1:-1] {-f(1, x)};
            \addplot[VectorGraph, samples = 4, domain = -1 : 1] {f(1, x)};
            \addplot[VectorGraph, samples = 2, domain = -0.5 : 0.5] {-f(1, x)};
            
            \addplot[domain=-1.5:1.5] {f(1.5, x)};
            \addplot[domain=-1.5:1.5] {-f(1.5, x)};
            \addplot[domain=1.5:-1.5] {f(1.5, x)};
            \addplot[domain=1.5:-1.5] {-f(1.5, x)};
            
            \addplot[domain=-2:2, samples = 1000] {f(2, x)};
            \addplot[domain=-2:2, samples = 1000] {-f(2, x)};
            \addplot[domain=2:-2, samples = 1000] {f(2, x)};
            \addplot[domain=2:-2, samples = 1000] {-f(2, x)};
            \addplot[VectorGraph, samples = 10, domain = -2 : 2] {f(2, x)};
            \addplot[VectorGraph, samples = 10, domain = -1.8 : 1.8] {-f(2, x)};
        \end{axis}
    \end{tikzpicture}
\end{floatingfigure}

Найдём наибольшее расстояние $\rho_{\max} = \rho_{\max}\br{C}$ от точки на этой кривой до начала координат. Для всякого $x \in \rbr{-C; C}$ расстояние равняется $\rho = \rho\br{x} \linebreak = \sqrt{x^2 + y^2} = \sqrt{C^4 + x^2 - x^4}$.

Наибольшее расстояние достигается в точках \linebreak $x = \pm \dfrac {1} {\sqrt{2}}$, если она входит в область определения; в противном случае на $x \in \rbr{0; C}$ расстояние возрастает и наибольшее достигается при $x = C$.

Итак, точное наибольшее расстояние равно
$$\rho_{\max}\br{C} =  \begin{cases} \sqrt{C^4 + \dfrac {1} {4}}\text{, если } C > \dfrac {1} {2} \\ C \text{, если } C \le \dfrac {1}{2} \end{cases}$$

\vspace{2.5cm}

Докажем теперь устойчивость нулевого решения. Имеем для всякого решения \linebreak $X\br{t} = \begin{pmatrix} x\br{t} \\ y\br{t} \end{pmatrix}$ и начального параметра $t_{0} = 0$
\begin{align*}
\vbr{X\br{t} - \Phi\br{t}} = \sqrt{x^2\br{t} + y^2\br{t}} && \vbr{X\br{t_0} - \Phi\br{t_0}} = \sqrt{x^2\br{0} + y^2\br{0}}
\end{align*}

Возьмём произвольное $\epsilon > 0$ и для удобства ограничим $\delta \le \dfrac {1} {4}$. Положим таким образом $\delta = \min\br{\epsilon^2; \dfrac {1} {4}}$. Заметим, что для всякой начальной точки $A_{0} = \br{x\br{0}; y\br{0}}$, отличной от начала координат, можно однозначно вычислить $C > 0$. Это значит, что через всякую начальную точку $A_{0}$ проходит некоторая траектория $y^2 + x^4 = C^4$, при этом единственная.

Итак, пусть $\sqrt{x^2\br{0} + y^2\br{0}} < \delta$. При таких условиях имеем $\vbr{x\br{0}} < \delta \le \dfrac {1} {4}$ и \linebreak $\vbr{y\br{0}} < \delta \le \dfrac {1} {4}$. Следовательно, $C = \sqrt[4]{x^4\br{0} + y^2\br{0}} < \sqrt[4]{x^2\br{0} + y^2\br{0}} < \sqrt{\delta} \le \dfrac {1} {2}$.

Для всякого $t \ge t_{0} = 0$ расстояние $\sqrt{x^2\br{t} + y^2\br{t}}$ от точки $X\br{t}$ на траектории до начала координат таково, что
$$\sqrt{x^2\br{t} + y^2\br{t}} \le \rho_{\max} = C < \sqrt{\delta} \le \sqrt{\epsilon^2} = \epsilon.$$

Таким образом, решение устойчиво, но не асимптотически, поскольку найденные траектории являются замкнутыми и к центру координат фазовой плоскости стремиться не могут.


\end{document}
