\documentclass[a5paper,10pt]{article}

% https://github.com/InDevRus/filippov-solutions
% Page setup
\pagenumbering{gobble}

% Margin setup
\usepackage{geometry}
\geometry{left=1cm}
\geometry{right=1cm}
\geometry{top=1cm}
\geometry{bottom=1.5cm}

% For inserting tables
\usepackage{array}

% Formula aligning
\usepackage{amsmath}

% For formula diacritics
\usepackage{amsfonts}

% Theorem definitions
\usepackage{amsthm}
\theoremstyle{remark}
\newtheorem*{necessity}{Необходимость}
\newtheorem*{sufficiency}{Достаточность}
\theoremstyle{definition}
\newtheorem*{theorem}{Теорема}
\newtheorem*{lemma}{Лемма}
\newtheorem*{corollary}{Следствие}
\newtheorem*{criteria}{Критерий}
\newtheorem*{algorithm}{Алгоритм}
\newtheorem*{formula}{Формула}
\newtheorem*{remark}{Замечание}
\newtheorem*{proposition}{Предложение}

% For bigger integral signs
\usepackage{bigints}

% For semantic advancements
\usepackage{enotez}

% For pictures
\usepackage{graphicx}

%For framing
\usepackage{framed}

% Automatic paragraph indentations
\usepackage{indentfirst}
\setlength{\parindent}{1em}

% Formula spacing configuration
\delimitershortfall-1sp
\usepackage{mleftright}
\mleftright

% For resizing
\usepackage{relsize}

% More convenient text-style common notations
\DeclareMathOperator*\lowlim{\underline{\lim}}
\DeclareMathOperator*\uplim{\overline{\lim}}

\newcommand{\tlowlim}[1]{\lowlim\limits_{#1}}
\newcommand{\tuplim}[1]{\uplim\limits_{#1}}

\newcommand{\tpow}[2]{{#1}^{\mathlarger{#2}}}

\newcommand{\tint}{\displaystyle{\int}}
\newcommand{\tintlim}[2]{\displaystyle{\int\limits_{#1}^{#2}}}
\newcommand{\tbigint}{\displaystyle{\mathlarger{\int}}}
\newcommand{\tbigintlim}[2]{\displaystyle{\mathlarger{\int}\limits_{#1}^{#2}}}

\newcommand{\tsumlim}[2]{\displaystyle{\mathlarger{\sum}\displaylimits_{#1}^{#2}}}
\newcommand{\tprodlim}[2]{\displaystyle{\mathlarger{\prod}\displaylimits_{#1}^{#2}}}

\newcommand{\tmin}[1]{\min\limits_{#1}}
\newcommand{\tmax}[1]{\max\limits_{#1}}
\newcommand{\tlim}[1]{\lim\limits_{#1}}

\newcommand{\norm}[1]{\left\lVert#1\right\rVert}

% For floaty text
\usepackage{wrapfig}
\usepackage{floatflt}

% For cyrillic characters support
\usepackage[english, russian]{babel}

% For proper font
\usepackage[no-math]{fontspec}

% TNR within text
\setmainfont{Times New Roman}

% TNR within formulas
\usepackage{newtxmath}
\DeclareSymbolFont{operators}{OT1}{ntxtlf}{m}{n}
\SetSymbolFont{operators}{bold}{OT1}{ntxtlf}{b}{n}

% For degree sign
\usepackage{siunitx}

% Automatic brackets placement
\newcommand{\br}[1]{\left(#1\right)}
\newcommand{\vbr}[1]{\left|#1\right|}
\newcommand{\cbr}[1]{\left\{#1\right\}}
\newcommand{\rbr}[1]{\left[#1\right]}
\renewcommand{\le}{\leqslant}
\renewcommand{\ge}{\geqslant} 

% Automatic replacement for two greek letters
\renewcommand{\epsilon}{\varepsilon}
\renewcommand{\phi}{\varphi}

% Redefinition of some operators (to the appropriation of Russian notation)
\DeclareMathOperator{\arcsh}{arcsh}
\DeclareMathOperator{\arcch}{arcch}
\DeclareMathOperator{\arcth}{arcth}
\DeclareMathOperator{\arccth}{arccth}
\DeclareMathOperator{\rank}{rank}
\DeclareMathOperator{\inv}{inv}
\DeclareMathOperator{\sgn}{sgn}
\renewcommand{\Re}{\operatorname{Re}}
\renewcommand{\Im}{\operatorname{Im}}

\renewcommand{\gcd}{\text{НОД}}
\newcommand{\lcm}{\text{НОК}}

% Restrict inlne formula breaking
\binoppenalty=10000 
\relpenalty=10000

% Greek letters setup
% Old greek letters setup
\DeclareSymbolFont{old_letters}{OML}{ztmcm}{m}{it}
\SetSymbolFont{old_letters}{bold}{OML}{ztmcm}{b}{it}

\newcommand{\Alpha}{\text{A}}
\newcommand{\Beta}{\text{B}}
\newcommand{\Epsilon}{\text{E}}
\newcommand{\Zeta}{\text{Z}}
\newcommand{\Eta}{\text{H}}
\newcommand{\Iota}{\text{I}}
\newcommand{\Kappa}{\text{K}}
\newcommand{\Mu}{\text{M}}
\newcommand{\Nu}{\text{N}}
\newcommand{\Omicron}{\text{O}}
\newcommand{\Rho}{\text{P}}
\newcommand{\Tau}{\text{T}}
\newcommand{\Chi}{\text{X}}

\DeclareMathSymbol{\alpha}{\mathord}{old_letters}{11}
\DeclareMathSymbol{\beta}{\mathord}{old_letters}{12}
\DeclareMathSymbol{\gamma}{\mathord}{old_letters}{13}
\DeclareMathSymbol{\delta}{\mathord}{old_letters}{14}
\DeclareMathSymbol{\varepsilon}{\mathord}{old_letters}{15}
\DeclareMathSymbol{\zeta}{\mathord}{old_letters}{16}
\DeclareMathSymbol{\eta}{\mathord}{old_letters}{17}
\DeclareMathSymbol{\theta}{\mathord}{old_letters}{18}
\DeclareMathSymbol{\iota}{\mathord}{old_letters}{19}
\DeclareMathSymbol{\kappa}{\mathord}{old_letters}{20}
\DeclareMathSymbol{\lambda}{\mathord}{old_letters}{21}
\DeclareMathSymbol{\mu}{\mathord}{old_letters}{22}
\DeclareMathSymbol{\nu}{\mathord}{old_letters}{23}
\DeclareMathSymbol{\xi}{\mathord}{old_letters}{24}
\newcommand{\omicron}{\text{\textit{\larger[1]{o}}}}
\DeclareMathSymbol{\pi}{\mathord}{old_letters}{25}
\DeclareMathSymbol{\rho}{\mathord}{old_letters}{26}
\DeclareMathSymbol{\sigma}{\mathord}{old_letters}{27}
\DeclareMathSymbol{\tau}{\mathord}{old_letters}{28}
\DeclareMathSymbol{\upsilon}{\mathord}{old_letters}{29}
\DeclareMathSymbol{\varphi}{\mathord}{old_letters}{39}
\DeclareMathSymbol{\chi}{\mathord}{old_letters}{31}
\DeclareMathSymbol{\psi}{\mathord}{old_letters}{32}
\DeclareMathSymbol{\omega}{\mathord}{old_letters}{33}





\usepackage{pgfplots}
\pgfplotsset{compat = 1.18, lua backend = true}
\pgfplotsset{
    every axis/.append style = {
        xlabel={$x$},
        ylabel={$y$},
        axis lines = middle,
        unit vector ratio = 1 1,
        font=\small
    },
    every axis plot/.append style = {smooth, samples = 100}
}


\begin{document}

$\dfrac {dy} {dx} = \dfrac {\dot{y}} {\dot{x}} = \dfrac {2y} {x}$.
$\dfrac {x^2 y'_{x}\br{x} - 2x y\br{x}} {x^4} = 0$. 
$\dfrac {d} {dx} \br{\dfrac {y\br{x}} {x^2}} = 0$.
$y = Cx^2$. Итак, траекториями системы являются параболы, проходящие через начало координат фазовой плоскости.

\begin{floatingfigure}{0.275\textwidth}
    \begin{tikzpicture}
        \pgfplotsset{
        VectorGraph/.style = {
                -stealth,
                quiver = {
                    u = -x / (x^2 + y^2) ^ 0.5, 
                    v = -2 * y / (x^2 + y^2) ^ 0.5, 
                    scale arrows = 0.2}}
        }
        \begin{axis}[
            width = \textwidth,
            height = 7cm,
            ymin = -4.2,
            ymax = 4.7,
            xmin = -2.2,
            xmax = 2.45, 
            xtick = {-3, ..., 3},
            ytick = {-5, ..., 5},
            minor tick num = 3,
            declare function = {
                f(\C, \x) = \C * x^2;
            }]
            
            \addplot[domain=-3:3] {f(0.5, x)};
            \addplot[VectorGraph, samples = 6, domain = -2 : -0.75] {f(0.5, x)};
            \addplot[VectorGraph, samples = 6, domain = 0.75 : 2] {f(0.5, x)};
            \addplot[domain=-3:3] {f(-0.5, x)};
            \addplot[VectorGraph, samples = 6, domain = -2 : -0.75] {f(-0.5, x)};
            \addplot[VectorGraph, samples = 6, domain = 0.75 : 2] {f(-0.5, x)};
            \addplot[domain=-3:3] {f(1, x)};
            \addplot[VectorGraph, samples = 6, domain = -2 : -0.75] {f(1, x)};
            \addplot[VectorGraph, samples = 6, domain = 0.75 : 2] {f(1, x)};
            \addplot[domain=-3:3] {f(-1, x)};
            \addplot[VectorGraph, samples = 6, domain = -2 : -0.75] {f(-1, x)};
            \addplot[VectorGraph, samples = 6, domain = 0.75 : 2] {f(-1, x)};            
            \addplot[domain=-3:3] {f(2, x)};
            \addplot[VectorGraph, samples = 6, domain = -2 : -0.75] {f(2, x)};
            \addplot[VectorGraph, samples = 6, domain = 0.75 : 2] {f(2, x)};
            \addplot[domain=-3:3] {f(-2, x)};
            \addplot[VectorGraph, samples = 6, domain = -1.4 : -0.75] {f(-2, x)};
            \addplot[VectorGraph, samples = 6, domain = 0.75 : 1.4] {f(-2, x)};
            
        \end{axis}
    \end{tikzpicture}
\end{floatingfigure}

На рисунке показаны траектории вместе с направляющими векторами, посчитанными в некоторых точках вдоль траекторий. По рисунку видно, что решения стремятся к началу координат, по сути к нулевому решению $\Phi\br{t} = \begin{pmatrix} 0 \\ 0 \end{pmatrix}$, так что решение асимптотически устойчиво, а особая точка $\left\{ \begin{matrix} x = 0 \\ y = 0 \end{matrix} \right.$ относится к типу "узел".

Докажем по определению устойчивость. Общее решение имеет вид $X\br{t} = \begin{pmatrix} A e^{-t} \\ B e^{-2t} \end{pmatrix}$. Для $t \ge t_{0} = 0$
\begin{gather*}
\br{X\br{t} - \Phi\br{t}}^2 = A^2 e^{-2t} + B^2 e^{-4t} \\
\br{X\br{t_{0}} - \Phi\br{t_{0}}}^2 = A^2 + B^2
\end{gather*}

Для $\epsilon > 0$ положим $\delta\br{\epsilon} = \epsilon$ и потребуем, чтобы $A$ и $B$ были таковы, что $A^2 + B^2 < \delta^2$. Поскольку обе функции $e^{-2t}$ и $e^{-4t}$ убывают, $A^2 e^{-2t} + B^2 e^{-4t} \le A^2 + B^2 < \delta^2 = \epsilon^2$, и устойчивость показана.

Асимптотическая устойчивость вытекает из того факта, что
$$\lim_{t \to +\infty} e^{-2t} = \lim_{t \to +\infty} e^{-4t} = 0.$$

\end{document}
