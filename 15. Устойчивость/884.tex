% Page setup
\documentclass[a5paper,10pt]{article}

% https://github.com/InDevRus/filippov-solutions
% Page setup
\pagenumbering{gobble}

% Margin setup
\usepackage{geometry}
\geometry{left=1cm}
\geometry{right=1cm}
\geometry{top=1cm}
\geometry{bottom=1.5cm}

% For inserting tables
\usepackage{array}

% Formula aligning
\usepackage{amsmath}

% For formula diacritics
\usepackage{amsfonts}

% Theorem definitions
\usepackage{amsthm}
\theoremstyle{definition}
\newtheorem*{theorem}{Теорема}
\newtheorem*{corollary}{Следствие}
\newtheorem*{criteria}{Критерий}
\newtheorem*{algorithm}{Алгоритм}
\newtheorem*{formula}{Формула}
\newtheorem*{remark}{Замечание}
\newtheorem*{proposition}{Предложение}

% For bigger integral signs
\usepackage{bigints}

% For semantic advancements
\usepackage{enotez}

% For pictures
\usepackage{graphicx}

%For framing
\usepackage{framed}

% Automatic paragraph indentations
\usepackage{indentfirst}
\setlength{\parindent}{1em}

% Formula spacing configuration
\delimitershortfall-1sp
\usepackage{mleftright}
\mleftright

% For resizing
\usepackage{relsize}
% More convenient text-style common notations
\newcommand{\tpow}[2]{{#1}^{\mathlarger{#2}}}
\newcommand{\tint}{\displaystyle{\int}}
\newcommand{\tintlim}[2]{\displaystyle{\int\limits_{#1}^{#2}}}
\newcommand{\tbigint}{\displaystyle{\mathlarger{\int}}}
\newcommand{\tbigintlim}[2]{\displaystyle{\mathlarger{\int}\limits_{#1}^{#2}}}
\newcommand{\tsumlim}[2]{\displaystyle{\mathlarger{\sum}\displaylimits_{#1}^{#2}}}
\newcommand{\tprodlim}[2]{\displaystyle{\mathlarger{\prod}\displaylimits_{#1}^{#2}}}
\newcommand{\tmin}[1]{\min\limits_{#1}}
\newcommand{\tmax}[1]{\max\limits_{#1}}
\newcommand{\tlim}[1]{\lim\limits_{#1}}

% For floaty text
\usepackage{wrapfig}
\usepackage{floatflt}

% For cyrillic characters support
\usepackage[english, russian]{babel}

% For proper font
\usepackage[no-math]{fontspec}

% TNR within text
\setmainfont{Times New Roman}

% TNR within formulas
\usepackage{newtxmath}
\DeclareSymbolFont{operators}{OT1}{ntxtlf}{m}{n}
\SetSymbolFont{operators}{bold}{OT1}{ntxtlf}{b}{n}

% For degree sign
\usepackage{siunitx}

% Automatic brackets placement
\newcommand{\br}[1]{\left(#1\right)}
\newcommand{\vbr}[1]{\left|#1\right|}
\newcommand{\cbr}[1]{\left\{#1\right\}}
\newcommand{\rbr}[1]{\left[#1\right]}
\renewcommand{\le}{\leqslant}
\renewcommand{\ge}{\geqslant} 

% Automatic replacement for two greek letters
\renewcommand{\epsilon}{\varepsilon}
\renewcommand{\phi}{\varphi}

% Redefinition of some operators (to the appropriation of Russian notation)
\DeclareMathOperator{\arcsh}{arcsh}
\DeclareMathOperator{\arcch}{arcch}
\DeclareMathOperator{\arcth}{arcth}
\DeclareMathOperator{\arccth}{arccth}
\DeclareMathOperator{\rank}{rank}
\DeclareMathOperator{\inv}{inv}
\renewcommand{\Re}{\operatorname{Re}}
\renewcommand{\Im}{\operatorname{Im}}

\renewcommand{\gcd}{\text{НОД}}
\newcommand{\lcm}{\text{НОК}}

% Restrict inlne formula breaking
\binoppenalty=10000 
\relpenalty=10000

% Greek letters setup
\usepackage{textalpha}

\newcommand{\zed}{\textit{z}}

\newcommand{\Alpha}{\text{A}}
\newcommand{\Beta}{\text{B}}
\newcommand{\Epsilon}{\text{E}}
\newcommand{\Zeta}{\text{Z}}
\newcommand{\Eta}{\text{H}}
\newcommand{\Iota}{\text{I}}
\newcommand{\Kappa}{\text{K}}
\newcommand{\Mu}{\text{M}}
\newcommand{\Nu}{\text{N}}
\newcommand{\Omicron}{\text{O}}
\newcommand{\Rho}{\text{P}}
\newcommand{\Tau}{\text{T}}
\newcommand{\Chi}{\text{X}}

\renewcommand{\alpha}{\text{\textalpha}}
\renewcommand{\beta}{\text{\textbeta}}
\renewcommand{\gamma}{\text{\textgamma}}
\renewcommand{\delta}{\text{\textdelta}}
\renewcommand{\epsilon}{\text{\textepsilon}}
\renewcommand{\zeta}{\text{\textzeta}}
\renewcommand{\eta}{\text{\texteta}}
\renewcommand{\theta}{\text{\texttheta}}
\renewcommand{\iota}{\text{\textiota}}
\renewcommand{\kappa}{\text{\textkappa}}
\renewcommand{\lambda}{\text{\textlambda}}
\renewcommand{\mu}{\text{\textmu}}
\renewcommand{\nu}{\text{\textnu}}
\renewcommand{\xi}{\text{\textxi}}
\newcommand{\omicron}{\text{\textomicron}}
\renewcommand{\pi}{\text{\textpi}}
\renewcommand{\rho}{\text{\textrho}}
\renewcommand{\sigma}{\text{\textsigma}}
\renewcommand{\tau}{\text{\texttau}}
\renewcommand{\upsilon}{\text{\textupsilon}}
\renewcommand{\phi}{\text{\textphi}}
\renewcommand{\chi}{\text{\textchi}}
\renewcommand{\psi}{\text{\textpsi}}
\renewcommand{\omega}{\text{\textomega}}






\usepackage{pgfplots}
\pgfplotsset{compat=newest}
\pgfplotsset{
    every axis/.append style = {
        xlabel={$x$},
        ylabel={$y$}, 
        axis lines = middle,
        unit vector ratio = 1 1,
        font=\small
    },
    every axis plot/.append style = {smooth, samples = 100}
}

\begin{document}

$\dfrac {dy} {dx} = \dfrac {\dot{y}} {\dot{x}} = -\dfrac {y} {x}$.
$x y'_{x}\br{x} + y = 0$. 
$xy = C$. $y = \dfrac {C} {x}$ и траектории системы представляют собой гиперболы, обе асимптоты которых суть оси фазовой плоскости.

На рисунке показаны траектории вместе с направляющими векторами, посчитанными в некоторых точках вдоль траекторий. Характер направлений указывает на то, что решения удаляются от начала координат, то есть от нулевого решения $\Phi\br{t} = \begin{pmatrix} 0 \\ 0 \end{pmatrix}$, так что решение неустойчиво, а особая точка $\left\{ \begin{matrix} x = 0 \\ y = 0 \end{matrix} \right.$ относится к типу "седло".

\begin{figure}[h]
    \begin{center}
    \begin{tikzpicture}
        \pgfplotsset{
        VectorGraph/.style = {
            -stealth,
            quiver = {
                u = -x / (x^2 + y^2) ^ 0.5, 
                v = y / (x^2 + y^2) ^ 0.5, 
                scale arrows = 0.2}
            },
        every axis plot/.append style = {thin, gray}
        }
        \begin{axis}[
            thick,
            width = \textwidth,
            height = 10cm,
            ymin = -3.4,
            ymax = 3.4,
            xmin = -3.2,
            xmax = 3.2, 
            xtick = {-3, ..., 3},
            ytick = {-5, ..., 5},
            minor tick num = 3,
            declare function = {
                f(\C, \x) = \C / x;
            }]
            
            \addplot[domain=-4:-0.05] {f(0.5, x)};
            
            \addplot[domain=0.05:4] {f(0.5, x)};
            
            \addplot[domain=-4:-0.05] {f(-0.5, x)};
            \addplot[domain=0.05:4] {f(-0.5, x)};
            \addplot[domain=-4:-0.05] {f(1, x)};
            \addplot[domain=0.05:4] {f(1, x)};
            \addplot[domain=-4:-0.05] {f(-1, x)};
            \addplot[domain=0.05:4] {f(-1, x)};
            \addplot[domain=-4:-0.05] {f(2, x)};
            \addplot[domain=0.05:4] {f(2, x)};
            \addplot[domain=-4:-0.05] {f(-2, x)};
            \addplot[domain=0.05:4] {f(-2, x)};
            
            \addplot[VectorGraph, samples = 6, domain = -3 : -0.2] {f(0.5, x)};
            \addplot[VectorGraph, samples = 6, domain = 0.2 : 3] {f(0.5, x)};
            \addplot[VectorGraph, samples = 6, domain = -3 : -0.2] {f(-0.5, x)};
            \addplot[VectorGraph, samples = 6, domain = 0.2 : 3] {f(-0.5, x)};
            \addplot[VectorGraph, samples = 6, domain = -3 : -0.2] {f(1, x)};
            \addplot[VectorGraph, samples = 6, domain = 0.2 : 3] {f(1, x)};
            \addplot[VectorGraph, samples = 6, domain = -3 : -0.2] {f(-1, x)};
            \addplot[VectorGraph, samples = 6, domain = 0.2 : 3] {f(-1, x)};
            \addplot[VectorGraph, samples = 6, domain = -3 : -0.2] {f(2, x)};
            \addplot[VectorGraph, samples = 6, domain = 0.2 : 3] {f(2, x)};
            \addplot[VectorGraph, samples = 6, domain = -3 : -0.2] {f(-2, x)};
            \addplot[VectorGraph, samples = 6, domain = 0.2 : 3] {f(-2, x)};
        \end{axis}
    \end{tikzpicture}
    \end{center}
\end{figure}

Докажем по определению неустойчивость. Общее решение равно $X\br{t} = \begin{pmatrix} A e^{-t} \\ B e^{t} \end{pmatrix}$. Для $t \ge t_{0} = 0$
\begin{align*}
\br{X\br{t} - \Phi\br{t}}^2 = A^2 e^{-2t} + B^2 e^{2t} &&
\br{X\br{t_{0}} - \Phi\br{t_{0}}}^2 = A^2 + B^2
\end{align*}

Для $\epsilon = 1$ и любого $\delta > 0$ потребуем, чтобы $A$ и $B$ были таковы, что $0 < A^2 + B^2 < \delta^2$. С другой стороны, при
$t \ge \max\br{\ln \dfrac {1} {\vbr{B}}; 0}$ имеем 
$A^2 e^{-2t} + B^2 e^{2t} \ge B^2 e^{2t} \ge 1 = \epsilon^2$, поэтому решение неустойчиво.

\end{document}
