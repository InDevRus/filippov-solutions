\documentclass[a5paper,10pt]{article}

% https://github.com/InDevRus/filippov-solutions
% Page setup
\pagenumbering{gobble}

% Margin setup
\usepackage{geometry}
\geometry{left=1cm}
\geometry{right=1cm}
\geometry{top=1cm}
\geometry{bottom=1.5cm}

% For inserting tables
\usepackage{array}

% Formula aligning
\usepackage{amsmath}

% For formula diacritics
\usepackage{amsfonts}

% Theorem definitions
\usepackage{amsthm}
\theoremstyle{definition}
\newtheorem*{theorem}{Теорема}
\newtheorem*{corollary}{Следствие}
\newtheorem*{criteria}{Критерий}
\newtheorem*{algorithm}{Алгоритм}
\newtheorem*{formula}{Формула}
\newtheorem*{remark}{Замечание}
\newtheorem*{proposition}{Предложение}

% For bigger integral signs
\usepackage{bigints}

% For semantic advancements
\usepackage{enotez}

% For pictures
\usepackage{graphicx}

%For framing
\usepackage{framed}

% Automatic paragraph indentations
\usepackage{indentfirst}
\setlength{\parindent}{1em}

% Formula spacing configuration
\delimitershortfall-1sp
\usepackage{mleftright}
\mleftright

% For resizing
\usepackage{relsize}
% More convenient text-style common notations
\newcommand{\tpow}[2]{{#1}^{\mathlarger{#2}}}
\newcommand{\tint}{\displaystyle{\int}}
\newcommand{\tintlim}[2]{\displaystyle{\int\limits_{#1}^{#2}}}
\newcommand{\tbigint}{\displaystyle{\mathlarger{\int}}}
\newcommand{\tbigintlim}[2]{\displaystyle{\mathlarger{\int}\limits_{#1}^{#2}}}
\newcommand{\tsumlim}[2]{\displaystyle{\mathlarger{\sum}\displaylimits_{#1}^{#2}}}
\newcommand{\tprodlim}[2]{\displaystyle{\mathlarger{\prod}\displaylimits_{#1}^{#2}}}
\newcommand{\tmin}[1]{\min\limits_{#1}}
\newcommand{\tmax}[1]{\max\limits_{#1}}
\newcommand{\tlim}[1]{\lim\limits_{#1}}

% For floaty text
\usepackage{wrapfig}
\usepackage{floatflt}

% For cyrillic characters support
\usepackage[english, russian]{babel}

% For proper font
\usepackage[no-math]{fontspec}

% TNR within text
\setmainfont{Times New Roman}

% TNR within formulas
\usepackage{newtxmath}
\DeclareSymbolFont{operators}{OT1}{ntxtlf}{m}{n}
\SetSymbolFont{operators}{bold}{OT1}{ntxtlf}{b}{n}

% For degree sign
\usepackage{siunitx}

% Automatic brackets placement
\newcommand{\br}[1]{\left(#1\right)}
\newcommand{\vbr}[1]{\left|#1\right|}
\newcommand{\cbr}[1]{\left\{#1\right\}}
\newcommand{\rbr}[1]{\left[#1\right]}
\renewcommand{\le}{\leqslant}
\renewcommand{\ge}{\geqslant} 

% Automatic replacement for two greek letters
\renewcommand{\epsilon}{\varepsilon}
\renewcommand{\phi}{\varphi}

% Redefinition of some operators (to the appropriation of Russian notation)
\DeclareMathOperator{\arcsh}{arcsh}
\DeclareMathOperator{\arcch}{arcch}
\DeclareMathOperator{\arcth}{arcth}
\DeclareMathOperator{\arccth}{arccth}
\DeclareMathOperator{\rank}{rank}
\DeclareMathOperator{\inv}{inv}
\renewcommand{\Re}{\operatorname{Re}}
\renewcommand{\Im}{\operatorname{Im}}

\renewcommand{\gcd}{\text{НОД}}
\newcommand{\lcm}{\text{НОК}}

% Restrict inlne formula breaking
\binoppenalty=10000 
\relpenalty=10000

% Greek letters setup
\usepackage{textalpha}

\newcommand{\zed}{\textit{z}}

\newcommand{\Alpha}{\text{A}}
\newcommand{\Beta}{\text{B}}
\newcommand{\Epsilon}{\text{E}}
\newcommand{\Zeta}{\text{Z}}
\newcommand{\Eta}{\text{H}}
\newcommand{\Iota}{\text{I}}
\newcommand{\Kappa}{\text{K}}
\newcommand{\Mu}{\text{M}}
\newcommand{\Nu}{\text{N}}
\newcommand{\Omicron}{\text{O}}
\newcommand{\Rho}{\text{P}}
\newcommand{\Tau}{\text{T}}
\newcommand{\Chi}{\text{X}}

\renewcommand{\alpha}{\text{\textalpha}}
\renewcommand{\beta}{\text{\textbeta}}
\renewcommand{\gamma}{\text{\textgamma}}
\renewcommand{\delta}{\text{\textdelta}}
\renewcommand{\epsilon}{\text{\textepsilon}}
\renewcommand{\zeta}{\text{\textzeta}}
\renewcommand{\eta}{\text{\texteta}}
\renewcommand{\theta}{\text{\texttheta}}
\renewcommand{\iota}{\text{\textiota}}
\renewcommand{\kappa}{\text{\textkappa}}
\renewcommand{\lambda}{\text{\textlambda}}
\renewcommand{\mu}{\text{\textmu}}
\renewcommand{\nu}{\text{\textnu}}
\renewcommand{\xi}{\text{\textxi}}
\newcommand{\omicron}{\text{\textomicron}}
\renewcommand{\pi}{\text{\textpi}}
\renewcommand{\rho}{\text{\textrho}}
\renewcommand{\sigma}{\text{\textsigma}}
\renewcommand{\tau}{\text{\texttau}}
\renewcommand{\upsilon}{\text{\textupsilon}}
\renewcommand{\phi}{\text{\textphi}}
\renewcommand{\chi}{\text{\textchi}}
\renewcommand{\psi}{\text{\textpsi}}
\renewcommand{\omega}{\text{\textomega}}






\begin{document}

Предположим обратное, то есть то, что одно решение $\Phi\br{t}$ устойчиво, но какое-то другое $\Psi\br{t}$ неустойчиво. Это значит, что для некоторого $\epsilon > 0$ и любого $\delta > 0$ существует другое ненулевое решение $X_{\delta}\br{t}$ и такое $t_{1} = t_{1}\br{\delta}> t_{0}$, что $\vbr{\Psi\br{t_{0}} - X_{\delta}\br{t_{0}}} < \delta$, но $\vbr{\Psi\br{t_{1}} - X_{\delta}\br{t_{1}}} \ge \epsilon$.

Система уравнений линейная, значит всякое её решение можно выразить в виде 
$C_{1} f_{1}\br{t} + C_{2} f_{2}\br{t} + \ldots + C_{n} f_{n}\br{t} + g\br{t}$,
где $g\br{t}$ -- частное решение неоднородной системы; $f_{i}\br{t}$, $1 \le i \le n$ -- это столбцы,  составляющие фундаментальную систему решений. В ней любую функцию можно умножить на ненулевой скаляр и полученная ФСР всё равно останется корректной.

Подберём фундаментальную систему решений надлежащим образом. Возьмём любую ФСР и $t_{0}$ такое, чтобы векторы $f_{1}\br{t_{0}}$, $\ldots$, $f_{n}\br{t_{0}}$ были линейно независимы (то есть матрица $A$, составленная из этих векторов-столбцов, имела $\det A \ne 0$). Далее, для положительного числа $\epsilon_{\Phi} = \dfrac {\epsilon} {n}$ подберём такое $\gamma > 0$, что для любого решения $X\br{t}$ из $\vbr{\Phi\br{t_{0}} - X\br{t_{0}}} < \gamma$ следует $\vbr{\Phi\br{t} - X\br{t}} < \dfrac {\epsilon} {n}$ для всех $t \ge t_{0}$. Наконец, в выбранной ФСР умножим каждую функцию на такие ненулевые скаляры, чтобы выполнялись $\vbr{f_{k}\br{t_{0}} - \Phi\br{t_{0}}} < \gamma$, $1 \le k \le n$. Из этого следуют неравенства $\vbr{f_{k}\br{t} - \Phi\br{t}} < \dfrac {\epsilon} {n}$, а линейная независимость векторов $f_{1}\br{t_{0}}$, $\ldots$, $f_{n}\br{t_{0}}$ не нарушается, так как умножение столбцов матрицы $A$ приведёт к умножению её определителя на ненулевые числа. Полученный набор функций положим новой ФСР. Стоит обратить особое внимание на тот факт, что построение ФСР указанным образом никак не зависит ни от $\delta$, ни от $t_{1} = t_{1}\br{\delta}$.

В терминах подобранной выше ФСР распишем
\begin{align*}
\Psi\br{t} = A_{1} f_{1}\br{t} + \ldots + A_{n} f_{n}\br{t} + g\br{t} 
&& X_{\delta}\br{t} = B_{1} f_{1}\br{t} + \ldots + B_{n} f_{n}\br{t} + g\br{t}
\end{align*}
\begin{align*}
\Psi\br{t} - X_{\delta}\br{t} = \Delta_{1} f_{1}\br{t} + \ldots + \Delta_{n} f_{n}\br{t} 
&& \Delta_{k} = A_{k} - B_{k}
\end{align*}
Разность $\Psi\br{t_{0}} - X_{\delta}\br{t_{0}}$ можно представить в следующей форме:
$$
\begin{pmatrix}
\Delta_{1} f_{11}\br{t_{0}} + \ldots + \Delta_{1} f_{1n}\br{t_{0}} \\
\ldots \\
\Delta_{1} f_{n1}\br{t_{0}} + \ldots + \Delta_{n} f_{nn}\br{t_{0}} 
\end{pmatrix}
= 
\begin{pmatrix}
f_{11}\br{t_{0}} & \ldots & f_{1n}\br{t_{0}} \\
\ldots & \ldots & \ldots \\
f_{n1}\br{t_{0}} & \ldots & f_{nn}\br{t_{0}} \\
\end{pmatrix}
\times
\begin{pmatrix}
\Delta_{1} \\ \ldots \\ \Delta_{n}
\end{pmatrix}
= F\br{t_{0}} \times \Delta,
$$
где за $F_{n \times n}\br{t}$ обозначена матрица, составленная по столбцам из выбранной ФСР; $\Delta_{n \times 1}$ -- столбец разностей произвольных постоянных.  Матрица $F\br{t_{0}}$ фиксирована, при этом ещё и невырождена. Следовательно, справедливы представления
\begin{align*}
\Psi\br{t_{0}} - X_{\delta}\br{t_{0}} = F\br{t_{0}} \times \Delta 
&& \Delta = \br{F\br{t_{0}}}^{-1} \times \br{\Psi\br{t_{0}} - X_{\delta}\br{t_{0}}}.
\end{align*}

За норму квадратной матрицы далее обозначена норма Фробениуса $\norm{A}_{F}$. По свойству согласованности этой нормы со спектральной нормой $\norm{x}_{2}$
\begin{multline*}
\vbr{\Delta} = \norm{\Delta}_{2}
= \sqrt{\Delta_{1}^2 + \ldots + \Delta_{n}^2}
= \vbr{\br{F\br{t_{0}}}^{-1} \times \br{\Psi\br{t_{0}} - X_{\delta}\br{t_{0}}}} \le 
\\ \le \norm{\br{F\br{t_{0}}}^{-1}}_{F} \cdot \vbr{\br{\Psi\br{t_{0}} - X_{\delta}\br{t_{0}}}}
< \norm{\br{F\br{t_{0}}}^{-1}}_{F} \cdot \delta.
\end{multline*}

Обозначим также $M_{\delta} = \tmax{1 \le k \le n} \vbr{\Delta_{k}} > 0$. Выше показано, что
$$M_{\delta} \le \sqrt{\Delta_{1}^2 + \ldots + \Delta_{n}^2} < \norm{\br{F\br{t_{0}}}^{-1}}_{F} \cdot \delta.$$

Например, если рассмотреть
$$\delta = \dfrac {\epsilon} {\epsilon + n \vbr{\Phi\br{t_{1}}}} \br{\norm{\br{F\br{t_{0}}}^{-1}}_{F}}^{-1} \text{, то получим } M_{\delta} < \dfrac {\epsilon} {\epsilon + n \vbr{\Phi\br{t_{1}}}}.$$

С одной стороны, из $\vbr{\Psi\br{t_{1}} - X\br{t_{1}}} \ge \epsilon$ вытекает 
\begin{equation*}\begin{split}
\epsilon & \le \vbr{\Psi\br{t_{1}} - X\br{t_{1}}}
= \vbr{\Delta_{1} f_{1}\br{t_{1}} + \ldots + \Delta_{n} f_{n}\br{t_{n}}} 
\le \vbr{\Delta_{1} f_{1}\br{t_{1}}} + \ldots + \vbr{\Delta_{n} f_{n}\br{t_{n}}} \le
\\ & \le M_{\delta} \br{ \vbr{f_{1}\br{t_{1}}} + \ldots + \vbr{f_{n}\br{t_{1}}} } = \\
 & = M_{\delta} \br{ \vbr{f_{1}\br{t_{1}} - \Phi\br{t_{1}} + \Phi\br{t_{1}}} + \ldots + \vbr{f_{n}\br{t_{1}} - \Phi\br{t_{1}} + \Phi\br{t_{1}} } } \le \\
 & \le M_{\delta} \br{ \vbr{f_{1}\br{t_{1}} - \Phi\br{t_{1}}} + \ldots + \vbr{f_{n}\br{t_{1}} - \Phi\br{t_{1}}} + n \vbr{\Phi\br{t_{1}}} },
\end{split}\end{equation*}
а тогда
$$\vbr{f_{1}\br{t_{1}} - \Phi\br{t_{1}}} + \ldots + \vbr{f_{n}\br{t_{1}} - \Phi\br{t_{1}}} \ge \dfrac {\epsilon} {M_{\delta}} - n \vbr{\Phi\br{t_{1}}} > \dfrac {\epsilon} {\br{\dfrac {\epsilon} {\epsilon + n \vbr{\Phi\br{t_{1}}}}}} - n \vbr{\Phi\br{t_{1}}} = \epsilon.$$

С другой стороны, для всех $t \ge t_{0}$ имеем
$\vbr{f_{1}\br{t} - \Phi\br{t}} + \ldots + \vbr{f_{n}\br{t} - \Phi\br{t}} < \dfrac {\epsilon} {n} \cdot n = \epsilon.$
В частности, подстановка в последнее неравенство $t = t_{1}$ даёт противоречие, которое доказывает теорему.

\end{document}
