% Page setup
\documentclass[a5paper,10pt]{article}
 
% https://github.com/InDevRus/filippov-solutions
% Page setup
\pagenumbering{gobble}

% Margin setup
\usepackage{geometry}
\geometry{left=1cm}
\geometry{right=1cm}
\geometry{top=1cm}
\geometry{bottom=1.5cm}

% For inserting tables
\usepackage{array}

% Formula aligning
\usepackage{amsmath}

% For formula diacritics
\usepackage{amsfonts}

% Theorem definitions
\usepackage{amsthm}
\theoremstyle{definition}
\newtheorem*{theorem}{Теорема}
\newtheorem*{corollary}{Следствие}
\newtheorem*{criteria}{Критерий}
\newtheorem*{algorithm}{Алгоритм}
\newtheorem*{formula}{Формула}
\newtheorem*{remark}{Замечание}
\newtheorem*{proposition}{Предложение}

% For bigger integral signs
\usepackage{bigints}

% For semantic advancements
\usepackage{enotez}

% For pictures
\usepackage{graphicx}

%For framing
\usepackage{framed}

% Automatic paragraph indentations
\usepackage{indentfirst}
\setlength{\parindent}{1em}

% Formula spacing configuration
\delimitershortfall-1sp
\usepackage{mleftright}
\mleftright

% For resizing
\usepackage{relsize}
% More convenient text-style common notations
\newcommand{\tpow}[2]{{#1}^{\mathlarger{#2}}}
\newcommand{\tint}{\displaystyle{\int}}
\newcommand{\tintlim}[2]{\displaystyle{\int\limits_{#1}^{#2}}}
\newcommand{\tbigint}{\displaystyle{\mathlarger{\int}}}
\newcommand{\tbigintlim}[2]{\displaystyle{\mathlarger{\int}\limits_{#1}^{#2}}}
\newcommand{\tsumlim}[2]{\displaystyle{\mathlarger{\sum}\displaylimits_{#1}^{#2}}}
\newcommand{\tprodlim}[2]{\displaystyle{\mathlarger{\prod}\displaylimits_{#1}^{#2}}}
\newcommand{\tmin}[1]{\min\limits_{#1}}
\newcommand{\tmax}[1]{\max\limits_{#1}}
\newcommand{\tlim}[1]{\lim\limits_{#1}}

% For floaty text
\usepackage{wrapfig}
\usepackage{floatflt}

% For cyrillic characters support
\usepackage[english, russian]{babel}

% For proper font
\usepackage[no-math]{fontspec}

% TNR within text
\setmainfont{Times New Roman}

% TNR within formulas
\usepackage{newtxmath}
\DeclareSymbolFont{operators}{OT1}{ntxtlf}{m}{n}
\SetSymbolFont{operators}{bold}{OT1}{ntxtlf}{b}{n}

% For degree sign
\usepackage{siunitx}

% Automatic brackets placement
\newcommand{\br}[1]{\left(#1\right)}
\newcommand{\vbr}[1]{\left|#1\right|}
\newcommand{\cbr}[1]{\left\{#1\right\}}
\newcommand{\rbr}[1]{\left[#1\right]}
\renewcommand{\le}{\leqslant}
\renewcommand{\ge}{\geqslant} 

% Automatic replacement for two greek letters
\renewcommand{\epsilon}{\varepsilon}
\renewcommand{\phi}{\varphi}

% Redefinition of some operators (to the appropriation of Russian notation)
\DeclareMathOperator{\arcsh}{arcsh}
\DeclareMathOperator{\arcch}{arcch}
\DeclareMathOperator{\arcth}{arcth}
\DeclareMathOperator{\arccth}{arccth}
\DeclareMathOperator{\rank}{rank}
\DeclareMathOperator{\inv}{inv}
\renewcommand{\Re}{\operatorname{Re}}
\renewcommand{\Im}{\operatorname{Im}}

\renewcommand{\gcd}{\text{НОД}}
\newcommand{\lcm}{\text{НОК}}

% Restrict inlne formula breaking
\binoppenalty=10000 
\relpenalty=10000

% Greek letters setup
\usepackage{textalpha}

\newcommand{\zed}{\textit{z}}

\newcommand{\Alpha}{\text{A}}
\newcommand{\Beta}{\text{B}}
\newcommand{\Epsilon}{\text{E}}
\newcommand{\Zeta}{\text{Z}}
\newcommand{\Eta}{\text{H}}
\newcommand{\Iota}{\text{I}}
\newcommand{\Kappa}{\text{K}}
\newcommand{\Mu}{\text{M}}
\newcommand{\Nu}{\text{N}}
\newcommand{\Omicron}{\text{O}}
\newcommand{\Rho}{\text{P}}
\newcommand{\Tau}{\text{T}}
\newcommand{\Chi}{\text{X}}

\renewcommand{\alpha}{\text{\textalpha}}
\renewcommand{\beta}{\text{\textbeta}}
\renewcommand{\gamma}{\text{\textgamma}}
\renewcommand{\delta}{\text{\textdelta}}
\renewcommand{\epsilon}{\text{\textepsilon}}
\renewcommand{\zeta}{\text{\textzeta}}
\renewcommand{\eta}{\text{\texteta}}
\renewcommand{\theta}{\text{\texttheta}}
\renewcommand{\iota}{\text{\textiota}}
\renewcommand{\kappa}{\text{\textkappa}}
\renewcommand{\lambda}{\text{\textlambda}}
\renewcommand{\mu}{\text{\textmu}}
\renewcommand{\nu}{\text{\textnu}}
\renewcommand{\xi}{\text{\textxi}}
\newcommand{\omicron}{\text{\textomicron}}
\renewcommand{\pi}{\text{\textpi}}
\renewcommand{\rho}{\text{\textrho}}
\renewcommand{\sigma}{\text{\textsigma}}
\renewcommand{\tau}{\text{\texttau}}
\renewcommand{\upsilon}{\text{\textupsilon}}
\renewcommand{\phi}{\text{\textphi}}
\renewcommand{\chi}{\text{\textchi}}
\renewcommand{\psi}{\text{\textpsi}}
\renewcommand{\omega}{\text{\textomega}}






\begin{document}
Для каждого пункта сначала будем находить общее решение $x = x\br{t}$, затем решение задачи Коши $\phi\br{t}$. Затем будем доказывать устойчивость или неустойчивость по Ляпунову.

а) $3\br{t - 1}\dot{x}\br{t} = x\br{t}$. 
$\dfrac {dx} {dt} \sqrt[3]{t - 1} - \dfrac {1} {3 \sqrt[3]{\br{t - 1}^{2}}} x\br{t} = 0$. 
$\dfrac {d}{dt} \br{\dfrac{x} {\sqrt[3]{t - 1}}} = 0$. 
Общее решение имеет вид
$x\br{t} = C \sqrt[3]{t - 1}$; решение задачи Коши -- это функция: $\phi\br{t} = 0$.
\begin{align*}
    \text{При } t \ge t_{0} = 1 \text{ имеем:} &&
    \vbr{x\br{t} - \phi\br{t}} = \vbr{C} \sqrt[3]{t - 1}, &&
    \vbr{x\br{t_{0}} - \phi\br{t_{0}}} = \vbr{C}.
\end{align*}

Условие устойчивости не выполняется, приведём пример. Пусть $\epsilon = 1$, $\delta > 0$ и $C$ такое, что $0 < \vbr{C} < \delta$. При $t \ge \tpow{\vbr{C}}{-3} + 1$ получаем $\vbr{C} \sqrt[3]{t - 1} \ge 1 = \epsilon$.

Таким образом, решение неустойчиво.

б) $\dot{x}\br{t} + \br{t^2 - 4} x\br{t} = 0$.
$\dot{x}\br{t} \tpow{e}{\frac {1} {3}t^3 - 4t} + \br{t^2 - 4} \tpow{e}{\frac {1} {3}t^3 - 4t} = 0$.
$\dfrac {d}{dt} \br{x\br{t} \tpow{e}{\frac {1} {3}t^3 - 4t}} = 0$.
$x\br{t} = C \tpow{e}{4t - \frac {1} {3}t^3}$ -- общее решение; $\phi\br{t} = 0$ -- решение задачи Коши.
\begin{align*}
    \text{При } t \ge t_{0} = 0 \text{ имеем:} &&
    \vbr{x\br{t} - \phi\br{t}} = \vbr{C} \tpow{e}{4t - \frac {1} {3}t^3}, &&
    \vbr{x\br{t_{0}} - \phi\br{t_{0}}} = \vbr{C}.
\end{align*}

Функция $f\br{t} = \tpow{e}{4t - \frac {1}{3}t^3}$ возрастает при $t\in \rbr{0; 2}$, убывает при $t \in \left[2; +\infty \right)$. Значит $t = 2$ -- точка максимума и $\tmax{\mathsmaller{t\in \left[0; +\infty \right)}}{f\br{t}} = f\br{2} = \tpow{e}{\frac {16} {3}}$.

Возьмём $\delta\br{\epsilon} = \tpow{e}{-\frac {16} {3}} \cdot \epsilon$ для наперёд выбранного $\epsilon > 0$. Если $\vbr{C} < \delta$, то $\vbr{C} \tpow{e}{4t - \frac {1} {3}t^3} \linebreak < \delta \tpow{e}{\frac {16} {3}} = \epsilon$, и показана устойчивость решения.

в) $\dot{x}\br{t} + x\br{t} = t$.
$\dot{x}\br{t} \tpow{e}{t} + x\br{t} \tpow{e}{t} = t \tpow{e}{t}$. 
$x\tpow{e}{t} = \br{t - 1}\tpow{e}{t} + C$. $x\br{t} = C\tpow{e}{-t} + t - 1$ -- общее решение; $\phi\br{t} = 2\tpow{e}{-t} + t - 1$ -- решение задачи Коши.
\begin{align*}
    \text{При } t \ge t_{0} = 0 \text{ имеем:} &&
    \vbr{x\br{t} - \phi\br{t}} = \vbr{C - 2} e^{-t}, &&
    \vbr{x\br{t_{0}} - \phi\br{t_{0}}} = \vbr{C - 2}.
\end{align*}
Для таких $t$ выполняется $\tpow{e}{-t} \le 1$, а потому при выборе $\delta\br{\epsilon} = \epsilon$ из $\vbr{C - 2} < \delta$ следует $\vbr{C - 2} \tpow{e}{-t} < \epsilon$. Итак, решение устойчиво.

г) Поделив на $x^3$, получаем 
$\br{\dfrac {1} {x^2}} - 2 \dfrac {1} {x^3} t x'\br{t} - 1 = 0$.
$\dfrac {d}{dt} \br{\dfrac {t}{x^2}} = 1$. 
$\dfrac {t} {x^2} = t + C$.
Итак, общее решение равняется $x\br{t} = \pm \sqrt{\dfrac {t} {t + C}}$; кроме того, есть дополнительное (но не особое) решение $\phi\br{t} = 0$, которое и является решением данной задачи Коши. В качестве переменного решения $x\br{t}$ будем рассматривать любое, кроме дополнительного.
\begin{align*}
    \text{При } t \ge t_{0} = 1 \text{ имеем:} &&
    \vbr{x\br{t} - \phi\br{t}} = \sqrt{\frac {t} {t + C}}, &&
    \vbr{x\br{t_{0}} - \phi\br{t_{0}}} = \sqrt{\frac {1} {1 + C}}.
\end{align*}
Кроме того, $C > 0$, так как, если $C < 0$, то решения непродолжаемы на $t \in \left[t_{0}; +\infty \right)$, а при $C = 0$ решение равняется $\pm 1$ и интереса не представляет.

Покажем, что решение неустойчиво. Возьмём $\epsilon = \dfrac {1} {2}$ и любое $\delta > 0$. Пусть $x\br{t}$ такое решение, что $\sqrt{\dfrac {1} {1 + C}} < \delta$, то есть $C > 1 + \dfrac {1} {\delta^2}$. Однако, всякая функция $f\br{t} = \sqrt{\dfrac {t} {t + C}}$ возрастает, значит при $t \ge \dfrac {1} {3} C > \dfrac {1} {3} \br{1 + \dfrac {1} {\delta^2}}$ получаем 
$\sqrt{\dfrac {t} {t + C}} \ge \sqrt{\dfrac {\frac {1} {3} C} {\frac {1} {3} C + C}} = \dfrac {1} {2} = \epsilon$.
Это означает неустойчивость решения.

\end{document}
