\documentclass[a5paper,10pt]{article}

% https://github.com/InDevRus/filippov-solutions
% Page setup
\pagenumbering{gobble}

% Margin setup
\usepackage{geometry}
\geometry{left=1cm}
\geometry{right=1cm}
\geometry{top=1cm}
\geometry{bottom=1.5cm}

% For inserting tables
\usepackage{array}

% Formula aligning
\usepackage{amsmath}

% For formula diacritics
\usepackage{amsfonts}

% Theorem definitions
\usepackage{amsthm}
\theoremstyle{definition}
\newtheorem*{theorem}{Теорема}
\newtheorem*{corollary}{Следствие}
\newtheorem*{criteria}{Критерий}
\newtheorem*{algorithm}{Алгоритм}
\newtheorem*{formula}{Формула}
\newtheorem*{remark}{Замечание}
\newtheorem*{proposition}{Предложение}

% For bigger integral signs
\usepackage{bigints}

% For semantic advancements
\usepackage{enotez}

% For pictures
\usepackage{graphicx}

%For framing
\usepackage{framed}

% Automatic paragraph indentations
\usepackage{indentfirst}
\setlength{\parindent}{1em}

% Formula spacing configuration
\delimitershortfall-1sp
\usepackage{mleftright}
\mleftright

% For resizing
\usepackage{relsize}
% More convenient text-style common notations
\newcommand{\tpow}[2]{{#1}^{\mathlarger{#2}}}
\newcommand{\tint}{\displaystyle{\int}}
\newcommand{\tintlim}[2]{\displaystyle{\int\limits_{#1}^{#2}}}
\newcommand{\tbigint}{\displaystyle{\mathlarger{\int}}}
\newcommand{\tbigintlim}[2]{\displaystyle{\mathlarger{\int}\limits_{#1}^{#2}}}
\newcommand{\tsumlim}[2]{\displaystyle{\mathlarger{\sum}\displaylimits_{#1}^{#2}}}
\newcommand{\tprodlim}[2]{\displaystyle{\mathlarger{\prod}\displaylimits_{#1}^{#2}}}
\newcommand{\tmin}[1]{\min\limits_{#1}}
\newcommand{\tmax}[1]{\max\limits_{#1}}
\newcommand{\tlim}[1]{\lim\limits_{#1}}

% For floaty text
\usepackage{wrapfig}
\usepackage{floatflt}

% For cyrillic characters support
\usepackage[english, russian]{babel}

% For proper font
\usepackage[no-math]{fontspec}

% TNR within text
\setmainfont{Times New Roman}

% TNR within formulas
\usepackage{newtxmath}
\DeclareSymbolFont{operators}{OT1}{ntxtlf}{m}{n}
\SetSymbolFont{operators}{bold}{OT1}{ntxtlf}{b}{n}

% For degree sign
\usepackage{siunitx}

% Automatic brackets placement
\newcommand{\br}[1]{\left(#1\right)}
\newcommand{\vbr}[1]{\left|#1\right|}
\newcommand{\cbr}[1]{\left\{#1\right\}}
\newcommand{\rbr}[1]{\left[#1\right]}
\renewcommand{\le}{\leqslant}
\renewcommand{\ge}{\geqslant} 

% Automatic replacement for two greek letters
\renewcommand{\epsilon}{\varepsilon}
\renewcommand{\phi}{\varphi}

% Redefinition of some operators (to the appropriation of Russian notation)
\DeclareMathOperator{\arcsh}{arcsh}
\DeclareMathOperator{\arcch}{arcch}
\DeclareMathOperator{\arcth}{arcth}
\DeclareMathOperator{\arccth}{arccth}
\DeclareMathOperator{\rank}{rank}
\DeclareMathOperator{\inv}{inv}
\renewcommand{\Re}{\operatorname{Re}}
\renewcommand{\Im}{\operatorname{Im}}

\renewcommand{\gcd}{\text{НОД}}
\newcommand{\lcm}{\text{НОК}}

% Restrict inlne formula breaking
\binoppenalty=10000 
\relpenalty=10000

% Greek letters setup
\usepackage{textalpha}

\newcommand{\zed}{\textit{z}}

\newcommand{\Alpha}{\text{A}}
\newcommand{\Beta}{\text{B}}
\newcommand{\Epsilon}{\text{E}}
\newcommand{\Zeta}{\text{Z}}
\newcommand{\Eta}{\text{H}}
\newcommand{\Iota}{\text{I}}
\newcommand{\Kappa}{\text{K}}
\newcommand{\Mu}{\text{M}}
\newcommand{\Nu}{\text{N}}
\newcommand{\Omicron}{\text{O}}
\newcommand{\Rho}{\text{P}}
\newcommand{\Tau}{\text{T}}
\newcommand{\Chi}{\text{X}}

\renewcommand{\alpha}{\text{\textalpha}}
\renewcommand{\beta}{\text{\textbeta}}
\renewcommand{\gamma}{\text{\textgamma}}
\renewcommand{\delta}{\text{\textdelta}}
\renewcommand{\epsilon}{\text{\textepsilon}}
\renewcommand{\zeta}{\text{\textzeta}}
\renewcommand{\eta}{\text{\texteta}}
\renewcommand{\theta}{\text{\texttheta}}
\renewcommand{\iota}{\text{\textiota}}
\renewcommand{\kappa}{\text{\textkappa}}
\renewcommand{\lambda}{\text{\textlambda}}
\renewcommand{\mu}{\text{\textmu}}
\renewcommand{\nu}{\text{\textnu}}
\renewcommand{\xi}{\text{\textxi}}
\newcommand{\omicron}{\text{\textomicron}}
\renewcommand{\pi}{\text{\textpi}}
\renewcommand{\rho}{\text{\textrho}}
\renewcommand{\sigma}{\text{\textsigma}}
\renewcommand{\tau}{\text{\texttau}}
\renewcommand{\upsilon}{\text{\textupsilon}}
\renewcommand{\phi}{\text{\textphi}}
\renewcommand{\chi}{\text{\textchi}}
\renewcommand{\psi}{\text{\textpsi}}
\renewcommand{\omega}{\text{\textomega}}






\begin{document}

\newtheorem{step}{Шаг}

\begin{step} {
Покажем, что любая краевая задача, в которой $a^2 + b^2 > 0$, всегда сводится к одной из двух краевых задач

$$
\begin{cases} \phi''\br{\xi} = -q\br{\xi}\phi\br{\xi} \\ \phi\br{\xi_{1}} = 1 \\ \phi\br{\xi_{2}} = \gamma \end{cases}
\text{-- задача А, }
$$
\begin{center} либо \end{center}
\begin{equation*}
\begin{cases} \phi''\br{\xi} = -q\br{\xi}\phi\br{\xi} \\ \phi\br{\xi_{1}} = 0 \\ \phi\br{\xi_{2}} = 1 \end{cases}
\text{-- задача Б,}
\end{equation*}

в каждой из которых обязательно $\xi_{1} < \xi_{2}$, а $\gamma$ -- число, равное либо $\dfrac {a} {b}$, либо $\dfrac {b} {a}$, в том числе возможно $\gamma = 0$.

} \end{step}

\begin{itemize}
    \item {
    Замена $x = -\xi$ не меняет вид уравнения, поскольку $dx = -d\xi$, $d^2 x = d^2 \xi$, но позволяет поменять местами точки $x_{1}$ и $x_{2}$, чтобы гарантированно получить $\xi_{1} < \xi_{2}$. Расстояние между точками $\xi_{1}$ и $\xi_{2}$ будет таким же, как и расстояние между исходными $x_{1}$ и $x_{2}$.
    }
    \item {
    Замена $\xi = \alpha + \check{\xi}$ также не не меняет вид уравнения, но позволяет передвинуть точки $\xi_{1}$ и $\xi_{2}$ на те же места, на которых были точки $x_{1}$ и $x_{2}$ изначально.
    }

    \item {
    Наконец, замена $y\br{x} = k \phi\br{x}$, где $k \ne 0$, не меняет вид уравнения. С надлежащим выбором числа $k$ можно привести исходную краевую задачу либо к задаче А, либо к задаче Б. Подробнее говоря, исходная краевая задача сводится к задаче Б, если и только если после первых двух замен и получения $\xi_{1} < \xi_{2}$ оказывается, что $y\br{\xi_{1}} = 0$.
    }
\end{itemize}

Случай, когда $a = b = 0$ -- особый, и для него этими заменами можно добиться, чтобы $\xi_{1} < \xi_{2}$.

Таким образом, для существования и единственности решения исходной краевой задачи необходимо и достаточно, чтобы задачи А и Б и особая задача имели единственное решение.

\begin{step} { Найдём фундаментальную систему решений. } \end{step}

Общее решение имеет вид $\phi\br{\xi} = A \phi_{1}\br{\xi} + B \phi_{2}\br{\xi}$, где $\phi_{1}\br{\xi}$, $\phi_{2}\br{\xi}$ -- линейно независимые частные решения уравнения $\phi'\br{\xi} = -q\br{\xi}\phi\br{\xi}$.

Рассмотрим задачу Коши
$$\begin{cases}
    \phi_{1}''\br{\xi} = -q\br{\xi}\phi_{1}\br{\xi} \\
    \phi_{1}\br{\xi_{1}} = 1 \\
    \phi_{1}'\br{\xi_{1}} = 1
\end{cases}$$

и положим $\phi_{1}\br{\xi}$ её решением. Поскольку $\phi_{1}''\br{\xi} = -q\br{\xi}\phi_{1}\br{\xi}$, а $q\br{\xi} \le 0$, \linebreak $\phi_{1}\br{\xi_{1}} > 0$, $\phi'_{1}\br{\xi_{1}} > 0$, имеем $\phi_{1}\br{\xi} \ge \phi_{1}\br{\xi_{1}} = 1$ при $\xi \in \left[ \xi_{1}; +\infty \right)$ в силу доказанного в задаче 723. При этом всём $\phi_{1}''\br{\xi} = -q\br{\xi}\phi_{1}\br{\xi} \ge 0$, а значит $\phi'_{1}\br{\xi} \ge \phi'_{1}\br{\xi_{1}} = 1$ и $\phi_{1}\br{\xi}$ строго монотонно возрастает на $\xi \in \left[ \xi_{1}; +\infty \right)$.

По формуле Остроградского-Лиувилля находим
$$\phi_{2}\br{\xi} = \phi_{1}\br{\xi} \br{\tintlim{\xi_{1}}{\xi} { \frac{1} {\phi^2_{1}\br{t}} \ dt} + C}.$$

При этом интеграл $\tintlim{\xi_{1}}{\xi} { \dfrac {1}{\phi^{2}\br{t}} \ dt}$ -- собственный для всех $\xi \in \left[ \xi_{1}; +\infty \right)$, поскольку $\phi_{1}\br{\xi} \ge 1$. Выбор постоянной $C$ произволен, и для простоты доказательства выберем $C = 0$, то есть

$$\phi_{2}\br{\xi} = \phi_{1}\br{\xi} \cdot \tintlim{\xi_{1}}{\xi} { \frac{1} {\phi^2_{1}\br{t}} \ dt}.$$

В силу этого выбора получаем $\phi_{2}\br{\xi_{1}} = 0$, но 

$$\phi_{2}\br{\xi_{2}} = \phi_{1}\br{\xi_{2}} \tintlim{\xi_{1}}{\xi_{2}} { \frac{1} {\phi^2_{1}\br{t}} \ dt} > 0 .$$

\begin{step} Докажем существование и единственность решения. \end{step}

В общем случае преобразованная краевая задача равносильна системе линейных уравнений 
$$\begin{cases}
    \phi\br{\xi_{1}} = A\phi_{1}\br{\xi_{1}} + B\phi_{2}\br{\xi_{1}} \\ 
    \phi\br{\xi_{2}} = A\phi_{1}\br{\xi_{2}} + B\phi_{2}\br{\xi_{2}}
\end{cases}$$

относительно чисел $A$, $B$. Поскольку $\phi_{2}\br{\xi_{1}} = 0$, определитель системы
$$\Delta_{0} = \phi_{1}\br{\xi_{1}} \phi_{2}\br{\xi_{1}}$$
ненулевой.

\newtheorem{case}{Cлучай}

\begin{case} $a = b = 0$. \end{case}

Особый случай преобразуется в краевую задачу 
$$
\begin{cases} \phi''\br{\xi} = -q\br{\xi}\phi\br{\xi} \\ \phi\br{\xi_{1}} = 0 \\ \phi\br{\xi_{2}} = 0 \end{cases}
$$
а система уравнений становится однородной
$$\begin{cases}
    0 = A\phi_{1}\br{\xi_{1}} + B\phi_{2}\br{\xi_{1}} \\ 
    0 = A\phi_{1}\br{\xi_{2}} + B\phi_{2}\br{\xi_{2}}
\end{cases}$$

Эта система уравнений имеет только тривиальное решение, а искомая функция \linebreak $y\br{x} = 0$ -- единственное решение исходной краевой задачи. 

\begin{case} $a^2 + b^2 > 0$. \end{case}

Задача сводится либо к задаче А, либо задаче Б. Для них система уравнений будет иметь вид

$$
\left\{
\begin{aligned} 
    & 1 = A\phi_{1}\br{\xi_{1}} + B\phi_{2}\br{\xi_{1}} \\
    & \gamma = A\phi_{1}\br{\xi_{2}} + B\phi_{2}\br{\xi_{2}}
\end{aligned} \right.
\text{ и }
\left\{
\begin{aligned}
    & 0 = A\phi_{1}\br{\xi_{1}} + B\phi_{2}\br{\xi_{1}} \\ 
    & 1 = A\phi_{1}\br{\xi_{2}} + B\phi_{2}\br{\xi_{2}}
\end{aligned} \right.
$$

соответственно для задач А и Б. В любом из этих двух случаев система неоднородна, а поскольку определитель системы $\Delta_{0}$ отличен от нуля, система имеет единственное решение. Окончательно, решение преобразованной краевой задачи существует и единственно, что и требовалось доказать.

\begin{step} Докажем монотонность решения не только при $b = 0$, а вообще во всех случаях, когда $ab = 0$. \end{step}

Особый случай $a = b = 0$ соответствует решению $y\br{x} = 0$, которое монотонно.

Если $ab = 0$, то исходная краевая задача сводится либо к задаче Б, либо к задаче А, в которой $\gamma = 0$. Важен тот факт, что все перечисленные замены переменных могут лишь изменить направление монотонности решения $\phi\br{\xi}$ относительно решения $y\br{x}$, но сам факт наличия или отсутствия монотонности остаётся неизменным.

Покажем, что задача Б в вопросе монотонности решения сводится к задаче А.

Пусть дана задача Б. В силу уже доказанного решение этой задачи существует и единственно для всех $\xi$. Далее выберем любую точку $\xi_{3} < \xi_{1}$ и рассмотрим новую краевую задачу
$$
\begin{cases} \phi''\br{\xi} = -q\br{\xi}\phi\br{\xi} \\ 
\phi\br{\xi_{3}} = \phi\br{\xi_{3}} \ne 0 \\
\phi\br{\xi_{1}} = 0 \end{cases}
$$
задающую такую же функцию, как и данная задача Б. Поскольку $\xi_{3} < \xi_{1}$, заменами переменных её можно свести к задаче А. Решение полученной задачи будет монотонным тогда и только тогда, когда будет монотонным решение данной задачи Б.

Итак, остаётся доказать, что решение задачи А
$$\begin{cases} \phi''\br{\xi} = -q\br{\xi}\phi\br{\xi} \\
\phi\br{\xi_{1}} = 1 \\
\phi\br{\xi_{2}} = 0 \end{cases}$$
монотонно убывает.

Задача приводит к системе уравнений
$$\begin{cases}
    1 = A\phi_{1}\br{\xi_{1}} + B\phi_{2}\br{\xi_{1}} \\
    0 = A\phi_{1}\br{\xi_{2}} + B\phi_{2}\br{\xi_{2}}
\end{cases}$$
имеющей решение $A = 1$, $B = - \br{\int\limits_{\xi_{1}}^{\xi_{2}} {\phi_{1}^{-2} \br{t} \ dt}}^{-1}$. Отсюда получаем решение задачи А
$$\phi\br{\xi} = \phi_{1}\br{\xi} \br{1 - \frac {\tintlim{\xi_{1}}{\xi} {\frac {1} {\phi_{1}^{2} \br{t}} \ dt}} {\tintlim{\xi_{1}}{\xi_{2}} {\frac {1} {\phi_{1}^{2} \br{t}} \ dt}}}.$$

Учитывая, что
$$\tintlim{\xi_{1}}{\xi_{2}} {\frac {1} {\phi_{1}^{2} \br{t}} \ dt} - \tintlim{\xi_{1}}{\xi_{2}} {\frac {1} {\phi_{1}^{2} \br{t}} \ dt}
= \tintlim{\xi}{\xi_{1}} {\frac {1} {\phi_{1}^{2} \br{t}} \ dt} + \tintlim{\xi_{1}}{\xi_{2}} {\frac {1} {\phi_{1}^{2} \br{t}} \ dt}
= -\tintlim{\xi_{2}}{\xi} {\frac {1} {\phi_{1}^{2} \br{t}} \ dt},$$
решение $\phi\br{\xi}$ упрощается до
$$\phi\br{\xi} = \frac {-\phi_{1} \br{\xi} \tintlim{\xi_{2}}{\xi} {\frac {1} {\phi^{2} \br{t}} \ dt}} {\tintlim{\xi_{1}}{\xi_{2}} {\frac {1} {\phi_{1}^{2} \br{t}} \ dt}}.$$

На множестве $\xi \in \left[\xi_{2}; +\infty \right)$ обе функции $\phi_{1}\br{\xi}$ и $\tintlim{\xi_{2}}{\xi} {\frac {1} {\phi_{1}^{2} \br{t}} \ dt}$ в числителе положительны и монотонно возрастают, в знаменателе находится положительное число, а потому $\phi\br{\xi}$ при таких $\xi$ убывает.

Пусть теперь $\xi \in \br{-\infty; \xi_{2}}$. Имеем
$$\phi'\br{\xi} = -\tintlim{\xi_{2}}{\xi} { q\br{t} \phi\br{t} \ dt} + \phi'\br{\xi_{2}}
= \tintlim{\xi}{\xi_{2}} { q\br{t} \phi\br{t} \ dt} + \phi'\br{\xi_{2}}.$$

Функция $\phi'\br{x}$ непрерывна. В частности, 
$$\phi'\br{\xi_{2}} = \lim_{\xi \to \xi_{2} + 0} \phi'\br{\xi} \le 0.$$

Помимо $\xi_{2}$, нулей у функции $\phi\br{\xi}$ нет, потому что иначе можно было бы составить из другого нуля и $\xi_{2}$ особый случай краевой задачи, единственное возможное решение которой -- тождественный нуль. Это значит, что на $\left[\xi; \xi_{2}\right)$ функция $\phi\br{\xi}$ положительна, а значит
$$\phi'\br{\xi} = \tintlim{\xi}{\xi_{2}} { q\br{t} \phi\br{t} dt} + \phi\br{\xi_{2}} \le 0.$$

и на луче $\xi \in \br{-\infty; \xi_{2}}$ функция $\phi'\br{\xi}$ также невозрастает.

Утверждение доказано.

\begin{remark} 
Условие $q\br{x} \le 0$ нельзя опустить.
\end{remark}

Например, краевая задача
$$\begin{cases}
    y''\br{x} + y\br{x} = 0 \\
    y\br{\dfrac {\pi} {4}} = -1 \vspace{0.1cm} \\
    y\br{\dfrac {5\pi} {4}} = 1
\end{cases}$$

равносильна системе уравнений
$$\begin{cases}
    -1 = A \cdot \dfrac {1} {\sqrt{2}} + B \cdot \dfrac {1} {\sqrt{2}} \vspace{0.1cm} \\
    1 = A \cdot \dfrac {1} {\sqrt{2}} + B \cdot \dfrac {1} {\sqrt{2}}
\end{cases}$$
и потому не имеет решений. 

А краевая задача 
$$\begin{cases}
    y''\br{x} + y\br{x} = 0 \\
    y\br{\dfrac {\pi} {4}} = 1 \vspace{0.1cm} \\
    y\br{\dfrac {5\pi} {4}} = 1
\end{cases}$$

приводит к 
$$\begin{cases}
    1 = A \cdot \dfrac {1} {\sqrt{2}} + B \cdot \dfrac {1} {\sqrt{2}} \vspace{0.1cm} \\
    1 = A \cdot \dfrac {1} {\sqrt{2}} + B \cdot \dfrac {1} {\sqrt{2}}
\end{cases}$$
и имеет бесконечно много решений.

\end{document}
