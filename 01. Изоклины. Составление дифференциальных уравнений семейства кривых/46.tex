\documentclass[a5paper,10pt]{article}

% https://github.com/InDevRus/filippov-solutions
% Page setup
\pagenumbering{gobble}

% Margin setup
\usepackage{geometry}
\geometry{left=1cm}
\geometry{right=1cm}
\geometry{top=1cm}
\geometry{bottom=1.5cm}

% For inserting tables
\usepackage{array}

% Formula aligning
\usepackage{amsmath}

% For formula diacritics
\usepackage{amsfonts}

% Theorem definitions
\usepackage{amsthm}
\theoremstyle{remark}
\newtheorem*{necessity}{Необходимость}
\newtheorem*{sufficiency}{Достаточность}
\theoremstyle{definition}
\newtheorem*{theorem}{Теорема}
\newtheorem*{lemma}{Лемма}
\newtheorem*{corollary}{Следствие}
\newtheorem*{criteria}{Критерий}
\newtheorem*{algorithm}{Алгоритм}
\newtheorem*{formula}{Формула}
\newtheorem*{remark}{Замечание}
\newtheorem*{proposition}{Предложение}

% For bigger integral signs
\usepackage{bigints}

% For semantic advancements
\usepackage{enotez}

% For pictures
\usepackage{graphicx}

%For framing
\usepackage{framed}

% Automatic paragraph indentations
\usepackage{indentfirst}
\setlength{\parindent}{1em}

% Formula spacing configuration
\delimitershortfall-1sp
\usepackage{mleftright}
\mleftright

% For resizing
\usepackage{relsize}

% More convenient text-style common notations
\DeclareMathOperator*\lowlim{\underline{\lim}}
\DeclareMathOperator*\uplim{\overline{\lim}}

\newcommand{\tlowlim}[1]{\lowlim\limits_{#1}}
\newcommand{\tuplim}[1]{\uplim\limits_{#1}}

\newcommand{\tpow}[2]{{#1}^{\mathlarger{#2}}}

\newcommand{\tint}{\displaystyle{\int}}
\newcommand{\tintlim}[2]{\displaystyle{\int\limits_{#1}^{#2}}}
\newcommand{\tbigint}{\displaystyle{\mathlarger{\int}}}
\newcommand{\tbigintlim}[2]{\displaystyle{\mathlarger{\int}\limits_{#1}^{#2}}}

\newcommand{\tsumlim}[2]{\displaystyle{\mathlarger{\sum}\displaylimits_{#1}^{#2}}}
\newcommand{\tprodlim}[2]{\displaystyle{\mathlarger{\prod}\displaylimits_{#1}^{#2}}}

\newcommand{\tmin}[1]{\min\limits_{#1}}
\newcommand{\tmax}[1]{\max\limits_{#1}}
\newcommand{\tlim}[1]{\lim\limits_{#1}}

\newcommand{\norm}[1]{\left\lVert#1\right\rVert}

% For floaty text
\usepackage{wrapfig}
\usepackage{floatflt}

% For cyrillic characters support
\usepackage[english, russian]{babel}

% For proper font
\usepackage[no-math]{fontspec}

% TNR within text
\setmainfont{Times New Roman}

% TNR within formulas
\usepackage{newtxmath}
\DeclareSymbolFont{operators}{OT1}{ntxtlf}{m}{n}
\SetSymbolFont{operators}{bold}{OT1}{ntxtlf}{b}{n}

% For degree sign
\usepackage{siunitx}

% Automatic brackets placement
\newcommand{\br}[1]{\left(#1\right)}
\newcommand{\vbr}[1]{\left|#1\right|}
\newcommand{\cbr}[1]{\left\{#1\right\}}
\newcommand{\rbr}[1]{\left[#1\right]}
\renewcommand{\le}{\leqslant}
\renewcommand{\ge}{\geqslant} 

% Automatic replacement for two greek letters
\renewcommand{\epsilon}{\varepsilon}
\renewcommand{\phi}{\varphi}

% Redefinition of some operators (to the appropriation of Russian notation)
\DeclareMathOperator{\arcsh}{arcsh}
\DeclareMathOperator{\arcch}{arcch}
\DeclareMathOperator{\arcth}{arcth}
\DeclareMathOperator{\arccth}{arccth}
\DeclareMathOperator{\rank}{rank}
\DeclareMathOperator{\inv}{inv}
\DeclareMathOperator{\sgn}{sgn}
\renewcommand{\Re}{\operatorname{Re}}
\renewcommand{\Im}{\operatorname{Im}}

\renewcommand{\gcd}{\text{НОД}}
\newcommand{\lcm}{\text{НОК}}

% Restrict inlne formula breaking
\binoppenalty=10000 
\relpenalty=10000

% Greek letters setup
% Old greek letters setup
\DeclareSymbolFont{old_letters}{OML}{ztmcm}{m}{it}
\SetSymbolFont{old_letters}{bold}{OML}{ztmcm}{b}{it}

\newcommand{\Alpha}{\text{A}}
\newcommand{\Beta}{\text{B}}
\newcommand{\Epsilon}{\text{E}}
\newcommand{\Zeta}{\text{Z}}
\newcommand{\Eta}{\text{H}}
\newcommand{\Iota}{\text{I}}
\newcommand{\Kappa}{\text{K}}
\newcommand{\Mu}{\text{M}}
\newcommand{\Nu}{\text{N}}
\newcommand{\Omicron}{\text{O}}
\newcommand{\Rho}{\text{P}}
\newcommand{\Tau}{\text{T}}
\newcommand{\Chi}{\text{X}}

\DeclareMathSymbol{\alpha}{\mathord}{old_letters}{11}
\DeclareMathSymbol{\beta}{\mathord}{old_letters}{12}
\DeclareMathSymbol{\gamma}{\mathord}{old_letters}{13}
\DeclareMathSymbol{\delta}{\mathord}{old_letters}{14}
\DeclareMathSymbol{\varepsilon}{\mathord}{old_letters}{15}
\DeclareMathSymbol{\zeta}{\mathord}{old_letters}{16}
\DeclareMathSymbol{\eta}{\mathord}{old_letters}{17}
\DeclareMathSymbol{\theta}{\mathord}{old_letters}{18}
\DeclareMathSymbol{\iota}{\mathord}{old_letters}{19}
\DeclareMathSymbol{\kappa}{\mathord}{old_letters}{20}
\DeclareMathSymbol{\lambda}{\mathord}{old_letters}{21}
\DeclareMathSymbol{\mu}{\mathord}{old_letters}{22}
\DeclareMathSymbol{\nu}{\mathord}{old_letters}{23}
\DeclareMathSymbol{\xi}{\mathord}{old_letters}{24}
\newcommand{\omicron}{\text{\textit{\larger[1]{o}}}}
\DeclareMathSymbol{\pi}{\mathord}{old_letters}{25}
\DeclareMathSymbol{\rho}{\mathord}{old_letters}{26}
\DeclareMathSymbol{\sigma}{\mathord}{old_letters}{27}
\DeclareMathSymbol{\tau}{\mathord}{old_letters}{28}
\DeclareMathSymbol{\upsilon}{\mathord}{old_letters}{29}
\DeclareMathSymbol{\varphi}{\mathord}{old_letters}{39}
\DeclareMathSymbol{\chi}{\mathord}{old_letters}{31}
\DeclareMathSymbol{\psi}{\mathord}{old_letters}{32}
\DeclareMathSymbol{\omega}{\mathord}{old_letters}{33}






\begin{document}

\begin{framed}
	Если в полярных координатах $\tsys{& x = r \cos\theta \\& y = r \sin\theta}$ имеем \vspace{1mm} производную $\dfrac {dr} {d\theta} = A\br{r; \theta}$,
	то соответствующая производная в декартовых координатах $\dfrac {dy} {dx} = B\br{r; \theta}$ связана с исходной $A\br{r; \theta}$ соотношениями
	\begin{align*}
		\dfrac {dy} {dx} 
		= B\br{A; r; \theta}
		= \dfrac {A \sin\theta + r \cos\theta} {A \cos\theta - r \sin\theta};&&
		\dfrac {dr} {d\theta} 
		= A\br{B; r; \theta} 
		= r \cdot \dfrac {B \sin\theta + \cos\theta} {B\cos\theta - \sin\theta}.
	\end{align*}
\end{framed}

\begin{framed}
	Для угла $\phi = \ang{45}$ имеем $\tg{\beta} = \tg\br{\alpha \pm \phi} =\tg\br{\alpha \pm \ang{45}} = \dfrac {\tg\alpha \pm 1} {1 \mp \tg\alpha }$.
	
	С учётом указанного выше для данного угла $\phi$ всякое семейство изогональных траекторий $\zed\br{x}$ и соответствующее семейство в полярной системе $\tsys{& x = \rho \cos\theta \\& \zed = \rho \sin\theta}$ удовлетворяют соотношениям
	\begin{align*}
		A_{\zed}\br{\rho; \theta}
		= \rho \cdot \dfrac {B_{\zed}\br{\rho; \theta} \sin\theta + \cos\theta} {B_{\zed}\br{\rho; \theta} \cos\theta - \sin\theta};
		&&
		B_{\zed}\br{\rho; \theta} = \dfrac {B_{y}\br{\rho; \theta} \pm 1} {1 \mp B_{y}\br{\rho; \theta}};
	\end{align*}
	$$B_{y}\br{r; \theta} = \dfrac {A_{y}\br{r; \theta} \sin\theta + r \cos\theta} {A_{y}\br{r; \theta} \cos\theta - r \sin\theta},$$
	где введены обозначения
	$A_{y} = \dfrac {dr} {d\theta}$,
	$B_{y} = \dfrac {dy} {dx}$,
	$A_{\zed} = \dfrac {d\rho} {d\theta}$,
	$B_{\zed} = \dfrac {d\zed} {dx}$. \vspace{1mm}
	
	Последовательно подставив и затем упростив, получим следующие формулы двух семейств изогональных траекторий:
	\begin{align*}
		A_{\zed}\br{\rho; \theta} = \rho \cdot \dfrac {A_{y}\br{\rho; \theta} - \rho} {A_{y}\br{\rho; \theta} + \rho} &&
		A_{\zed}\br{\rho; \theta} = -\rho \cdot \dfrac {A_{y}\br{\rho; \theta} + \rho} {A_{y}\br{\rho; \theta} - \rho}
	\end{align*}
\end{framed}

$r = a \sin\theta$.
$a = \dfrac {r} {\sin\theta}$.
$0 = \dfrac {r' \sin\theta - r\cos\theta} {\sin^2 \theta}$.
$r'\br{\theta} = r\ctg\theta = A_{y}$.

Таким образом, два уравнения изогональных траекторий в обозначениях, указанных выше, принимают вид
$\dfrac {d\rho} {d\theta} 
= A_{\zed} 
= \rho \cdot \dfrac {A_{y} - \rho} {A_{y} + \rho}
= \rho \cdot \dfrac {\rho \ctg\theta - \rho} {\rho \ctg\theta + \rho}
= \rho \ctg\br{\theta + \ang{45}}$, и по аналогии
$\dfrac {d\rho} {d\theta} = A_{\zed}
= -\rho \cdot \dfrac {\ctg\theta + 1} {\ctg\theta + 1} 
= \rho \ctg\br{\theta - \ang{45}}$. \vspace{1mm}

$\rho'\br{\theta} = \rho \ctg\br{\theta \pm \ang{45}}$;

\begin{remark}
	Уравнения
	$\rho = \rho \ctg\br{\theta \pm \ang{45}}$
	имеют общие решения $\rho = C \sin\br{\theta \pm \ang{45}}$.
\end{remark}

\end{document}

