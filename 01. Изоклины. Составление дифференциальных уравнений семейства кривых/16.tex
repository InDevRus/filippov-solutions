\documentclass[a5paper,10pt]{article}

% https://github.com/InDevRus/filippov-solutions
% Page setup
\pagenumbering{gobble}

% Margin setup
\usepackage{geometry}
\geometry{left=1cm}
\geometry{right=1cm}
\geometry{top=1cm}
\geometry{bottom=1.5cm}

% For inserting tables
\usepackage{array}

% Formula aligning
\usepackage{amsmath}

% For formula diacritics
\usepackage{amsfonts}

% Theorem definitions
\usepackage{amsthm}
\theoremstyle{remark}
\newtheorem*{necessity}{Необходимость}
\newtheorem*{sufficiency}{Достаточность}
\theoremstyle{definition}
\newtheorem*{theorem}{Теорема}
\newtheorem*{lemma}{Лемма}
\newtheorem*{corollary}{Следствие}
\newtheorem*{criteria}{Критерий}
\newtheorem*{algorithm}{Алгоритм}
\newtheorem*{formula}{Формула}
\newtheorem*{remark}{Замечание}
\newtheorem*{proposition}{Предложение}

% For bigger integral signs
\usepackage{bigints}

% For semantic advancements
\usepackage{enotez}

% For pictures
\usepackage{graphicx}

%For framing
\usepackage{framed}

% Automatic paragraph indentations
\usepackage{indentfirst}
\setlength{\parindent}{1em}

% Formula spacing configuration
\delimitershortfall-1sp
\usepackage{mleftright}
\mleftright

% For resizing
\usepackage{relsize}

% More convenient text-style common notations
\DeclareMathOperator*\lowlim{\underline{\lim}}
\DeclareMathOperator*\uplim{\overline{\lim}}

\newcommand{\tlowlim}[1]{\lowlim\limits_{#1}}
\newcommand{\tuplim}[1]{\uplim\limits_{#1}}

\newcommand{\tpow}[2]{{#1}^{\mathlarger{#2}}}

\newcommand{\tint}{\displaystyle{\int}}
\newcommand{\tintlim}[2]{\displaystyle{\int\limits_{#1}^{#2}}}
\newcommand{\tbigint}{\displaystyle{\mathlarger{\int}}}
\newcommand{\tbigintlim}[2]{\displaystyle{\mathlarger{\int}\limits_{#1}^{#2}}}

\newcommand{\tsumlim}[2]{\displaystyle{\mathlarger{\sum}\displaylimits_{#1}^{#2}}}
\newcommand{\tprodlim}[2]{\displaystyle{\mathlarger{\prod}\displaylimits_{#1}^{#2}}}

\newcommand{\tmin}[1]{\min\limits_{#1}}
\newcommand{\tmax}[1]{\max\limits_{#1}}
\newcommand{\tlim}[1]{\lim\limits_{#1}}

\newcommand{\norm}[1]{\left\lVert#1\right\rVert}

% For floaty text
\usepackage{wrapfig}
\usepackage{floatflt}

% For cyrillic characters support
\usepackage[english, russian]{babel}

% For proper font
\usepackage[no-math]{fontspec}

% TNR within text
\setmainfont{Times New Roman}

% TNR within formulas
\usepackage{newtxmath}
\DeclareSymbolFont{operators}{OT1}{ntxtlf}{m}{n}
\SetSymbolFont{operators}{bold}{OT1}{ntxtlf}{b}{n}

% For degree sign
\usepackage{siunitx}

% Automatic brackets placement
\newcommand{\br}[1]{\left(#1\right)}
\newcommand{\vbr}[1]{\left|#1\right|}
\newcommand{\cbr}[1]{\left\{#1\right\}}
\newcommand{\rbr}[1]{\left[#1\right]}
\renewcommand{\le}{\leqslant}
\renewcommand{\ge}{\geqslant} 

% Automatic replacement for two greek letters
\renewcommand{\epsilon}{\varepsilon}
\renewcommand{\phi}{\varphi}

% Redefinition of some operators (to the appropriation of Russian notation)
\DeclareMathOperator{\arcsh}{arcsh}
\DeclareMathOperator{\arcch}{arcch}
\DeclareMathOperator{\arcth}{arcth}
\DeclareMathOperator{\arccth}{arccth}
\DeclareMathOperator{\rank}{rank}
\DeclareMathOperator{\inv}{inv}
\DeclareMathOperator{\sgn}{sgn}
\renewcommand{\Re}{\operatorname{Re}}
\renewcommand{\Im}{\operatorname{Im}}

\renewcommand{\gcd}{\text{НОД}}
\newcommand{\lcm}{\text{НОК}}

% Restrict inlne formula breaking
\binoppenalty=10000 
\relpenalty=10000

% Greek letters setup
% Old greek letters setup
\DeclareSymbolFont{old_letters}{OML}{ztmcm}{m}{it}
\SetSymbolFont{old_letters}{bold}{OML}{ztmcm}{b}{it}

\newcommand{\Alpha}{\text{A}}
\newcommand{\Beta}{\text{B}}
\newcommand{\Epsilon}{\text{E}}
\newcommand{\Zeta}{\text{Z}}
\newcommand{\Eta}{\text{H}}
\newcommand{\Iota}{\text{I}}
\newcommand{\Kappa}{\text{K}}
\newcommand{\Mu}{\text{M}}
\newcommand{\Nu}{\text{N}}
\newcommand{\Omicron}{\text{O}}
\newcommand{\Rho}{\text{P}}
\newcommand{\Tau}{\text{T}}
\newcommand{\Chi}{\text{X}}

\DeclareMathSymbol{\alpha}{\mathord}{old_letters}{11}
\DeclareMathSymbol{\beta}{\mathord}{old_letters}{12}
\DeclareMathSymbol{\gamma}{\mathord}{old_letters}{13}
\DeclareMathSymbol{\delta}{\mathord}{old_letters}{14}
\DeclareMathSymbol{\varepsilon}{\mathord}{old_letters}{15}
\DeclareMathSymbol{\zeta}{\mathord}{old_letters}{16}
\DeclareMathSymbol{\eta}{\mathord}{old_letters}{17}
\DeclareMathSymbol{\theta}{\mathord}{old_letters}{18}
\DeclareMathSymbol{\iota}{\mathord}{old_letters}{19}
\DeclareMathSymbol{\kappa}{\mathord}{old_letters}{20}
\DeclareMathSymbol{\lambda}{\mathord}{old_letters}{21}
\DeclareMathSymbol{\mu}{\mathord}{old_letters}{22}
\DeclareMathSymbol{\nu}{\mathord}{old_letters}{23}
\DeclareMathSymbol{\xi}{\mathord}{old_letters}{24}
\newcommand{\omicron}{\text{\textit{\larger[1]{o}}}}
\DeclareMathSymbol{\pi}{\mathord}{old_letters}{25}
\DeclareMathSymbol{\rho}{\mathord}{old_letters}{26}
\DeclareMathSymbol{\sigma}{\mathord}{old_letters}{27}
\DeclareMathSymbol{\tau}{\mathord}{old_letters}{28}
\DeclareMathSymbol{\upsilon}{\mathord}{old_letters}{29}
\DeclareMathSymbol{\varphi}{\mathord}{old_letters}{39}
\DeclareMathSymbol{\chi}{\mathord}{old_letters}{31}
\DeclareMathSymbol{\psi}{\mathord}{old_letters}{32}
\DeclareMathSymbol{\omega}{\mathord}{old_letters}{33}






\begin{document}

\begin{framed}
\begin{property}[точки перегиба]
Если в некоторой окрестности точки перегиба $x_{0}$ функция дважды дифференцируема, то $f''\br{x_{0}} = 0$.
\end{property}

\begin{test}[точки перегиба]
Точка $x_{0}$ дважды дифференцируемой в окрестности $x_{0}$ функции $f\br{x}$ является точкой перегиба, если $f''\br{x_{0}} = 0$, а при переходе через $x_{0} $ функция $f''\br{x}$ меняет знак. В частности, если в окрестности точки $x_{0}$ функция $f\br{x}$ трижды дифференцируема, то достаточно $f'''\br{x_{0}} \ne 0$.
\end{test}
\end{framed}

Пользоваться будем тем фактом, что 
$y''\br{x} = \dfrac {d} {dx} \br {\dfrac {d} {dx} y\br{x} } 
= \dfrac {d} {dx} \br{f\br{x; y\br{x}}}$.

\begin{enumerate}[label={\asbuk*})]
\item $y' = y - x^2$.
$y''\br{x} = \dfrac {d} {dx} \br{y - x^2} = y' - 2x = y - x^{2} - 2x$. \\
$y'''\br{x} = \dfrac {d} {dx} \br{y - x^2 - 2x} = y' - 2x - 2 = y - x^2 - 2x - 2$.

Пусть $y''\br{x} = y - x^2 - 2x = 0$. Тогда $y = x^2 + 2x$. При этом $y'''\br{x} = y - x^2 - 2x - 2 = \linebreak = -2 \ne 0$, а потому все точки этой кривой являются точками перегиба. Итак, точки перегиба данного уравнения лежат на и только на параболе $y = x^2 + 2x$.

\item $y' = x - e^y$.
$y''\br{x} = 1 - y' e^y 
= 1 - \br{x - e^y} e^y
= 1 - xe^y + e^{2y}$. \\
$y'''\br{x} = \dfrac {d} {dx} \br{1 - xe^y + e^{2y}} 
= -e^y - x y' e^y + 2 y' e^{2y} 
= -e^y - x \br{x - e^y} e^y + 2 \br{x - e^y} e^{2y} \vspace{1mm} \linebreak 
= -e^y - x^2 e^y + 3 x e^{2y} - 2 e^{3y}$.

Пусть $y''\br{x} = 1 - xe^y + e^{2y} = 0$. Тогда $xe^{y} = 1 + e^{2y}$, $x = e^{-y} + e^{y}$, и
$y'''\br{x} 
= -e^y - \linebreak - \br{e^{-y} + e^{y}}^2 e^y + 3 \br{e^{-y} + e^{y}} e^{2y} - 2 e^{3y} 
= -e^{y} < 0$.

Таким образом, всюду, где $y''\br{x} = 0$ имеем $y'''\br{x} \ne 0$, то есть точки перегиба лежат на кривой $x = e^{-y} + e^{y}$ (или $ x= 2\ch\br{y}$) и только на ней.

\item $x^2 + y^2 y' = 1$.
$y' = y^{-2} \br{1 - x^2} = y^{-2} - x^2 y^{-2}$. \\
$y''\br{x} = -2 y^{-3} y' + 2 x^2 y^{-3} y' - 2x y^{-2} 
= -2y^{-3} y' \br{1 - x^2} - 2x y^{-2}
= -2y^{-5} \br{1 - x^2}^2 - 2x y^{-2}$. \\
\begin{multline*}
    y'''\br{x} 
    = 10 y^{-6} \br{1 - x^2}^2 y' + 8 x y^{-5} \br{1 - x^2} - 2y^{-2} + 4xy^{-3} y' = \\ 
    = 10 y^{-8} \br{1 - x^2}^3 + 8 x y^{-5} \br{1 - x^2} - 2y^{-2} + 4xy^{-5} \br{1 - x^2} = \\
    = 10 y^{-8} \br{1 - x^2}^3 + 12 xy^{-5} \br{1 - x^2} - 2y^{-2}.
\end{multline*}
Пусть $y''\br{x} = -2y^{-5} \br{1 - x^2}^2 - 2x y^{-2} = 0$, то есть $\br{1 - x^2}^2 = -xy^3$.
\begin{multline*}
    y'''\br{x} 
    = 10 y^{-8} \br{1 - x^2} \cdot \br{-xy^3} + 12 xy^{-5} \br{1 - x^2} - 2y^{-2}
    = 2 xy^{-5} \br{1 - x^2} - 2y^{-2} = \\
    = -2 y^{-8} \br{1 - x^2}^2 - 2y^{-2} 
    = -2y^{-2} \br{y^{-6} \br{1 - x^2}^2 + 1} < 0.
\end{multline*}

Итак, всякая точка кривой $xy^3 = -\br{1 - x^2}^2$ является точкой перегиба.

На прямой $y = 0$ первая производная не определена, а потому исследовать функцию $y\br{x}$ на наличие перегиба на этой прямой с помощью производной не получится.

\item $y'\br{x} = f\br{x; y}$.
Обозначим $\xi = x$. Тогда по правилу произодной сложной функции
$y''\br{x} 
= \dfrac {d} {dx} f\br{x; y\br{x}}
= \dfrac {d} {d\xi} f\br{x\br{\xi}; y\br{\xi}}
= \parder{f}{x} \dfrac {dx}{d\xi} + \parder{f}{y} \dfrac {dy} {d\xi}
= \parder{f}{x} + \parder{f}{y} f\br{x; y}$,
ибо $\dfrac {dx}{d\xi} = 1$ и $\dfrac {dy}{d\xi} = \dfrac{dy}{dx} = f\br{x; y}$.

Итак, уравнение кривой точек перегиба имеет вид
$f'_{x} + f \cdot f'_{y} = 0$. Это условие необходимое, но не достаточное, так что проверку на наличие перегиба при фиксированных $x$ и $y$ необходимо делать исследованием знака $y''\br{x}$.

Предположим теперь, что функция $f\br{x; y}$ имеет непрерывные вторые частные \linebreak производные $\pardersecond{f}{x}$, $\pardersecond{f}{y}$ и $\pardersecond[f]{x}{y}$ и найдём в этом предположении $y'''\br{x}$.

$$y'''\br{x} = \dfrac {d} {dx} \br{y''\br{x}} = \dfrac {d} {d\xi} \br{\parder{f}{x} + f \parder{f}{y}}.$$
$\dfrac {d} {d\xi} \parder{f}{x}
= \dfrac {\partial} {\partial x} \br{\parder{f}{x}} \cdot \dfrac {dx} {d\xi} + \dfrac {\partial} {\partial y} \br{\parder{f}{y}} \cdot \dfrac {dy} {d\xi}
= \pardersecond{f}{x} + \pardersecond[f]{x}{y} \cdot \dfrac {dy} {dx}
= \pardersecond{f}{x} + \pardersecond[f]{x}{y} \cdot f$; \\
\begin{multline*}\dfrac {d} {d\xi} \br{f \cdot \parder{f}{y}} 
= \dfrac {d} {d\xi} f \cdot \parder{f}{y} + f \cdot \dfrac {d} {d\xi} \parder{f}{y} \\
= \br{\parder{f}{x} + \parder{f}{y} \cdot f} \parder{f}{y} + f \cdot \br{\dfrac {\partial}{\partial x}\br{\parder{f}{y}} \cdot \dfrac{dx}{d\xi} + \dfrac {\partial}{\partial y}\br{\parder{f}{y}} \cdot \dfrac{dy}{d\xi}} = \\
= \parder{f}{x}\parder{f}{y} + \br{\parder{f}{y}}^2 \cdot f + \pardersecond[f]{x}{y} \cdot f + \pardersecond{f}{y} \cdot f^2.
\end{multline*}

Итого, $y'''\br{x} = \dfrac {d} {d\xi} \br{\parder{f}{x}} + \dfrac {d} {d\xi} \br{f \parder{f}{y}} = f''_{xx} + f''_{xy} f + f'_{x}f'_{y} + \br{f'_{y}}^2 f + f''_{xy} f + f''_{yy} f^2 \linebreak 
= f''_{xx} + 2f''_{xy} f + f''_{yy} f^2 + f'_{y} \cdot \br{f'_{x} + f \cdot f'_{y}}$.

Пусть теперь $f'_{x} + f \cdot f'_{y} = 0$. При таком условии 
$y'''\br{x} = f''_{xx} + 2f''_{xy} f + f''_{yy} f^2$.

Окончательно получаем достаточные условия наличия перегиба в точке:
$$\tsys{& f'_{x} + f \cdot f'_{y} = 0 \\& f''_{xx} + 2f''_{xy} f + f''_{yy} f^2 \ne 0}.$$

\end{enumerate}
\end{document}
