\documentclass[a5paper,10pt]{article}

% https://github.com/InDevRus/filippov-solutions

% Page setup
\pagenumbering{gobble}

% Margin setup
\usepackage{geometry}
\geometry{left=1cm}
\geometry{right=1cm}
\geometry{top=1cm}
\geometry{bottom=1.5cm}

% For inserting tables
\usepackage{array}

% Formula aligning
\usepackage{amsmath}

% For formula diacritics
\usepackage{amsfonts}

% Theorem definitions
\usepackage{amsthm}
\theoremstyle{remark}
\newtheorem*{necessity}{Необходимость}
\newtheorem*{sufficiency}{Достаточность}
\theoremstyle{definition}
\newtheorem*{theorem}{Теорема}
\newtheorem*{lemma}{Лемма}
\newtheorem*{corollary}{Следствие}
\newtheorem*{criteria}{Критерий}
\newtheorem*{algorithm}{Алгоритм}
\newtheorem*{formula}{Формула}
\newtheorem*{remark}{Замечание}
\newtheorem*{proposition}{Предложение}

% For bigger integral signs
\usepackage{bigints}

% For semantic advancements
\usepackage{enotez}

% For pictures
\usepackage{graphicx}

%For framing
\usepackage{framed}

% Automatic paragraph indentations
\usepackage{indentfirst}
\setlength{\parindent}{1em}

% Formula spacing configuration
\delimitershortfall-1sp
\usepackage{mleftright}
\mleftright

% For resizing
\usepackage{relsize}

% More convenient text-style common notations
\DeclareMathOperator*\lowlim{\underline{\lim}}
\DeclareMathOperator*\uplim{\overline{\lim}}

\newcommand{\tlowlim}[1]{\lowlim\limits_{#1}}
\newcommand{\tuplim}[1]{\uplim\limits_{#1}}

\newcommand{\tpow}[2]{{#1}^{\mathlarger{#2}}}

\newcommand{\tint}{\displaystyle{\int}}
\newcommand{\tintlim}[2]{\displaystyle{\int\limits_{#1}^{#2}}}
\newcommand{\tbigint}{\displaystyle{\mathlarger{\int}}}
\newcommand{\tbigintlim}[2]{\displaystyle{\mathlarger{\int}\limits_{#1}^{#2}}}

\newcommand{\tsumlim}[2]{\displaystyle{\mathlarger{\sum}\displaylimits_{#1}^{#2}}}
\newcommand{\tprodlim}[2]{\displaystyle{\mathlarger{\prod}\displaylimits_{#1}^{#2}}}

\newcommand{\tmin}[1]{\min\limits_{#1}}
\newcommand{\tmax}[1]{\max\limits_{#1}}
\newcommand{\tlim}[1]{\lim\limits_{#1}}

\newcommand{\norm}[1]{\left\lVert#1\right\rVert}

% For floaty text
\usepackage{wrapfig}
\usepackage{floatflt}

% For cyrillic characters support
\usepackage[english, russian]{babel}

% For proper font
\usepackage[no-math]{fontspec}

% TNR within text
\setmainfont{Times New Roman}

% TNR within formulas
\usepackage{newtxmath}
\DeclareSymbolFont{operators}{OT1}{ntxtlf}{m}{n}
\SetSymbolFont{operators}{bold}{OT1}{ntxtlf}{b}{n}

% For degree sign
\usepackage{siunitx}

% Automatic brackets placement
\newcommand{\br}[1]{\left(#1\right)}
\newcommand{\vbr}[1]{\left|#1\right|}
\newcommand{\cbr}[1]{\left\{#1\right\}}
\newcommand{\rbr}[1]{\left[#1\right]}
\renewcommand{\le}{\leqslant}
\renewcommand{\ge}{\geqslant} 

% Automatic replacement for two greek letters
\renewcommand{\epsilon}{\varepsilon}
\renewcommand{\phi}{\varphi}

% Redefinition of some operators (to the appropriation of Russian notation)
\DeclareMathOperator{\arcsh}{arcsh}
\DeclareMathOperator{\arcch}{arcch}
\DeclareMathOperator{\arcth}{arcth}
\DeclareMathOperator{\arccth}{arccth}
\DeclareMathOperator{\rank}{rank}
\DeclareMathOperator{\inv}{inv}
\DeclareMathOperator{\sgn}{sgn}
\renewcommand{\Re}{\operatorname{Re}}
\renewcommand{\Im}{\operatorname{Im}}

\renewcommand{\gcd}{\text{НОД}}
\newcommand{\lcm}{\text{НОК}}

% Restrict inlne formula breaking
\binoppenalty=10000 
\relpenalty=10000

% Greek letters setup
% Old greek letters setup
\DeclareSymbolFont{old_letters}{OML}{ztmcm}{m}{it}
\SetSymbolFont{old_letters}{bold}{OML}{ztmcm}{b}{it}

\newcommand{\Alpha}{\text{A}}
\newcommand{\Beta}{\text{B}}
\newcommand{\Epsilon}{\text{E}}
\newcommand{\Zeta}{\text{Z}}
\newcommand{\Eta}{\text{H}}
\newcommand{\Iota}{\text{I}}
\newcommand{\Kappa}{\text{K}}
\newcommand{\Mu}{\text{M}}
\newcommand{\Nu}{\text{N}}
\newcommand{\Omicron}{\text{O}}
\newcommand{\Rho}{\text{P}}
\newcommand{\Tau}{\text{T}}
\newcommand{\Chi}{\text{X}}

\DeclareMathSymbol{\alpha}{\mathord}{old_letters}{11}
\DeclareMathSymbol{\beta}{\mathord}{old_letters}{12}
\DeclareMathSymbol{\gamma}{\mathord}{old_letters}{13}
\DeclareMathSymbol{\delta}{\mathord}{old_letters}{14}
\DeclareMathSymbol{\varepsilon}{\mathord}{old_letters}{15}
\DeclareMathSymbol{\zeta}{\mathord}{old_letters}{16}
\DeclareMathSymbol{\eta}{\mathord}{old_letters}{17}
\DeclareMathSymbol{\theta}{\mathord}{old_letters}{18}
\DeclareMathSymbol{\iota}{\mathord}{old_letters}{19}
\DeclareMathSymbol{\kappa}{\mathord}{old_letters}{20}
\DeclareMathSymbol{\lambda}{\mathord}{old_letters}{21}
\DeclareMathSymbol{\mu}{\mathord}{old_letters}{22}
\DeclareMathSymbol{\nu}{\mathord}{old_letters}{23}
\DeclareMathSymbol{\xi}{\mathord}{old_letters}{24}
\newcommand{\omicron}{\text{\textit{\larger[1]{o}}}}
\DeclareMathSymbol{\pi}{\mathord}{old_letters}{25}
\DeclareMathSymbol{\rho}{\mathord}{old_letters}{26}
\DeclareMathSymbol{\sigma}{\mathord}{old_letters}{27}
\DeclareMathSymbol{\tau}{\mathord}{old_letters}{28}
\DeclareMathSymbol{\upsilon}{\mathord}{old_letters}{29}
\DeclareMathSymbol{\varphi}{\mathord}{old_letters}{39}
\DeclareMathSymbol{\chi}{\mathord}{old_letters}{31}
\DeclareMathSymbol{\psi}{\mathord}{old_letters}{32}
\DeclareMathSymbol{\omega}{\mathord}{old_letters}{33}





\usepackage{pgfplots}
\pgfplotsset{compat = 1.18, lua backend = true}
\pgfplotsset{
    every axis/.append style = {
        xlabel={$x$},
        ylabel={$y$},
        axis lines = middle,
        unit vector ratio = 1 1,
        font=\small
    },
    every axis plot/.append style = {smooth, samples = 100}
}


\begin{document}

$f\br{x; y} = \dfrac {y} {y + x}$. Уравнение изоклины: $\dfrac {y} {y + x} = k$. $y\br{x} = \dfrac {k} {1 - k} x$. Таким образом, изоклинами будут прямые. На левом рисунке изображены изоклины, на правом -- решения уравнений.

\begin{figure}[!h]
\begin{minipage}{.5\textwidth}
    \centering
    \begin{tikzpicture}
        \pgfplotsset{
            Graph/.style = {
                samples = 60,
                restrict y to domain = -5 : 5,
            },
            DirectionGraph/.style = {
                samples = 20,
                quiver = {scale arrows = 0.075},
                restrict y to domain = -5 : 5,
            },
            DirectionGraph1/.style = {
                DirectionGraph,
                quiver = {
                    u = {dot_x(x, y) / sqrt(dot_x(x, y) ^ 2 + dot_y(x, y) ^ 2)},
                    v = {dot_y(x, y) / sqrt(dot_x(x, y) ^ 2 + dot_y(x, y) ^ 2)}
                }
            },
            DirectionGraph2/.style = {
                DirectionGraph,
                quiver = {
                    u = {-dot_x(x, y) / sqrt(dot_x(x, y) ^ 2 + dot_y(x, y) ^ 2)},
                    v = {-dot_y(x, y) / sqrt(dot_x(x, y) ^ 2 + dot_y(x, y) ^ 2)}
                }
            },
        }
        \begin{axis}[
            width = 7.5cm,
            height = 10cm,
            ymin = -4.2,
            ymax = 4.2,
            xmin = -4.2,
            xmax = 4.2, 
            xtick = {-10, ..., 10},
            ytick = {-10, ..., 10},
            minor tick num = 3,
            declare function = {
                f(\k, \x) = x * \k;
                dot_x(\x, \y) = y + x;
                dot_y(\x, \y) = y;
            }]
            
            \foreach \k in {0.1, 0.2, 0.3, 0.5, 0.75, 1, 1.5, 2, 3} {
                \foreach \s in {-1, 1} {
                    \addplot[Graph, gray, domain = -4.5 : 4.5]
                    {f(\s * \k, x)};
                    \addplot[DirectionGraph1, domain = -4 : -0.25]
                    {f(\s * \k, x)};
                    \addplot[DirectionGraph1, domain = 0.25 : 4]
                    {f(\s * \k, x)};
                    \addplot[DirectionGraph2, domain = -4 : -0.25]
                    {f(\s * \k, x)};
                    \addplot[DirectionGraph2, domain = 0.25 : 4]
                    {f(\s * \k, x)};
                }
            }
            
            \foreach \k in {4.5, 15} {
                \foreach \s in {-1, 1} {
                    \addplot[Graph, gray, domain = -1 : 1]
                    {f(\s * \k, x)};
                    \addplot[DirectionGraph1, domain = -1.5 : -0.25]
                    {f(\s * \k, x)};
                    \addplot[DirectionGraph1, domain = 0.25 : 1.5]
                    {f(\s * \k, x)};
                    \addplot[DirectionGraph2, domain = -1.5 : -0.25]
                    {f(\s * \k, x)};
                    \addplot[DirectionGraph2, domain = 0.25 : 1.5]
                    {f(\s * \k, x)};
                }
            }
        \end{axis}
    \end{tikzpicture}
\end{minipage}
\begin{minipage}{.5\textwidth}
    \centering
    \begin{tikzpicture}
        \pgfplotsset{
            Graph/.style = {
                samples = 60,
                restrict y to domain = -5 : 5,
                opacity = 0.5,
            },
            PreciseGraph/.style = {
                Graph,
                thick,
                samples = 400,
                restrict y to domain = -5 : 5,
                opacity = 1,
            },
            DirectionGraph/.style = {
                opacity = 0.5,
                samples = 20,
                quiver = {scale arrows = 0.075},
                restrict y to domain = -5 : 5,
            },
            DirectionGraph1/.style = {
                DirectionGraph,
                quiver = {
                    u = {dot_x(x, y) / sqrt(dot_x(x, y) ^ 2 + dot_y(x, y) ^ 2)},
                    v = {dot_y(x, y) / sqrt(dot_x(x, y) ^ 2 + dot_y(x, y) ^ 2)}
                }
            },
            DirectionGraph2/.style = {
                DirectionGraph,
                quiver = {
                    u = {-dot_x(x, y) / sqrt(dot_x(x, y) ^ 2 + dot_y(x, y) ^ 2)},
                    v = {-dot_y(x, y) / sqrt(dot_x(x, y) ^ 2 + dot_y(x, y) ^ 2)}
                }
            },
        }
        \begin{axis}[
            trig format = rad,
            width = 7.5cm,
            height = 10cm,
            ymin = -4.2,
            ymax = 4.2,
            xmin = -4.2,
            xmax = 4.2, 
            xtick = {-10, ..., 10},
            ytick = {-10, ..., 10},
            minor tick num = 3,
            declare function = {
                f(\k, \x) = x * \k;
                dot_x(\x, \y) = y + x;
                dot_y(\x, \y) = y;
                h(\D, \x) = \D * exp(cot(x)) / sin(x);
            }]
            
            \addplot[PreciseGraph, cyan, domain = -20 : 20]
            ({h(0.5, x) * cos(x)}, {h(0.5, x) * sin(x)});
            \addplot[PreciseGraph, cyan, domain = -20 : 20]
            ({h(-0.5, x) * cos(x)}, {h(-0.5, x) * sin(x)});
            \addplot[PreciseGraph, violet, domain = -20 : 20]
            ({h(1, x) * cos(x)}, {h(1, x) * sin(x)});
            \addplot[PreciseGraph, violet, domain = -20 : 20]
            ({h(-1, x) * cos(x)}, {h(-1, x) * sin(x)});
            \addplot[PreciseGraph, blue, domain = -20 : 20]
            ({h(1.5, x) * cos(x)}, {h(1.5, x) * sin(x)});
            \addplot[PreciseGraph, blue, domain = -20 : 20]
            ({h(-1.5, x) * cos(x)}, {h(-1.5, x) * sin(x)});
            \addplot[PreciseGraph, olive, domain = -20 : 20]
            ({h(2, x) * cos(x)}, {h(2, x) * sin(x)});
            \addplot[PreciseGraph, olive, domain = -20 : 20]
            ({h(-2, x) * cos(x)}, {h(-2, x) * sin(x)});
            \addplot[PreciseGraph, red, domain = -20 : 20]
            ({h(2.5, x) * cos(x)}, {h(2.5, x) * sin(x)});
            \addplot[PreciseGraph, red, domain = -20 : 20]
            ({h(-2.5, x) * cos(x)}, {h(-2.5, x) * sin(x)});
            \addplot[PreciseGraph, gray, domain = -20 : 20]
            ({h(5, x) * cos(x)}, {h(5, x) * sin(x)});
            \addplot[PreciseGraph, gray, domain = -20 : 20]
            ({h(-5, x) * cos(x)}, {h(-5, x) * sin(x)});
            \addplot[PreciseGraph, orange, domain = -20 : 20]
            ({h(9, x) * cos(x)}, {h(9, x) * sin(x)});
            \addplot[PreciseGraph, orange, domain = -20 : 20]
            ({h(-9, x) * cos(x)}, {h(-9, x) * sin(x)});
            
            \foreach \k in {0.1, 0.2, 0.3, 0.5, 0.75, 1, 1.5, 2, 3} {
                \foreach \s in {-1, 1} {
                    \addplot[Graph, gray, domain = -4.5 : 4.5]
                    {f(\s * \k, x)};
                    \addplot[DirectionGraph1, domain = -4 : -0.25]
                    {f(\s * \k, x)};
                    \addplot[DirectionGraph1, domain = 0.25 : 4]
                    {f(\s * \k, x)};
                    \addplot[DirectionGraph2, domain = -4 : -0.25]
                    {f(\s * \k, x)};
                    \addplot[DirectionGraph2, domain = 0.25 : 4]
                    {f(\s * \k, x)};
                }
            }
            
            \foreach \k in {4.5, 15} {
                \foreach \s in {-1, 1} {
                    \addplot[Graph, gray, domain = -1 : 1]
                    {f(\s * \k, x)};
                    \addplot[DirectionGraph1, domain = -1.5 : -0.25]
                    {f(\s * \k, x)};
                    \addplot[DirectionGraph1, domain = 0.25 : 1.5]
                    {f(\s * \k, x)};
                    \addplot[DirectionGraph2, domain = -1.5 : -0.25]
                    {f(\s * \k, x)};
                    \addplot[DirectionGraph2, domain = 0.25 : 1.5]
                    {f(\s * \k, x)};
                }
            }
        \end{axis}
    \end{tikzpicture}
\end{minipage}
\end{figure}

\begin{remark} Кривые на рисунке справа построены \vspace{1mm} по общему решению исходного уравнения, которое можно задать либо в параметрической форме
$$\tsys{& x\br{t} = C f\br{t} \\& y\br{t} = Ct f\br{t}}\text{, где } f\br{t} = \dfrac {\tpow{e}{\elevate{\frac {1} {t}}}} {t},$$
либо в форме в полярных координатах
$\tsys{& x\br{\phi} = C r\br{\phi} \cos\phi \\& y\br{\phi} = C r\br{\phi} \sin\phi}$, где $r\br{\phi} = \dfrac { \tpow{e}{\ctg\phi}} {\sin\phi}$.
\end{remark} 

\end{document}
