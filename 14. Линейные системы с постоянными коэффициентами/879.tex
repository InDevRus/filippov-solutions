\documentclass[a5paper,10pt]{article}

% https://github.com/InDevRus/filippov-solutions
% Page setup
\pagenumbering{gobble}

% Margin setup
\usepackage{geometry}
\geometry{left=1cm}
\geometry{right=1cm}
\geometry{top=1cm}
\geometry{bottom=1.5cm}

% For inserting tables
\usepackage{array}

% Formula aligning
\usepackage{amsmath}

% For formula diacritics
\usepackage{amsfonts}

% Theorem definitions
\usepackage{amsthm}
\theoremstyle{remark}
\newtheorem*{necessity}{Необходимость}
\newtheorem*{sufficiency}{Достаточность}
\theoremstyle{definition}
\newtheorem*{theorem}{Теорема}
\newtheorem*{lemma}{Лемма}
\newtheorem*{corollary}{Следствие}
\newtheorem*{criteria}{Критерий}
\newtheorem*{algorithm}{Алгоритм}
\newtheorem*{formula}{Формула}
\newtheorem*{remark}{Замечание}
\newtheorem*{proposition}{Предложение}

% For bigger integral signs
\usepackage{bigints}

% For semantic advancements
\usepackage{enotez}

% For pictures
\usepackage{graphicx}

%For framing
\usepackage{framed}

% Automatic paragraph indentations
\usepackage{indentfirst}
\setlength{\parindent}{1em}

% Formula spacing configuration
\delimitershortfall-1sp
\usepackage{mleftright}
\mleftright

% For resizing
\usepackage{relsize}

% More convenient text-style common notations
\DeclareMathOperator*\lowlim{\underline{\lim}}
\DeclareMathOperator*\uplim{\overline{\lim}}

\newcommand{\tlowlim}[1]{\lowlim\limits_{#1}}
\newcommand{\tuplim}[1]{\uplim\limits_{#1}}

\newcommand{\tpow}[2]{{#1}^{\mathlarger{#2}}}

\newcommand{\tint}{\displaystyle{\int}}
\newcommand{\tintlim}[2]{\displaystyle{\int\limits_{#1}^{#2}}}
\newcommand{\tbigint}{\displaystyle{\mathlarger{\int}}}
\newcommand{\tbigintlim}[2]{\displaystyle{\mathlarger{\int}\limits_{#1}^{#2}}}

\newcommand{\tsumlim}[2]{\displaystyle{\mathlarger{\sum}\displaylimits_{#1}^{#2}}}
\newcommand{\tprodlim}[2]{\displaystyle{\mathlarger{\prod}\displaylimits_{#1}^{#2}}}

\newcommand{\tmin}[1]{\min\limits_{#1}}
\newcommand{\tmax}[1]{\max\limits_{#1}}
\newcommand{\tlim}[1]{\lim\limits_{#1}}

\newcommand{\norm}[1]{\left\lVert#1\right\rVert}

% For floaty text
\usepackage{wrapfig}
\usepackage{floatflt}

% For cyrillic characters support
\usepackage[english, russian]{babel}

% For proper font
\usepackage[no-math]{fontspec}

% TNR within text
\setmainfont{Times New Roman}

% TNR within formulas
\usepackage{newtxmath}
\DeclareSymbolFont{operators}{OT1}{ntxtlf}{m}{n}
\SetSymbolFont{operators}{bold}{OT1}{ntxtlf}{b}{n}

% For degree sign
\usepackage{siunitx}

% Automatic brackets placement
\newcommand{\br}[1]{\left(#1\right)}
\newcommand{\vbr}[1]{\left|#1\right|}
\newcommand{\cbr}[1]{\left\{#1\right\}}
\newcommand{\rbr}[1]{\left[#1\right]}
\renewcommand{\le}{\leqslant}
\renewcommand{\ge}{\geqslant} 

% Automatic replacement for two greek letters
\renewcommand{\epsilon}{\varepsilon}
\renewcommand{\phi}{\varphi}

% Redefinition of some operators (to the appropriation of Russian notation)
\DeclareMathOperator{\arcsh}{arcsh}
\DeclareMathOperator{\arcch}{arcch}
\DeclareMathOperator{\arcth}{arcth}
\DeclareMathOperator{\arccth}{arccth}
\DeclareMathOperator{\rank}{rank}
\DeclareMathOperator{\inv}{inv}
\DeclareMathOperator{\sgn}{sgn}
\renewcommand{\Re}{\operatorname{Re}}
\renewcommand{\Im}{\operatorname{Im}}

\renewcommand{\gcd}{\text{НОД}}
\newcommand{\lcm}{\text{НОК}}

% Restrict inlne formula breaking
\binoppenalty=10000 
\relpenalty=10000

% Greek letters setup
% Old greek letters setup
\DeclareSymbolFont{old_letters}{OML}{ztmcm}{m}{it}
\SetSymbolFont{old_letters}{bold}{OML}{ztmcm}{b}{it}

\newcommand{\Alpha}{\text{A}}
\newcommand{\Beta}{\text{B}}
\newcommand{\Epsilon}{\text{E}}
\newcommand{\Zeta}{\text{Z}}
\newcommand{\Eta}{\text{H}}
\newcommand{\Iota}{\text{I}}
\newcommand{\Kappa}{\text{K}}
\newcommand{\Mu}{\text{M}}
\newcommand{\Nu}{\text{N}}
\newcommand{\Omicron}{\text{O}}
\newcommand{\Rho}{\text{P}}
\newcommand{\Tau}{\text{T}}
\newcommand{\Chi}{\text{X}}

\DeclareMathSymbol{\alpha}{\mathord}{old_letters}{11}
\DeclareMathSymbol{\beta}{\mathord}{old_letters}{12}
\DeclareMathSymbol{\gamma}{\mathord}{old_letters}{13}
\DeclareMathSymbol{\delta}{\mathord}{old_letters}{14}
\DeclareMathSymbol{\varepsilon}{\mathord}{old_letters}{15}
\DeclareMathSymbol{\zeta}{\mathord}{old_letters}{16}
\DeclareMathSymbol{\eta}{\mathord}{old_letters}{17}
\DeclareMathSymbol{\theta}{\mathord}{old_letters}{18}
\DeclareMathSymbol{\iota}{\mathord}{old_letters}{19}
\DeclareMathSymbol{\kappa}{\mathord}{old_letters}{20}
\DeclareMathSymbol{\lambda}{\mathord}{old_letters}{21}
\DeclareMathSymbol{\mu}{\mathord}{old_letters}{22}
\DeclareMathSymbol{\nu}{\mathord}{old_letters}{23}
\DeclareMathSymbol{\xi}{\mathord}{old_letters}{24}
\newcommand{\omicron}{\text{\textit{\larger[1]{o}}}}
\DeclareMathSymbol{\pi}{\mathord}{old_letters}{25}
\DeclareMathSymbol{\rho}{\mathord}{old_letters}{26}
\DeclareMathSymbol{\sigma}{\mathord}{old_letters}{27}
\DeclareMathSymbol{\tau}{\mathord}{old_letters}{28}
\DeclareMathSymbol{\upsilon}{\mathord}{old_letters}{29}
\DeclareMathSymbol{\varphi}{\mathord}{old_letters}{39}
\DeclareMathSymbol{\chi}{\mathord}{old_letters}{31}
\DeclareMathSymbol{\psi}{\mathord}{old_letters}{32}
\DeclareMathSymbol{\omega}{\mathord}{old_letters}{33}






\begin{document}

\begin{framed}
\begin{formula}[вспомогательного угла]
$$A \sin{x} \pm B \cos{x} = \sqrt{A^2 + B^2} \sin\br{x \pm \phi},$$

где $\phi$ называется вспомогательным углом таким, что

$$
\sin{\phi} = \frac {B}{\sqrt{A^2 + B^2}},\ 
\cos{\phi} = \frac {A}{\sqrt{A^2 + B^2}},\
\tg{\phi} = \frac {B}{A}.
$$
В частности, можно взять $\phi = \arctg\br{\dfrac {B}{A}}$.
\end{formula}
\end{framed}

Обозначим за $I_{L}\br{t}$ силу тока на самоиндукции, за $I_{C}\br{t}$ -- силу тока на конденсаторе. Поскольку самоиндукция и конденсатор подключены параллельно, $U_{L}\br{t} = U_{C}\br{t}$, но $I_{\br{L, C}}\br{t} = I_{R}\br{t} = I_{L}\br{t} + I_{C}\br{t}$, так как сопротивление и пара из самоиндукции и конденсатора подключены уже последовательно. При этом $I_{R}\br{t}$ -- искомая сила тока на сопротивлении.

$U_{L}\br{t} = L \cdot I'_{L}\br{t}$, $U_{C}\br{t} = \frac {1}{C} q\br{t}$, где $q\br{t}$ -- заряд конденсатора, производная которого в точности равна $I_{C}\br{t}$. Взяв производную, получим $LC I''_{L}\br{t} = I_{C}\br{t}$.

Вся цепь соединена так, что падения напряжений её элементов равняются $U_{R}\br{t} = \linebreak = R I_{R}\br{t} = R \br{I_{L}\br{t} + I_{C}\br{t}}$ и $U_{\br{L, C}}\br{t} = U_{L}\br{t} = L \cdot I'_{L}\br{t}$. Отсюда имеем
$$R I_{L}\br{t} + R I_{C}\br{t} + L \cdot I'_{L}\br{t} = V \sin\br{\omega t}.$$

Неоднородная система уравнений 
$$\begin{cases}
    LC I''_{L}\br{t} - I_{C}\br{t} = 0 \\
    R I_{L}\br{t} + R I_{C}\br{t} + L \cdot I'_{L}\br{t} = V \sin\br{\omega t}
\end{cases}$$
хотя и не приведена к нормальному виду, проще всего может быть решена исключением $I_{C}\br{t}$ и нахождением функций по отдельности.

После подстановки имеем уравнение $RLC I''_{L}\br{t} + L \cdot I'_{L}\br{t} + R I_{L}\br{t} = V \sin\br{\omega t}$. Решение соответствующего однородного уравнения стремится к нулю при $t \to +\infty$ с показательной скоростью из-за того, что оба корня харакеристического уравнения имеют вещественные части, меньшие нуля. Следовательно, установившийся режим происходит из частного решения неоднородного уравнения.

Это решение будем искать в форме $I_{L}\br{t} = a \sin\br{\omega t} + b \cos\br{\omega t}$. Подставив в уравнение, получаем
$$\begin{cases}
        \br{-RLC\omega^2 + R} a - L\omega b = V \\
        L\omega a + \br{-RLC\omega^2 + R} b = 0
    \end{cases}$$
\begin{align*}
    \Delta_{0} = R^2\br{1 - LC\omega^2}^2 + \br{L\omega}^2 &&
    a = \frac {VR\br{1 - LC\omega^2}} {\Delta_{0}} &&
    b = \frac {-VL\omega} {\Delta_{0}} &&
\end{align*}

$$I_{L}\br{t} = \frac{V}{R^2\br{1 - LC\omega^2}^2 + \br{L\omega}^2} \br{R\br{1 - LC\omega^2} \sin\br{\omega t} - L \cos\br{\omega t}}$$
\begin{equation*}
    \begin{split}
        I_{R}\br{t} = I_{L}\br{t} + LCI''_{L}\br{t} = \frac {V\br{1 - LC\omega^2}} {R^2\br{1 - LC\omega^2}^2 + \br{L\omega}^2} \br{R \sin\br{\omega t} - \frac {L\omega} {1 - LC\omega^2} \cos\br{\omega t}} & \\
        = \frac {V} {R^2 + \br{\dfrac {L\omega}{1 - LC\omega^2}}^2} \br{R \sin\br{\omega t} - \frac {L\omega} {1 - LC\omega^2} \cos\br{\omega t}} & \\
        = \frac {V} {\sqrt{R^2 + \br{\dfrac {L\omega}{1 - LC\omega^2}}^2}} \sin\br{\omega t - \phi},
    \end{split}
\end{equation*}
где $\phi = \arctg\br{\dfrac{L\omega}{R\br{1 - LC\omega^2}}}$.
Амплитуда установившегося режима равняется $A\br{\omega} = \linebreak = \dfrac {V} {\sqrt{R^2 + \br{\dfrac {L\omega}{1 - LC\omega^2}}^2}}$ и она убывает с возрастанием выражения $\br{\dfrac {L\omega}{1 - LC\omega^2}}^2$.

Наибольшая амплитуда получается при $\omega = 0$ и при $\omega \to +\infty$ и равна $A_{\max} = \dfrac{V}{R}$. Наименьшая будет при $\omega = \pm \dfrac {1} {\sqrt{LC}}$ и равна $A_{\min} = 0$.
\end{document}
