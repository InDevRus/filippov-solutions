\documentclass[a5paper,10pt]{article}

% https://github.com/InDevRus/filippov-solutions
% Page setup
\pagenumbering{gobble}

% Margin setup
\usepackage{geometry}
\geometry{left=1cm}
\geometry{right=1cm}
\geometry{top=1cm}
\geometry{bottom=1.5cm}

% For inserting tables
\usepackage{array}

% Formula aligning
\usepackage{amsmath}

% For formula diacritics
\usepackage{amsfonts}

% Theorem definitions
\usepackage{amsthm}
\theoremstyle{remark}
\newtheorem*{necessity}{Необходимость}
\newtheorem*{sufficiency}{Достаточность}
\theoremstyle{definition}
\newtheorem*{theorem}{Теорема}
\newtheorem*{lemma}{Лемма}
\newtheorem*{corollary}{Следствие}
\newtheorem*{criteria}{Критерий}
\newtheorem*{algorithm}{Алгоритм}
\newtheorem*{formula}{Формула}
\newtheorem*{remark}{Замечание}
\newtheorem*{proposition}{Предложение}

% For bigger integral signs
\usepackage{bigints}

% For semantic advancements
\usepackage{enotez}

% For pictures
\usepackage{graphicx}

%For framing
\usepackage{framed}

% Automatic paragraph indentations
\usepackage{indentfirst}
\setlength{\parindent}{1em}

% Formula spacing configuration
\delimitershortfall-1sp
\usepackage{mleftright}
\mleftright

% For resizing
\usepackage{relsize}

% More convenient text-style common notations
\DeclareMathOperator*\lowlim{\underline{\lim}}
\DeclareMathOperator*\uplim{\overline{\lim}}

\newcommand{\tlowlim}[1]{\lowlim\limits_{#1}}
\newcommand{\tuplim}[1]{\uplim\limits_{#1}}

\newcommand{\tpow}[2]{{#1}^{\mathlarger{#2}}}

\newcommand{\tint}{\displaystyle{\int}}
\newcommand{\tintlim}[2]{\displaystyle{\int\limits_{#1}^{#2}}}
\newcommand{\tbigint}{\displaystyle{\mathlarger{\int}}}
\newcommand{\tbigintlim}[2]{\displaystyle{\mathlarger{\int}\limits_{#1}^{#2}}}

\newcommand{\tsumlim}[2]{\displaystyle{\mathlarger{\sum}\displaylimits_{#1}^{#2}}}
\newcommand{\tprodlim}[2]{\displaystyle{\mathlarger{\prod}\displaylimits_{#1}^{#2}}}

\newcommand{\tmin}[1]{\min\limits_{#1}}
\newcommand{\tmax}[1]{\max\limits_{#1}}
\newcommand{\tlim}[1]{\lim\limits_{#1}}

\newcommand{\norm}[1]{\left\lVert#1\right\rVert}

% For floaty text
\usepackage{wrapfig}
\usepackage{floatflt}

% For cyrillic characters support
\usepackage[english, russian]{babel}

% For proper font
\usepackage[no-math]{fontspec}

% TNR within text
\setmainfont{Times New Roman}

% TNR within formulas
\usepackage{newtxmath}
\DeclareSymbolFont{operators}{OT1}{ntxtlf}{m}{n}
\SetSymbolFont{operators}{bold}{OT1}{ntxtlf}{b}{n}

% For degree sign
\usepackage{siunitx}

% Automatic brackets placement
\newcommand{\br}[1]{\left(#1\right)}
\newcommand{\vbr}[1]{\left|#1\right|}
\newcommand{\cbr}[1]{\left\{#1\right\}}
\newcommand{\rbr}[1]{\left[#1\right]}
\renewcommand{\le}{\leqslant}
\renewcommand{\ge}{\geqslant} 

% Automatic replacement for two greek letters
\renewcommand{\epsilon}{\varepsilon}
\renewcommand{\phi}{\varphi}

% Redefinition of some operators (to the appropriation of Russian notation)
\DeclareMathOperator{\arcsh}{arcsh}
\DeclareMathOperator{\arcch}{arcch}
\DeclareMathOperator{\arcth}{arcth}
\DeclareMathOperator{\arccth}{arccth}
\DeclareMathOperator{\rank}{rank}
\DeclareMathOperator{\inv}{inv}
\DeclareMathOperator{\sgn}{sgn}
\renewcommand{\Re}{\operatorname{Re}}
\renewcommand{\Im}{\operatorname{Im}}

\renewcommand{\gcd}{\text{НОД}}
\newcommand{\lcm}{\text{НОК}}

% Restrict inlne formula breaking
\binoppenalty=10000 
\relpenalty=10000

% Greek letters setup
% Old greek letters setup
\DeclareSymbolFont{old_letters}{OML}{ztmcm}{m}{it}
\SetSymbolFont{old_letters}{bold}{OML}{ztmcm}{b}{it}

\newcommand{\Alpha}{\text{A}}
\newcommand{\Beta}{\text{B}}
\newcommand{\Epsilon}{\text{E}}
\newcommand{\Zeta}{\text{Z}}
\newcommand{\Eta}{\text{H}}
\newcommand{\Iota}{\text{I}}
\newcommand{\Kappa}{\text{K}}
\newcommand{\Mu}{\text{M}}
\newcommand{\Nu}{\text{N}}
\newcommand{\Omicron}{\text{O}}
\newcommand{\Rho}{\text{P}}
\newcommand{\Tau}{\text{T}}
\newcommand{\Chi}{\text{X}}

\DeclareMathSymbol{\alpha}{\mathord}{old_letters}{11}
\DeclareMathSymbol{\beta}{\mathord}{old_letters}{12}
\DeclareMathSymbol{\gamma}{\mathord}{old_letters}{13}
\DeclareMathSymbol{\delta}{\mathord}{old_letters}{14}
\DeclareMathSymbol{\varepsilon}{\mathord}{old_letters}{15}
\DeclareMathSymbol{\zeta}{\mathord}{old_letters}{16}
\DeclareMathSymbol{\eta}{\mathord}{old_letters}{17}
\DeclareMathSymbol{\theta}{\mathord}{old_letters}{18}
\DeclareMathSymbol{\iota}{\mathord}{old_letters}{19}
\DeclareMathSymbol{\kappa}{\mathord}{old_letters}{20}
\DeclareMathSymbol{\lambda}{\mathord}{old_letters}{21}
\DeclareMathSymbol{\mu}{\mathord}{old_letters}{22}
\DeclareMathSymbol{\nu}{\mathord}{old_letters}{23}
\DeclareMathSymbol{\xi}{\mathord}{old_letters}{24}
\newcommand{\omicron}{\text{\textit{\larger[1]{o}}}}
\DeclareMathSymbol{\pi}{\mathord}{old_letters}{25}
\DeclareMathSymbol{\rho}{\mathord}{old_letters}{26}
\DeclareMathSymbol{\sigma}{\mathord}{old_letters}{27}
\DeclareMathSymbol{\tau}{\mathord}{old_letters}{28}
\DeclareMathSymbol{\upsilon}{\mathord}{old_letters}{29}
\DeclareMathSymbol{\varphi}{\mathord}{old_letters}{39}
\DeclareMathSymbol{\chi}{\mathord}{old_letters}{31}
\DeclareMathSymbol{\psi}{\mathord}{old_letters}{32}
\DeclareMathSymbol{\omega}{\mathord}{old_letters}{33}






\usepackage{pgfplots}
\pgfplotsset{compat=newest}
\pgfplotsset{
    every axis/.append style = {
        xlabel={$x$},
        ylabel={$y$}, 
        axis lines = middle,
        unit vector ratio = 1 1,
        font=\small,
    },
    every axis plot/.append style = {smooth, samples = 100}
}

\begin{document}

$\tsys{& \dot{x} = 2x - y \\& \dot{y} = x}$.
$A = \begin{pmatrix} 2 & -1 \\ 1 & 0 \end{pmatrix}$.
$\det\br{\lambda E - A} = \lambda \br{\lambda - 2} + 1$.
$\lambda_{1} = \lambda_{2} = 1$. Особая точка является <<вырожденным узлом>>, причём неустойчивым, так как $\lambda_{1} = \lambda_{2} > 0$.

$A - \lambda_{1}E = 
\begin{pmatrix} 1 & -1 \\ 1 & -1 \end{pmatrix}
\sim \begin{pmatrix} 1 & -1 \end{pmatrix}$, $v_{1} = \begin{pmatrix} 1 \\ 1 \end{pmatrix}$.
Единственная сепаратриса $y_{1} = x$ будет являться касательной для всех траекторий в начале координат.

Решения можно построить как при помощи изоклин, так и путём построения графика параметрических функций: 
$\tsys{& x = e^{t} \br{t + 1} + C t e^{t} \\& y = t e^{t} + C e^{t} \br{t - 1} }$.
На левом рисунке указаны изоклины, на правом -- траектории.

\begin{figure}[!ht]
    \centering
    \begin{subfigure}[t]{0.495\textwidth}
        \centering
        \begin{tikzpicture}
            \pgfplotsset{
                VectorGraph/.style = {
                    samples = 17,
                    domain = -1 : 1,
                    restrict y to domain = -2 : 2,
                    skip coords between index = {8}{9},
                    quiver = {
                        scale arrows = 0.05,
                        every arrow/.append style = {
                            thin,
                            line width = 0.1pt
                        }
                    },
                },
                VectorGraph_1/.style = {
                    VectorGraph,
                    -stealth,
                    quiver = {
                        u = {dot_x(x, y) / (dot_x(x, y) ^ 2 + dot_y(x, y) ^ 2) ^ 0.5},
                        v = {dot_y(x, y) / (dot_x(x, y) ^ 2 + dot_y(x, y) ^ 2) ^ 0.5}
                    },
                },
                VectorGraph_2/.style = {
                    VectorGraph,
                    quiver = {
                        u = {-dot_x(x, y) / (dot_x(x, y) ^ 2 + dot_y(x, y) ^ 2) ^ 0.5},
                        v = {-dot_y(x, y) / (dot_x(x, y) ^ 2 + dot_y(x, y) ^ 2) ^ 0.5}
                    }
                },
            }
            \begin{axis}[
                height = 9.5cm,
                width = \linewidth,
                ymin = -2.4,
                ymax = 2.4,
                xmin = -1.4,
                xmax = 1.4, 
                xtick = {-3, -2, ..., 3},
                ytick = {-5, -4, ..., 5},
                minor tick num = 1,
                declare function = {
                    f(\C, \x) = \C * ((x - 1) ^ 2) / ((x - 2) ^ 3);
                    dot_x(\x, \y) = 2 * x - y;
                    dot_y(\x, \y) = x;
                }],
                
                \foreach \s in {-1, 1} {
                    \foreach \k in {1 / 8, 1 / 3, 1 / 2, 2 / 3, 2, 3, 3 / 2, 5, 10} {
                        \addplot[dotted] {\s * \k * x};
                        \addplot[VectorGraph_1] {\s * \k * x};
                        \addplot[VectorGraph_2] {\s * \k * x};
                    }
                }
                
                \addplot[VectorGraph_1, domain = -2 : 2] ({0}, {x});
                \addplot[VectorGraph_2, domain = -2 : 2] ({0}, {x});
                
                \foreach \k in {-1} {
                    \addplot[dotted] {\k * x};
                    \addplot[VectorGraph_1] {\k * x};
                    \addplot[VectorGraph_2] {\k * x};
                }
                
                \addplot [thick] {x};
            \end{axis}
        \end{tikzpicture}
    \end{subfigure}
    \begin{subfigure}[t]{0.495\textwidth}
        \centering
        \begin{tikzpicture}
            \pgfplotsset{
                GeneralGraph/.style = {
                    domain = -10 : 3,
                    samples = 250
                },
                VectorGraph/.style = {
                    samples = 10,
                    domain = -2 : 1.5,
                    quiver = {
                        scale arrows = 0.05,
                        every arrow/.append style = {
                            thin,
                            line width = 0.1pt
                        }
                    },
                },
                VectorGraph_1/.style = {
                    VectorGraph,
                    -stealth,
                    quiver = {
                        u = {dot_x(x, y) / (dot_x(x, y) ^ 2 + dot_y(x, y) ^ 2) ^ 0.5},
                        v = {dot_y(x, y) / (dot_x(x, y) ^ 2 + dot_y(x, y) ^ 2) ^ 0.5}
                    },
                },
                VectorGraph_2/.style = {
                    VectorGraph_1,
                    samples = 5,
                    domain = -1 : 1,
                },
            }
            \begin{axis}[
                height = 9.5cm,
                width = \textwidth,
                ymin = -2.4,
                ymax = 2.4,
                xmin = -1.4,
                xmax = 1.4, 
                xtick = {-3, -2, ..., 3},
                ytick = {-5, -4, ..., 5},
                minor tick num = 1,
                declare function = {
                    f(\C, \x) = exp(x) * (x + 1) + \C * x * exp(x);
                    g(\C, \x) = x * exp(x) + \C * exp(x) * (x - 1);
                    dot_x(\x, \y) = 2 * x - y;
                    dot_y(\x, \y) = x;
                }],
                
                \addplot [thick] {x};
                \addplot [VectorGraph_1] {x};
                
                \foreach \s in {-1, 1} {
                    \addplot [GeneralGraph, blue] ({\s * f(1, x)}, {\s * g(1, x)});
                    \addplot [VectorGraph_1, blue] ({\s * f(1, x)}, {\s * g(1, x)});
                }
                
                \foreach \s in {-1, 1} {
                    \addplot [GeneralGraph, brown] ({\s * f(-2, x)}, {\s * g(-2, x)});
                    \addplot [VectorGraph_1, brown] ({\s * f(-2, x)}, {\s * g(-2, x)});
                }
                
                \foreach \s in {-1, 1} {
                    \addplot [GeneralGraph, green] ({\s * f(1.5, x)}, {\s * g(1.5, x)});
                    \addplot [VectorGraph_1, green] ({\s * f(1.5, x)}, {\s * g(1.5, x)});
                }
                
                \foreach \s in {-1, 1} {
                    \addplot [GeneralGraph, red] ({\s * f(2, x)}, {\s * g(2, x)});
                    \addplot [VectorGraph_1, red] ({\s * f(2, x)}, {\s * g(2, x)});
                }
                
                \foreach \s in {-1, 1} {
                    \addplot [GeneralGraph, gray] ({\s * f(0.25, x)}, {\s * g(0.25, x)});
                    \addplot [VectorGraph_2, gray] ({\s * f(0.25, x)}, {\s * g(0.25, x)});
                }
                
                \foreach \s in {-1, 1} {
                    \addplot [GeneralGraph, magenta] ({\s * f(-0.5, x)}, {\s * g(-0.5, x)});
                    \addplot [VectorGraph_2, magenta] ({\s * f(-0.5, x)}, {\s * g(-0.5, x)});
                }
            \end{axis}
        \end{tikzpicture}
    \end{subfigure}
\end{figure}

\end{document}
