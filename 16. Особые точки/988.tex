\documentclass[a5paper,10pt]{article}

% https://github.com/InDevRus/filippov-solutions
% Page setup
\pagenumbering{gobble}

% Margin setup
\usepackage{geometry}
\geometry{left=1cm}
\geometry{right=1cm}
\geometry{top=1cm}
\geometry{bottom=1.5cm}

% For inserting tables
\usepackage{array}

% Formula aligning
\usepackage{amsmath}

% For formula diacritics
\usepackage{amsfonts}

% Theorem definitions
\usepackage{amsthm}
\theoremstyle{remark}
\newtheorem*{necessity}{Необходимость}
\newtheorem*{sufficiency}{Достаточность}
\theoremstyle{definition}
\newtheorem*{theorem}{Теорема}
\newtheorem*{lemma}{Лемма}
\newtheorem*{corollary}{Следствие}
\newtheorem*{criteria}{Критерий}
\newtheorem*{algorithm}{Алгоритм}
\newtheorem*{formula}{Формула}
\newtheorem*{remark}{Замечание}
\newtheorem*{proposition}{Предложение}

% For bigger integral signs
\usepackage{bigints}

% For semantic advancements
\usepackage{enotez}

% For pictures
\usepackage{graphicx}

%For framing
\usepackage{framed}

% Automatic paragraph indentations
\usepackage{indentfirst}
\setlength{\parindent}{1em}

% Formula spacing configuration
\delimitershortfall-1sp
\usepackage{mleftright}
\mleftright

% For resizing
\usepackage{relsize}

% More convenient text-style common notations
\DeclareMathOperator*\lowlim{\underline{\lim}}
\DeclareMathOperator*\uplim{\overline{\lim}}

\newcommand{\tlowlim}[1]{\lowlim\limits_{#1}}
\newcommand{\tuplim}[1]{\uplim\limits_{#1}}

\newcommand{\tpow}[2]{{#1}^{\mathlarger{#2}}}

\newcommand{\tint}{\displaystyle{\int}}
\newcommand{\tintlim}[2]{\displaystyle{\int\limits_{#1}^{#2}}}
\newcommand{\tbigint}{\displaystyle{\mathlarger{\int}}}
\newcommand{\tbigintlim}[2]{\displaystyle{\mathlarger{\int}\limits_{#1}^{#2}}}

\newcommand{\tsumlim}[2]{\displaystyle{\mathlarger{\sum}\displaylimits_{#1}^{#2}}}
\newcommand{\tprodlim}[2]{\displaystyle{\mathlarger{\prod}\displaylimits_{#1}^{#2}}}

\newcommand{\tmin}[1]{\min\limits_{#1}}
\newcommand{\tmax}[1]{\max\limits_{#1}}
\newcommand{\tlim}[1]{\lim\limits_{#1}}

\newcommand{\norm}[1]{\left\lVert#1\right\rVert}

% For floaty text
\usepackage{wrapfig}
\usepackage{floatflt}

% For cyrillic characters support
\usepackage[english, russian]{babel}

% For proper font
\usepackage[no-math]{fontspec}

% TNR within text
\setmainfont{Times New Roman}

% TNR within formulas
\usepackage{newtxmath}
\DeclareSymbolFont{operators}{OT1}{ntxtlf}{m}{n}
\SetSymbolFont{operators}{bold}{OT1}{ntxtlf}{b}{n}

% For degree sign
\usepackage{siunitx}

% Automatic brackets placement
\newcommand{\br}[1]{\left(#1\right)}
\newcommand{\vbr}[1]{\left|#1\right|}
\newcommand{\cbr}[1]{\left\{#1\right\}}
\newcommand{\rbr}[1]{\left[#1\right]}
\renewcommand{\le}{\leqslant}
\renewcommand{\ge}{\geqslant} 

% Automatic replacement for two greek letters
\renewcommand{\epsilon}{\varepsilon}
\renewcommand{\phi}{\varphi}

% Redefinition of some operators (to the appropriation of Russian notation)
\DeclareMathOperator{\arcsh}{arcsh}
\DeclareMathOperator{\arcch}{arcch}
\DeclareMathOperator{\arcth}{arcth}
\DeclareMathOperator{\arccth}{arccth}
\DeclareMathOperator{\rank}{rank}
\DeclareMathOperator{\inv}{inv}
\DeclareMathOperator{\sgn}{sgn}
\renewcommand{\Re}{\operatorname{Re}}
\renewcommand{\Im}{\operatorname{Im}}

\renewcommand{\gcd}{\text{НОД}}
\newcommand{\lcm}{\text{НОК}}

% Restrict inlne formula breaking
\binoppenalty=10000 
\relpenalty=10000

% Greek letters setup
% Old greek letters setup
\DeclareSymbolFont{old_letters}{OML}{ztmcm}{m}{it}
\SetSymbolFont{old_letters}{bold}{OML}{ztmcm}{b}{it}

\newcommand{\Alpha}{\text{A}}
\newcommand{\Beta}{\text{B}}
\newcommand{\Epsilon}{\text{E}}
\newcommand{\Zeta}{\text{Z}}
\newcommand{\Eta}{\text{H}}
\newcommand{\Iota}{\text{I}}
\newcommand{\Kappa}{\text{K}}
\newcommand{\Mu}{\text{M}}
\newcommand{\Nu}{\text{N}}
\newcommand{\Omicron}{\text{O}}
\newcommand{\Rho}{\text{P}}
\newcommand{\Tau}{\text{T}}
\newcommand{\Chi}{\text{X}}

\DeclareMathSymbol{\alpha}{\mathord}{old_letters}{11}
\DeclareMathSymbol{\beta}{\mathord}{old_letters}{12}
\DeclareMathSymbol{\gamma}{\mathord}{old_letters}{13}
\DeclareMathSymbol{\delta}{\mathord}{old_letters}{14}
\DeclareMathSymbol{\varepsilon}{\mathord}{old_letters}{15}
\DeclareMathSymbol{\zeta}{\mathord}{old_letters}{16}
\DeclareMathSymbol{\eta}{\mathord}{old_letters}{17}
\DeclareMathSymbol{\theta}{\mathord}{old_letters}{18}
\DeclareMathSymbol{\iota}{\mathord}{old_letters}{19}
\DeclareMathSymbol{\kappa}{\mathord}{old_letters}{20}
\DeclareMathSymbol{\lambda}{\mathord}{old_letters}{21}
\DeclareMathSymbol{\mu}{\mathord}{old_letters}{22}
\DeclareMathSymbol{\nu}{\mathord}{old_letters}{23}
\DeclareMathSymbol{\xi}{\mathord}{old_letters}{24}
\newcommand{\omicron}{\text{\textit{\larger[1]{o}}}}
\DeclareMathSymbol{\pi}{\mathord}{old_letters}{25}
\DeclareMathSymbol{\rho}{\mathord}{old_letters}{26}
\DeclareMathSymbol{\sigma}{\mathord}{old_letters}{27}
\DeclareMathSymbol{\tau}{\mathord}{old_letters}{28}
\DeclareMathSymbol{\upsilon}{\mathord}{old_letters}{29}
\DeclareMathSymbol{\varphi}{\mathord}{old_letters}{39}
\DeclareMathSymbol{\chi}{\mathord}{old_letters}{31}
\DeclareMathSymbol{\psi}{\mathord}{old_letters}{32}
\DeclareMathSymbol{\omega}{\mathord}{old_letters}{33}






\begin{document}

\begin{framed}
    \begin{proposition}\end{proposition}
    При $r\br{\xi; \phi} = \sqrt{\xi^2 + \phi^2} \to 0$ имеем $\xi \phi = O\br{r^2}$. В частности, $\dfrac{\xi \phi} {r^{1 + \epsilon}} = \dfrac {\gamma\br{r} r^2} {r^{1 + \epsilon}} \to 0$ для всех $\epsilon \in \br{0; 1}$.
    
    Действительно, 
    $0 \le \br{\vbr{\xi} - \vbr{\phi}}^2$
    и 
    $\vbr{\xi \phi} 
    \le \dfrac {1} {2} \br{\xi^2 + \phi^2}
    \le \dfrac {1} {2} \br{r^2 + r^2}
    = r^2$,
    что показывает $\xi \phi = O\br{r^2}$.
\end{framed}

$\tsys{
    & P\br{x; y} = \sqrt{x^2 - y + 2} - 2
    \\& Q\br{x; y} = \arctg\br{x^2 + xy}
}$;
$\tsys{
    & \sqrt{x^2 - y + 2} - 2 = 0
    \\& \arctg\br{x^2 + xy} = 0
}$;
$\tsys{
    & y = x^2 - 2
    \\& x^2 + xy = 0
}$;
$x \br{x + x^2 - 2} = 0$.
$\tsys{
    & x_{1} = 0
    \\& y_{1} = -2
}$,
$\tsys{
    & x_{2} = 1
    \\& y_{2} = -1 
}$,
$\tsys{
    & x_{3} = -2
    \\& y_{3} = 2
}$, и особые точки -- $A_{1}\br{0\; -2}$, $A_{2}\br{1; -1}$ и $A_{3} = \br{-2; 2}$.

Будем также пользоваться тем, что $\sqrt{1 + u} = 1 + \dfrac {1} {2}u + O\br{u^2}$ и $\arctg\br{u} = u + O\br{u^3}$ при $u \to 0$.

\begin{enumerate}
    \item $y = \phi - 2$. \\
    $\dot{x} = \sqrt{x^2 + \phi + 4} - 2 
    = 2\sqrt{1 + \dfrac {x^2} {4} - \dfrac {\phi} {4}} - 2 
    = 2\ \br{1 + \dfrac {1} {2} \br{\dfrac {x^2} {4} - \dfrac {\phi} {4}} + O\br{\br{\dfrac {x^2} {4} - \dfrac {\phi} {4}}^2}} - 2 = \linebreak 
    = - \dfrac {\phi} {4} + O\br{r^2}$. $\dot{\phi} = \arctg\br{x^2 + x\phi - 2x} = x^2 + x\phi - 2x + O\br{\br{x^2 + x\phi - 2x}^3} = -2x + O\br{r^2}$.
    $A = \begin{pmatrix} 0 & - \frac {1} {4} \vspace{1mm} \\ -2 & \hspace{2mm} 0 \end{pmatrix}$.
    $\lambda^2 - \dfrac {1} {2} = 0$.
    $\lambda_{1} = -\dfrac {1} {\sqrt{2}}$, $\lambda_{2} = \dfrac {1} {\sqrt{2}}$, а особая точка $A_{1}\br{0; -2}$ -- <<седло>>.
    
    Найдём сепаратрисы. 
    $A - \lambda_{1} E \sim \begin{pmatrix} -2\sqrt{2} & 1\end{pmatrix}$,
    $v_{1} = \begin{pmatrix} 1 \\ 2\sqrt{2} \end{pmatrix}$,
    $\phi_{1} = 2\sqrt{2} x$. 
    Аналогично, 
    $A - \lambda_{2} E \sim \begin{pmatrix} 2\sqrt{2} & 1\end{pmatrix}$,
    $v_{2} = \begin{pmatrix} 1 \\ -2\sqrt{2} \end{pmatrix}$, 
    $\phi_{2} = -2\sqrt{2} x$.
    
    \item $\tsys{& x = \xi + 1 \\& y = \phi - 1}$. $\dot{\xi} = \sqrt{\xi^2 + 2\xi - \phi + 4} - 2 
    = 2\sqrt{1 + \dfrac {\xi^2} {4} + \dfrac {\xi} {2} - \dfrac {\phi} {4}} - 2 
    = 2\ \left( 1 + \dfrac {1}{2} \left(\dfrac {\xi^2} {4} + \dfrac {\xi} {2} - \right. \right. \linebreak 
    - \left.\left. \dfrac {\phi} {4} \right) + O\br{\br{\dfrac {\xi^2} {4} + \dfrac {\xi} {2} - \dfrac {\phi} {4}}^2} \right) - 2 = \dfrac {\xi} {2} - \dfrac {\phi} {4} + O\br{r^2}$. $\dot{\phi} = \arctg\br{\br{\xi + 1}\br{\xi + \phi}} = \xi^2 + \xi + \xi\phi + \linebreak 
    +\ \phi + O\br{r^3} = \xi + \phi + O\br{r^2}$.
    $A = \begin{pmatrix} \frac {1} {2} & - \frac {1} {4} \vspace{1mm} \\ 1 & \hspace{2mm} 1 \end{pmatrix}$.
    $\lambda_{1, 2} = \dfrac {3 \mp \sqrt{3} i} {4}$, и особая точка $A_{2}\br{1; -1}$ является <<фокусом>>, причём неустойчивым, то есть траектории раскручиваются.
    
    В $\tsys{& \xi = -0,6 \\& \phi = 0,6}$ для 
    $\tsys{
        & \dot{\xi} = \sqrt{\xi^2 + 2\xi - \phi + 4} - 2 
        \\& \dot{\phi} = \arctg\br{\br{\xi + 1}\br{\xi + \phi}}
    }$ имеем вектор $\tsys{& \dot{\xi}\br{-0,6; 0,6} = -0,4 \\& \dot{\phi}\br{-0,6; 0,6} = 0}$. \linebreak Следовательно, направление раскручивания траекторий -- против часовой.
    
    \item $\tsys{& x = \xi - 2 \\& y = \phi + 2}$.
    $\dot{\xi} = \sqrt{\xi^2 - 4\xi - \phi + 4} - 2
    = 2\ \br{1 + \dfrac {1} {2} \br{\dfrac {\xi^2} {4} - \xi - \dfrac {\phi} {4}} + O\br{r^2}} - 2
    = -\xi - \linebreak - \dfrac {\phi} {4} + O\br{r^2}$.
    $\dot{\phi} = \arctg\br{\br{\xi - 2} \br{\xi + \phi}}
    = \xi^2 - 2\xi + \xi \phi - 2\phi + O\br{r^3}
    = -2\xi - 2\phi + O\br{r^2}$.
    $A = \begin{pmatrix} -1 & - \frac {1} {4} \vspace{1mm} \\ 1 & \hspace{2mm} 1 \end{pmatrix}$.
    $\lambda_{1} = \dfrac {-3 - \sqrt{3}} {2}$.
    $\lambda_{2} = \dfrac {-3 + \sqrt{3}} {2}$.
    $\lambda_{1} < \lambda_{2} < 0$, поэтому особая точка $A_{3}\br{-2; 2}$ -- <<узел>>, при этом устойчивый.
    В полной аналогии с предыдущими случаями находим
    $v_{1} = \begin{pmatrix} 1 \\ 2 + 2\sqrt{3} \end{pmatrix}$,
    $v_{2} = \begin{pmatrix} 1 \\ 2 - 2\sqrt{3} \end{pmatrix}$;
    $\phi_{1} = \br{2 + 2\sqrt{3}} \xi$ -- трансверсаль, а
    $\phi_{2} = \br{2 - 2\sqrt{3}} \xi$ -- касательная.
\end{enumerate}

\end{document}
