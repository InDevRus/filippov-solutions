\documentclass[a5paper,10pt]{article}

% https://github.com/InDevRus/filippov-solutions
% Page setup
\pagenumbering{gobble}

% Margin setup
\usepackage{geometry}
\geometry{left=1cm}
\geometry{right=1cm}
\geometry{top=1cm}
\geometry{bottom=1.5cm}

% For inserting tables
\usepackage{array}

% Formula aligning
\usepackage{amsmath}

% For formula diacritics
\usepackage{amsfonts}

% Theorem definitions
\usepackage{amsthm}
\theoremstyle{remark}
\newtheorem*{necessity}{Необходимость}
\newtheorem*{sufficiency}{Достаточность}
\theoremstyle{definition}
\newtheorem*{theorem}{Теорема}
\newtheorem*{lemma}{Лемма}
\newtheorem*{corollary}{Следствие}
\newtheorem*{criteria}{Критерий}
\newtheorem*{algorithm}{Алгоритм}
\newtheorem*{formula}{Формула}
\newtheorem*{remark}{Замечание}
\newtheorem*{proposition}{Предложение}

% For bigger integral signs
\usepackage{bigints}

% For semantic advancements
\usepackage{enotez}

% For pictures
\usepackage{graphicx}

%For framing
\usepackage{framed}

% Automatic paragraph indentations
\usepackage{indentfirst}
\setlength{\parindent}{1em}

% Formula spacing configuration
\delimitershortfall-1sp
\usepackage{mleftright}
\mleftright

% For resizing
\usepackage{relsize}

% More convenient text-style common notations
\DeclareMathOperator*\lowlim{\underline{\lim}}
\DeclareMathOperator*\uplim{\overline{\lim}}

\newcommand{\tlowlim}[1]{\lowlim\limits_{#1}}
\newcommand{\tuplim}[1]{\uplim\limits_{#1}}

\newcommand{\tpow}[2]{{#1}^{\mathlarger{#2}}}

\newcommand{\tint}{\displaystyle{\int}}
\newcommand{\tintlim}[2]{\displaystyle{\int\limits_{#1}^{#2}}}
\newcommand{\tbigint}{\displaystyle{\mathlarger{\int}}}
\newcommand{\tbigintlim}[2]{\displaystyle{\mathlarger{\int}\limits_{#1}^{#2}}}

\newcommand{\tsumlim}[2]{\displaystyle{\mathlarger{\sum}\displaylimits_{#1}^{#2}}}
\newcommand{\tprodlim}[2]{\displaystyle{\mathlarger{\prod}\displaylimits_{#1}^{#2}}}

\newcommand{\tmin}[1]{\min\limits_{#1}}
\newcommand{\tmax}[1]{\max\limits_{#1}}
\newcommand{\tlim}[1]{\lim\limits_{#1}}

\newcommand{\norm}[1]{\left\lVert#1\right\rVert}

% For floaty text
\usepackage{wrapfig}
\usepackage{floatflt}

% For cyrillic characters support
\usepackage[english, russian]{babel}

% For proper font
\usepackage[no-math]{fontspec}

% TNR within text
\setmainfont{Times New Roman}

% TNR within formulas
\usepackage{newtxmath}
\DeclareSymbolFont{operators}{OT1}{ntxtlf}{m}{n}
\SetSymbolFont{operators}{bold}{OT1}{ntxtlf}{b}{n}

% For degree sign
\usepackage{siunitx}

% Automatic brackets placement
\newcommand{\br}[1]{\left(#1\right)}
\newcommand{\vbr}[1]{\left|#1\right|}
\newcommand{\cbr}[1]{\left\{#1\right\}}
\newcommand{\rbr}[1]{\left[#1\right]}
\renewcommand{\le}{\leqslant}
\renewcommand{\ge}{\geqslant} 

% Automatic replacement for two greek letters
\renewcommand{\epsilon}{\varepsilon}
\renewcommand{\phi}{\varphi}

% Redefinition of some operators (to the appropriation of Russian notation)
\DeclareMathOperator{\arcsh}{arcsh}
\DeclareMathOperator{\arcch}{arcch}
\DeclareMathOperator{\arcth}{arcth}
\DeclareMathOperator{\arccth}{arccth}
\DeclareMathOperator{\rank}{rank}
\DeclareMathOperator{\inv}{inv}
\DeclareMathOperator{\sgn}{sgn}
\renewcommand{\Re}{\operatorname{Re}}
\renewcommand{\Im}{\operatorname{Im}}

\renewcommand{\gcd}{\text{НОД}}
\newcommand{\lcm}{\text{НОК}}

% Restrict inlne formula breaking
\binoppenalty=10000 
\relpenalty=10000

% Greek letters setup
% Old greek letters setup
\DeclareSymbolFont{old_letters}{OML}{ztmcm}{m}{it}
\SetSymbolFont{old_letters}{bold}{OML}{ztmcm}{b}{it}

\newcommand{\Alpha}{\text{A}}
\newcommand{\Beta}{\text{B}}
\newcommand{\Epsilon}{\text{E}}
\newcommand{\Zeta}{\text{Z}}
\newcommand{\Eta}{\text{H}}
\newcommand{\Iota}{\text{I}}
\newcommand{\Kappa}{\text{K}}
\newcommand{\Mu}{\text{M}}
\newcommand{\Nu}{\text{N}}
\newcommand{\Omicron}{\text{O}}
\newcommand{\Rho}{\text{P}}
\newcommand{\Tau}{\text{T}}
\newcommand{\Chi}{\text{X}}

\DeclareMathSymbol{\alpha}{\mathord}{old_letters}{11}
\DeclareMathSymbol{\beta}{\mathord}{old_letters}{12}
\DeclareMathSymbol{\gamma}{\mathord}{old_letters}{13}
\DeclareMathSymbol{\delta}{\mathord}{old_letters}{14}
\DeclareMathSymbol{\varepsilon}{\mathord}{old_letters}{15}
\DeclareMathSymbol{\zeta}{\mathord}{old_letters}{16}
\DeclareMathSymbol{\eta}{\mathord}{old_letters}{17}
\DeclareMathSymbol{\theta}{\mathord}{old_letters}{18}
\DeclareMathSymbol{\iota}{\mathord}{old_letters}{19}
\DeclareMathSymbol{\kappa}{\mathord}{old_letters}{20}
\DeclareMathSymbol{\lambda}{\mathord}{old_letters}{21}
\DeclareMathSymbol{\mu}{\mathord}{old_letters}{22}
\DeclareMathSymbol{\nu}{\mathord}{old_letters}{23}
\DeclareMathSymbol{\xi}{\mathord}{old_letters}{24}
\newcommand{\omicron}{\text{\textit{\larger[1]{o}}}}
\DeclareMathSymbol{\pi}{\mathord}{old_letters}{25}
\DeclareMathSymbol{\rho}{\mathord}{old_letters}{26}
\DeclareMathSymbol{\sigma}{\mathord}{old_letters}{27}
\DeclareMathSymbol{\tau}{\mathord}{old_letters}{28}
\DeclareMathSymbol{\upsilon}{\mathord}{old_letters}{29}
\DeclareMathSymbol{\varphi}{\mathord}{old_letters}{39}
\DeclareMathSymbol{\chi}{\mathord}{old_letters}{31}
\DeclareMathSymbol{\psi}{\mathord}{old_letters}{32}
\DeclareMathSymbol{\omega}{\mathord}{old_letters}{33}






\begin{document}

\begin{framed}
    \begin{proposition}\end{proposition}
    При $r\br{\xi; \phi} = \sqrt{\xi^2 + \phi^2} \to 0$ имеем $\xi \phi = O\br{r^2}$. В частности, $\dfrac{\xi \phi} {r^{1 + \epsilon}} = \dfrac {\gamma\br{r} r^2} {r^{1 + \epsilon}} \to 0$ для всех $\epsilon \in \br{0; 1}$.
    
    Действительно, 
    $0 \le \br{\vbr{\xi} - \vbr{\phi}}^2$
    и 
    $\vbr{\xi \phi} 
    \le \dfrac {1} {2} \br{\xi^2 + \phi^2}
    \le \dfrac {1} {2} \br{r^2 + r^2}
    = r^2$,
    что показывает $\xi \phi = O\br{r^2}$.
\end{framed}

$\tsys{
    & P\br{x; y} = \ln\br{1 - y + y^2}
    \\& Q\br{x; y} = 3 - \sqrt{x^2 + 8y}
}$;
$\tsys{
    & y^2 - y = 0
    \\& x^2 = 9 - 8y
}$.
Итак, получаем 4 особые точки: $A_{1}\br{-3; 0}$, \linebreak $A_{2}\br{-1; 1}$, $A_{3}\br{1; 1}$, $A_{4}\br{3; 0}$.

Для оптимизации процесса разложим каждую функцию заранее. Далее будут применены замены вида $\tsys{& x = \xi_{i} + x_{i} \\& y = \phi_{i} + y_{i}}$, где $A_{i}\br{x_{i}; y_{i}}$, $1 \le i \le 4$. Предположим, что $\xi_{i} \to 0$ и $\phi_{i} \to 0$. Тогда $y^2 - y \to 0$ и $x^2 + 8y \to 9$, $O\br{\br{x - x_{i}}^k} = O\br{\xi_{i}^k} = O\br{\br{O\br{r_{i}}}^k} = O\br{r_{i}^k}$ для натуральных $k$, и аналогично для $y - y_{i}$ и $\phi_{i}$.
\begin{itemize}
    \item $\ln\br{1 - y + y^2} 
    = -y + y^2 + O\br{\br{-y + y^2}^2} = -y + y^2 + O\br{r_{i}^2}
    = -\phi_{i} - y_{i} + \phi^2_{i} + 2\phi_{i}y_{i} + \linebreak + y^2_{i} + O\br{r_{i}^2}
    = \br{y^2_{i} - y_{i}} + \br{2y_{i} - 1}\phi_{i} + O\br{r_{i}^2}
    \vspace{1mm} = \br{2y_{i} - 1}\phi_{i} + O\br{r_{i}^2}$, так как $y^2_{i} - y_{i} = 0$;
    
    \item $3 - \sqrt{x^2 + 8y} 
    = 3 - 3\sqrt{1 + \br{\dfrac {1} {9} \br{x^2 + 8y} - 1}} 
    = - 3\br{1 + \dfrac {1}{2} \br{\dfrac {1}{9} \br{x^2 + 8y} - 1} + O\br{r_{i}^2}} \linebreak + 3 = -\dfrac {1}{6} \br{x^2 + 8y} + \dfrac {3} {2} + O\br{r_{i}^2}
    = -\dfrac {1}{6} \br{x^2_{i} + 2\xi_{i} x_{i} + \xi^2_{i} + 8\phi_{i} + 8y_{i} } + \dfrac {3} {2} + O\br{r_{i}^2} \linebreak
    = -\dfrac {1}{6} \br{\br{x^2_{i} + 8y_{i}} + 2\xi_{i} x_{i} + 8\phi_{i} } + \dfrac {3} {2} + O\br{r_{i}^2} 
    \vspace{1mm} = -\dfrac {1} {3} x_{i} \xi_{i} - \dfrac {4} {3} \phi_{i} + O\br{r_{i}^2}$, поскольку $x^2_{i} + 8y_{i} = 9$. \vspace{1mm}
\end{itemize}

Итак, после всякой замены 
$\tsys{& x = \xi_{i} + x_{i} \\& y = \phi_{i} + y_{i}}$
имеем $\tsys{& \dot{\xi}\br{t} = \br{2y_{i} - 1}\phi_{i} + O\br{r_{i}^2} \\& \dot{\phi}\br{t} = -\dfrac {1} {3} x_{i} \xi_{i} - \dfrac {4} {3} \phi_{i} + O\br{r_{i}^2}}$.

\begin{enumerate}
    \item $\tsys{& x_{1} = -3 \\& y_{1} = 0}$; $\tsys{& x = \xi - 3 \\& y = \phi}$.
    $\tsys{& \dot{\xi} = -\phi + O\br{r^2} \\& \dot{\phi} = \xi - \dfrac {4} {3} \phi + O\br{r^2}}$.
    $A = \begin{pmatrix} 0 &  -1 \\ 1 & -\frac {4} {3} \end{pmatrix}$.
    $\lambda^2 - \dfrac {4} {3} \lambda + 1 = 0$.
    $\lambda_{1, 2} = \linebreak = \dfrac {-2 \mp \sqrt{5}i} {3}$, особая точка $A_{1}\br{-3; 0}$ -- <<фокус>>, причём устойчивый.
    
    Система до линеаризации выглядит, как
    $\tsys{& \dot{\xi} = \ln\br{1 + \phi^2 - \phi} \\& \dot{\phi} = 3 - \sqrt{\xi^2 - 6\xi + 8\phi + 9}}$ и $\tsys{& \dot{\xi}\br{\frac {1} {4}; 0} = 0 \\& \dot{\phi}\br{\frac {1} {4}; 0} = \frac {1} {4}}$, то есть направления закручивания -- против часовой.
    
    \item $\tsys{& x_{2} = -1 \\& y_{2} = 1}$; $\tsys{& x = \xi - 1 \\& y = \phi + 1}$.
    $\tsys{& \dot{\xi} = \phi + O\br{r^2} \\& \dot{\phi} = \dfrac {1} {3} \xi - \dfrac {4} {3} \phi + O\br{r^2}}$.
    $A = \begin{pmatrix} 0 &  1 \\ \frac {1} {3} & -\frac {4} {3} \end{pmatrix}$.
    $\lambda^2 + \dfrac {4} {3} \lambda - \dfrac {1} {3} = 0$.
    $\lambda_{1, 2} = \dfrac {-2 \mp \sqrt{7}} {3}$, особая точка $A_{2}\br{-1; 1}$ -- <<седло>>.
    $v_{1} = \begin{pmatrix} -3 \\ 2 + \sqrt{7}\end{pmatrix}$, $v_{2} = \begin{pmatrix} -3 \\ 2 - \sqrt{7}\end{pmatrix}$, а $\phi = \dfrac {-2 -\sqrt{7}} {3} \xi$ и $\phi = \dfrac {-2 +\sqrt{7}} {3} \xi$ -- сепаратрисы.
    
    \item $\tsys{& x_{3} = 1 \\& y_{3} = 1}$; $\tsys{& x = \xi + 1 \\& y = \phi + 1}$.
    $\tsys{& \dot{\xi} = \phi + O\br{r^2} \\& \dot{\phi} = -\dfrac {1} {3} \xi - \dfrac {4} {3} \phi + O\br{r^2}}$.
    $A = \begin{pmatrix} 0 &  1 \\ -\frac {1} {3} & -\frac {4} {3} \end{pmatrix}$.
    $\lambda^2 + \dfrac {4} {3} \lambda + \dfrac {1} {3} = 0$.
    $\lambda_{1} = -1$, $\lambda_{2} = -\dfrac {1} {3}$, особая точка $A_{3}\br{1; 1}$ -- <<узел>>, притом устойчивый.
    
    $A - \lambda_{1} E 
    = \begin{pmatrix} 1 & 1 \\ -\frac {1} {3} & -\frac {1} {3} \end{pmatrix}
    \sim \begin{pmatrix} 1 & 1 \end{pmatrix}$.
    $v_{1} = \begin{pmatrix} 1 \\ -1 \end{pmatrix}$.
    $A - \lambda_{2} E 
    = \begin{pmatrix} \frac {1} {3} & 1 \\ -\frac {1} {3} & 1 \end{pmatrix}
    \sim \begin{pmatrix} 1 & 3 \end{pmatrix}$.
    $v_{2} = \begin{pmatrix} 3 \\ -1 \end{pmatrix}$.
    Следовательно, $\phi = -\xi$ -- трансверсаль, $\phi = -\dfrac {1} {3} \xi$ -- касательная.
    
    \item $\tsys{& x_{4} = 3 \\& y_{4} = 0}$; $\tsys{& x = \xi + 3 \\& y = \phi}$.
    $\tsys{& \dot{\xi} = -\phi + O\br{r^2} \\& \dot{\phi} = -\xi - \dfrac {4} {3} \phi + O\br{r^2}}$.
    $A = \begin{pmatrix} 0 & -1 \\ -1 & -\frac {4} {3} \end{pmatrix}$.
    $\lambda^2 + \dfrac {4} {3} \lambda - 1 = 0$.
    $\lambda_{1, 2} = \linebreak = \dfrac {-2 \mp \sqrt{13}} {3}$, особая точка $A_{4}\br{3; 0}$ -- <<седло>>.
    $v_{1} = \begin{pmatrix} 3 \\ 2 + \sqrt{13} \end{pmatrix}$, $v_{2} = \begin{pmatrix} 3 \\ 2 - \sqrt{13} \end{pmatrix}$.
    
    $\phi = \dfrac {\sqrt{13} + 2} {3} \xi$, $\phi = \dfrac {\sqrt{13} - 2} {3} \xi$ -- сепаратрисы.
\end{enumerate}

\end{document}
